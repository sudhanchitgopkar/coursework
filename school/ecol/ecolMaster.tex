% Created 2020-12-09 Wed 15:25
% Intended LaTeX compiler: pdflatex
\documentclass[11pt]{article}
\usepackage[utf8]{inputenc}
\usepackage[T1]{fontenc}
\usepackage{graphicx}
\usepackage{grffile}
\usepackage{longtable}
\usepackage{wrapfig}
\usepackage{rotating}
\usepackage[normalem]{ulem}
\usepackage{amsmath}
\usepackage{textcomp}
\usepackage{amssymb}
\usepackage{capt-of}
\usepackage{hyperref}
\author{Sudhan Chitgopkar}
\date{\today}
\title{}
\hypersetup{
 pdfauthor={Sudhan Chitgopkar},
 pdftitle={},
 pdfkeywords={},
 pdfsubject={},
 pdfcreator={Emacs 27.1 (Org mode 9.4)}, 
 pdflang={English}}
\begin{document}

\tableofcontents

\section{12.09.20}
\label{sec:orgc61bf32}
\subsection{Wicked Problems}
\label{sec:orgb8cf567}
\begin{itemize}
\item Interconnected, complex problems that are very difficult to solve because of the multitude of opinions and effects they have
\item Scientific knowledge is reliable and durable but it is never absolute or certain
\end{itemize}
\subsection{Spheres of Influence}
\label{sec:org44e9b24}
\begin{itemize}
\item Economy -> Community -> Environment
\item Economy is a social construct that exists within the community which resides inside on an environment
\end{itemize}
\subsection{Sustainable Living}
\label{sec:org62e89da}
\begin{itemize}
\item Need to meet the needs of today without compromising the needs of future generations
\item Must be belowing carrying capacity and live a sustainable lifestyle
\end{itemize}
\subsection{Weather v Climate}
\label{sec:org03b8eae}
\begin{itemize}
\item Weather: Conditions of atmosphere at a particular time/place
\item Climate: Longer term than weather
\end{itemize}
\subsection{Keeling Curve}
\label{sec:org1a794af}
\begin{itemize}
\item Measure atmospheric CO2 from the Monnoloah observatory, showed increasing CO2 causing increased global warming and lower oceanwater PH
\end{itemize}
\subsection{Storm and Hurricane Intensification}
\label{sec:org39484ea}
\begin{itemize}
\item Climate change leading to a significant increase in weather and climate disasters
\item Such disasters are extremely troublesome because of their impact on lives and economics
\end{itemize}
\subsection{Antibiotics}
\label{sec:orgeb80ae2}
\begin{itemize}
\item Development of antibiotic resistance through natural selection
\item Overuse of antibiotics in both humans and animals
\item Natural selection results in changes in gene frequencies
\begin{itemize}
\item Some individuals are more successful at obtaining resources, are able to produce more offspring
\end{itemize}
\item Evolution: Changes in the heritable traits of a population of organisms as successive generations continue
\item Increased connectivity will increase the rate and speed at which diseases are transmitted across the world.
\end{itemize}
\subsection{Freshwater}
\label{sec:org8dabe5b}
\begin{itemize}
\item Need approx 50l. to survive per day
\item Not evenly distributed across the world
\item Found mostly in aquifers
\item Cannabis industry having a significant impact on social, cultural, economic, and environmental issues
\item 
\end{itemize}
\section{12.07.20}
\label{sec:org6900222}
\subsection{EROEI Ratio}
\label{sec:org8815d14}
\begin{itemize}
\item A measure of the net energy from an energy source
\item Energy in the source - the energy required to get it, process it, ship it, and use itself
\end{itemize}
\subsection{Wicked Problem}
\label{sec:org14c1ef7}
\begin{itemize}
\item How to best meet our future energy needs is a wicked problem
\item We need abundant, reliable energy sources to power our societies but many choices have significant problems
\item No single sustainable energy source can replace fossil fuels, but together, the wide variety of energy sources at our disposal can
\item Efforts should focus on aquiring a variety of domestic sources of energy and reducing energy needs
\end{itemize}
\subsection{Global Renewable Energy Consumption}
\label{sec:org2127f79}
\begin{itemize}
\item 57\% Hydro
\item 23\% Biomass
\item 9\% Solar
\item 7\% Wind
\end{itemize}
\subsection{Biomass Fuels}
\label{sec:org868df7c}
\begin{itemize}
\item Bioenergy: energy contained in firewood and other plant matter
\item Derived from solar energy
\item Traditional direct fuels- wood, charcoal, animal dung
\item Indirect fuels: food crops, crop waste, forestry waste
\item 3 types of energy:
\begin{itemize}
\item Heat
\item Electricity
\item Biofuels such as biodiesel
\end{itemize}
\end{itemize}
\subsubsection{Biofuel advantages:}
\label{sec:org29bcc8e}
\begin{itemize}
\item Carbon neutral
\item Lower carbon footprint than fossil fuels
\item Lower NO and SO
\item Abundant
\item Producted domestically, increasing energy independence
\end{itemize}
\subsubsection{Biofuel Disadvantages}
\label{sec:orge6d8850}
\begin{itemize}
\item Is it really carbon neutral?
\item Cost of transport processing, fertilizers, water, etc.
\item Clearing land for crops releases CO2
\end{itemize}
\subsubsection{Cellulosic Ethanol v Starch Ethanol}
\label{sec:orgbc5c9dd}
\begin{itemize}
\item Ethanol is usually made from starch found in corn and similar high surgar crops
\begin{itemize}
\item Yeasts easily ferment these sugars and produce ethanol as a by-product
\end{itemize}
\item Making ethanol from cellulose is more difficult
\begin{itemize}
\item Cellulose in grasses is not easily degraded
\item Very few species have the ability to digest it
\end{itemize}
\end{itemize}
\subsection{Wind Power}
\label{sec:orgec7175f}
\subsubsection{Advantages}
\label{sec:orgb1b961b}
\begin{itemize}
\item No fuel needed
\item No pollutants released
\item Domestic energy source
\item Not subject to disruptions of single energy plants
\end{itemize}
\subsubsection{Disadvantages}
\label{sec:org285ff96}
\begin{itemize}
\item Expensive
\item Bird and Bat deaths (may interfere with bird movement corridors)
\item Aesthetics
\item Intermittent winds
\item Windy areas not generally near urban areas
\end{itemize}
\subsubsection{General}
\label{sec:org7f77399}
\begin{itemize}
\item 2/3 of wind power in the US: Texas, Cali, Iowa, Minnesota, Washington
\item Wind Farms: clusters of turbines
\item Currently produce < 0.5\% of global electricity needs
\item \textasciitilde{}6\% of the total renewable energy
\item Fastest growing renewable sector
\item Conservation and Innovation can be profitable
\begin{itemize}
\item Denmark produces 20\% of their electricity through wind power
\item Projected to increase to 50\% by 2030
\item Denmark commands 60\% of global makret for wind turbines
\end{itemize}
\end{itemize}
\subsection{Hydropower}
\label{sec:org5a7ab0e}
\begin{itemize}
\item Capturing the energy of water moving under the force of gravity
\item Supplies more lectricity than any other single renewable resource
\item Generates 7\% of electricity in the US, 17\% around the world
\item Most hydroelectric power generated by large plants within dams
\end{itemize}
\subsubsection{Advantages}
\label{sec:org288992c}
\begin{itemize}
\item Fossil fuels not needed
\item No pollutants
\item Energy cheap to produce
\item Power produced can be matched to demand
\item Dams also used for flood control
\item Water retention
\end{itemize}
\subsubsection{Disadvantages}
\label{sec:org612d848}
\begin{itemize}
\item Effect on environment and human communities
\item Ecosystem alteration and flow changes
\item Peak flows no longer seasonal
\item Sediment accumulation and transport altered
\item Temperature and oxygen in stream changed
\end{itemize}
\subsubsection{General}
\label{sec:orgd37090a}
\begin{itemize}
\item Countries producing most hydroelectric power: China, Brazil, Canada, US, Russia, Norway, India (in order highest to lowest)
\end{itemize}
\subsection{Solar Energy}
\label{sec:orgc805904}
\begin{itemize}
\item Photovoltaic (PV) Cell
\item Small fraction of power, although PV cell production growing by 25\% per year
\item Currently inefficient energy storage mechanism
\item Low rainfall and limited cloud cover make the SW US an ideal region for solar power usage
\end{itemize}
\subsubsection{Advantages}
\label{sec:orga144a7b}
\begin{itemize}
\item Free and renewable
\item No emissions
\item Roof installations/More space options
\item Jobs in the sector increasing 35\%/year
\end{itemize}
\subsubsection{Disadvantages}
\label{sec:orgc888353}
\begin{itemize}
\item High start up costs (10,000-20,000 for home to save hundreds of dollars per year)
\item High cost of electricity compared to coal produced electricity
\item Daylight requirements
\item Geograohical limitations: solar energy unevenly distributed
\item Hazardous materials used in construction
\end{itemize}
\subsection{Geothermal Energy}
\label{sec:orgc7dc851}
\begin{itemize}
\item Heat energy generated and stored in Earth
\item Energy does not originate in the sun
\item In a dry-steam power plant, underground hot water and steam used to generate electricity
\end{itemize}
\subsubsection{Advantages}
\label{sec:org7cbd9b4}
\begin{itemize}
\item Reliable and inexpensive
\item Pollution free
\item Uses 30-70\% less energy than conventional methods
\end{itemize}
\subsubsection{Disadvantages}
\label{sec:orgfa1907a}
\begin{itemize}
\item Ecosystem disturbance
\item Replenishment rates of ground heat
\item High startup costs
\item Geographic limitations
\end{itemize}
\subsection{Ocean Energy}
\label{sec:org3e52a85}
\begin{itemize}
\item Using kinetic energy of tides and waves
\end{itemize}
\subsubsection{Advantages}
\label{sec:orge353dfb}
\begin{itemize}
\item Renewage and pollution free
\item Low cost of production
\end{itemize}
\subsubsection{Disadvantages}
\label{sec:org89fd3bf}
\begin{itemize}
\item Ecosystem disturbance
\item Tidal barrage systems need high tide amplitude
\item Geographic limitations
\end{itemize}
\subsection{The Path Forward}
\label{sec:org21fbacd}
\begin{itemize}
\item One recurring theme with renewable enrgy is that they are more expensive than fossil fuels
\item Important to ask why fossil fuels are so cheap, need to consider all costs of these technologies
\item When true cost is factored in, fossil fuels have highest external costs
\item No single fuel can replace fossil fuels but a wide variety of options are available and considering more renewable sources is an important part of the solution
\item More likely to meet energy needs with a wide variety of renewable technologies that fit each locale
\end{itemize}
\section{12.04.20}
\label{sec:org36447c1}
\subsection{Locust Outbreaks}
\label{sec:orgd7d097d}
\begin{itemize}
\item What triggers locust outbreaks?
\begin{itemize}
\item Locust outbreaks are caused by excessive rain, contributing to a moist area
\item Recent cyclones have contirbuted to very high amounts of recent rainfall
\end{itemize}
\item Why are they so devastating?
\begin{itemize}
\item A single locust can eat their own body weight (2g) per day, multiplied by the millions of locusts that comprise a single swarm
\end{itemize}
\item What can be done to help control outbreaks?
\begin{itemize}
\item Loud noises and preventing the locusts from laying their eggs is critical
\item Pesticide application and swarm location identification through an application
\end{itemize}
\end{itemize}
\subsection{Small Scale Farming and Food Security}
\label{sec:org81b389f}
\begin{itemize}
\item Progress requires creating opportunities for innovation and entrepreneurship
\item Increased public research and extension investment
\item Small scale sustainability poses challenging policy choices: payment for ecological services, decentralized governance systems, and technology choices
\end{itemize}
\subsection{Economics and the Environment}
\label{sec:org2ebfa6d}
\begin{itemize}
\item Economics isn't just about money, it is also about the environmental resources we depend on
\begin{itemize}
\item Economics: the social science that deals with the production, distribution, and consumption of goods and services
\end{itemize}
\item Environmental resources include ecosystem services
\begin{itemize}
\item Ecosystem services: essential ecological processes that make life on Earth possible
\end{itemize}
\item Environmental economists argue that mainstream economics will fail in the long run because it makes some assumptions that are inconsistent with the way nature operates.
\item When the price of a product doesn't reflect the external social and environmental costs, those costs are paid by people other than the consumer. Internalizing external costs better reflects the true cost of a product
\item Many US industries can't exist without illegal immigrants
\item Undocumented immigrants make up more than half of hired labors on farms, 15\% of construction workforce, 5\% of total workforce
\end{itemize}
\subsection{The Great Climate Migration}
\label{sec:org9995a23}
\begin{itemize}
\item As climate change effects become more significant, more people will be required to move away from their homes
\item Climate change also uniquely harms agricultural productivity
\end{itemize}
\subsection{Summary}
\label{sec:orga7b848d}
\begin{itemize}
\item The Green revolution supported exponential increases in global population. Temporal changes in human behavior, production, and pollution are asociated with the Green Revoluution
\item There are different ways to approach agricultural production
\item Human migration patterns, economic development, and agricultural production are inextricably linked
\end{itemize}
\subsection{Energy, Coal, Oil, and Gas}
\label{sec:orgb1b3934}
\begin{itemize}
\item Forms of hyrdocarbons
\begin{itemize}
\item Fossil fuels: nonrenewable resources like coal, oil, and natural gas that were formed over millions of years from the remains of dead organisms
\item Forms the majority of the energy used by modern society
\item Crude oil (liquid). natural gas (gas), coal (solid)
\end{itemize}
\end{itemize}
\subsection{Electricity Generation}
\label{sec:orgcbc65c8}
\begin{itemize}
\item US: 33\% coal, 33\% natural gases, 19\% nuclear, 7\% renewable, 6\% hydroelectric
\begin{itemize}
\item Drawbacks; greenhouse gases, acid rain, acid mine drainage, particulates
\end{itemize}
\item EROEI of Coal
\begin{itemize}
\item Energy return on energy investment (EROEI): A measure of the net energy from an energy source
\item Energy in the source - Energy required to get, process, ship, and use it
\item EROEI of coal = 11.5:1
\end{itemize}
\end{itemize}
\subsection{Coal Formation}
\label{sec:org779a3ed}
\begin{itemize}
\item Process of coal formation occurs best in river deltas/coastal plains
\item Formation of Appalachian Mountains were the last and greatest of the events that joined all of the world's land into one super-continent
\item Coal is mined from underground or surface mines (now accounts for 60\% of coal produced in the US)
\begin{itemize}
\item Top coal producing states: Wyoming > Kentucky > W Virgina
\end{itemize}
\item Surface mining: Occurs at depts above 180 ft
\begin{itemize}
\item Strip mining
\item Open pit mining
\item Mountaintop removal mining
\item Highwall mining
\end{itemize}
\item Underground mining: Below 300 ft, accounts for 60\% of world coal production
\end{itemize}
\subsection{Coal Transportation}
\label{sec:org4e58bc5}
\begin{itemize}
\item shipped by train or barge to destination
\item may be refined before shipping (washing w water or chemical bath to remove impurities)
\item when coal arrives at the power plant, it is pulverized into a heavy powder that is suitable for burning
\end{itemize}
\subsection{Advantage of Coal}
\label{sec:orgde65be0}
\begin{itemize}
\item Widely distributed around the world
\item Proven reserves are est to last for 100 more years
\item Easier and sager to transport compared to other fossil fuels
\item Affordable
\end{itemize}
\subsection{Disadvantages of Mountaintop Removal}
\label{sec:org27008d2}
\begin{itemize}
\item Higher rates of cancer, respiratory diseases amongst miners and local communities
\item Loss of forest and compaction of soil
\item Explosions send rock and debris flying
\item Toxic substances permeate region's rivers
\item Coal extraction creates toxic slurry
\item Toxic fly ash created
\end{itemize}
\section{12.02.20}
\label{sec:org6b52cd4}
\subsection{Provisioning Services}
\label{sec:org809760d}
\begin{itemize}
\item Products obtained from ecosystems
\item Pharmaceuticals
\item Fruits and vegetables
\item Fish and game
\item Timber, fiber, fuels
\end{itemize}
\subsection{Plants as food}
\label{sec:org6becc42}
\begin{itemize}
\item 400,000 species described
\item 14 species provide most of our food
\item 3 crops (corn, wheat, rice) account for half of the world's calorie consumption
\end{itemize}
\subsection{The Green Revolution}
\label{sec:org428d862}
\begin{itemize}
\item Normal Bourlaug - Father of the Green Revolution
\begin{itemize}
\item Selective breeding increased crop yields throughout the world in the 1960s
\item Won many different prizes (National Medal of Science, Congressional Gold Measure, Public Welfare Medal, Nobel Peace Prize, Presidential Medal of Freedom)
\end{itemize}
\item Shortly after the 20's and the rise of higher crop yields, the dust bowl and great depression hurt agriculture significantly
\item The 1960s Green Revolution increased food supplies worldwide
\end{itemize}
\subsection{The Basis of the Green Revolution}
\label{sec:org1f179d3}
\begin{itemize}
\item Nitrogen and phsphorus fertilizer, irrigation and pesticide use are unsustainable
\end{itemize}
\begin{itemize}
\item Production increased during the green revolution even though the area under cultivation remained approx. the same
\item Production leveld off in the 1980s, only to rise again at the turn of the century, where genetic engineering technologies were used to develop new crop varieties
\end{itemize}
\subsection{Feeding the Global Population}
\label{sec:orga136136}
\begin{itemize}
\item Global pop expected to reach 10B by 2050
\item Experts say we will need to produce twice as much food
\item Eventually crop productivity reaches a limit
\end{itemize}
\subsection{Green Revolution Impacts, Limits, Future}
\label{sec:org4f761e7}
\begin{itemize}
\item Povery and food insecurity persisted despite advances
\item Nutrition: Calorie availability increases but micronutrient intake is lagging
\item Environmental impacts have been mixed
\begin{itemize}
\item Positive: Less overall land used
\item Negative: Increased pesticide use
\end{itemize}
\item Gains in Africa lag significanrlt but are catching up
\item Green Revolution 2.0
\end{itemize}
\subsection{Importance of Food Self-sufficiency}
\label{sec:org415d2e1}
\begin{itemize}
\item Food self-sufficiencyL ability of an individual nation to grow enough food to feed its people
\begin{itemize}
\item Africa largely overlooked by green revolution, lack food self-sufficiency
\end{itemize}
\item Food sovereignty: ability for an individual nation to control its own food system
\begin{itemize}
\item Africa also lacks food sovereignty
\end{itemize}
\end{itemize}
\subsection{Africa \& Hunger}
\label{sec:org90352b9}
\begin{itemize}
\item Industrialization and farm subsidies enabled mostly US farmers to produce vast surpluses of what, corn, and soybeans
\item Global market flooded w cheap food
\item Smaller countries can't compete
\item Much of their farmable land used for cash crops, which are exported
\begin{itemize}
\item Cash crops: Food and fiber crops grown to sell for profit rather than for use by local families, communities
\end{itemize}
\end{itemize}
\subsection{Industrial Agriculture: Pros and Cons}
\label{sec:org5dacb29}
\begin{itemize}
\item Pros
\begin{itemize}
\item Large scale farming
\item Higher yields
\item Growth in nutrient-poor soil
\item Fewer blemishes
\item Less labor intensive
\end{itemize}
\item Cons
\begin{itemize}
\item Dependence on mechanization
\item Monocultures
\item Decrease in biodiversity
\item 40\% of all agriculture consumed by pests/disease
\end{itemize}
\end{itemize}
\subsection{Locust Outbreaks}
\label{sec:org7073ff8}
\begin{itemize}
\item One of the world's most devastating pests
\item Can swarm to 80M, eat 2+ grams/day
\end{itemize}
\section{11.30.20}
\label{sec:orgad1f81f}
\subsection{Human Impacts on the Phosphorus Cycle}
\label{sec:org3477dff}
\begin{itemize}
\item inc fetrilizer use increases phosphorus runoff into waterways, increases eutrophication
\item all farmers need phosphorus, but 5 countries control \textasciitilde{}85\% of world's remaining phosphate reserves
\item phsophorus critical to fertility, high crop yields, overall necessary to food production
\item phosphorus quickly becoming more expensive
\end{itemize}
\subsection{Major Issues Assiciated with Current Phosphorus Usage}
\label{sec:org10db85a}
\begin{itemize}
\item Inefficient global food system
\begin{itemize}
\item 4/5 of phosphorus is lost/wasted in the supply chain from mine to field to fork
\end{itemize}
\item Cheap fertilizer
\begin{itemize}
\item Farmers need access to phosphorus, yet up to 1B farmers lack access to fertilizer markets
\end{itemize}
\item No Monitoring
\begin{itemize}
\item Currently no international or national policies, guidelines, or organizations responsible for ensuring long term availability and accessibility of phosphorus of food production
\end{itemize}
\end{itemize}
\subsection{Phosphate Prices}
\label{sec:orgd559502}
\begin{itemize}
\item temporary phosphorus price spikes in 2008 affected farmers from Australia to Ethopia, leading to farmer riots and suicides - was a wake up call to the fragility of the world food sitation
\end{itemize}
\subsection{Sustainable Phosphorus Measures}
\label{sec:org7bc83b9}
\begin{itemize}
\item Much of phosphorus is lost in waste, can be recycled and recouped
\item Critical to maintain a more sustainable diet
\end{itemize}
\subsection{Soil \& Soil Dynamics}
\label{sec:orge06041d}
\begin{itemize}
\item Critical Concepts:
\begin{itemize}
\item Physical and Chemical Weathering
\item Erosion
\item How long do soils take to form?
\item Soil loss \& conversion
\end{itemize}
\end{itemize}
\subsection{Physical and Chemical Weathering}
\label{sec:orgf5b2e1d}
\begin{itemize}
\item Weathering = breakdown of rocks, soil, minerals often done by water or organisms
\begin{itemize}
\item Physical weathering = simply breaking down rocks
\item Chemical weathering - changing the chemical structure of rock, soil, etc
\end{itemize}
\item Makes up \textasciitilde{}50\% of the soil
\item Takes an extremely long time
\end{itemize}
\subsection{Soil Complexity}
\label{sec:org3bca555}
\begin{itemize}
\item Diverse organisms contribute to soil structure
\item Soil is often linked between the different spheres of the world and supports biodiversity significantly
\item Climate effects soil in different ways
\begin{itemize}
\item Tropical forest: generally infertile due to heavy rainfall and decomposition rates
\item Temperate forest: usually fertile and neutral to slightly aciding
\item Temperate grassland: neutral to slightly basic, high to moderate fertility
\item Desert: low fertility and organic magter, often high in salt content
\end{itemize}
\item It can take 500+ years for 2.5 cm of soil to form
\item Poor land use practices cause soil erosion
\end{itemize}
\subsection{Weathering vs Erosion}
\label{sec:org9c68abf}
\begin{itemize}
\item Weathering: the breakdown of rock to soil (creating soil)
\item Erosion: the movement of soil, minerals, etc to another place (taking away soil)
\item Overgrazing: removal of plant matter by livestock exposes soil to erosion
\item Deforestation exposes soil to loss
\item Desetification: Semiarid range land becomes permanently less productive when overgrazed
\end{itemize}
\subsection{Summary}
\label{sec:orgad3b6e5}
\begin{itemize}
\item There are biotic and abiotic components of ecosystems
\item Global biogeochemical cycles are essential to supportiong biology and the economy
\item Phosphorus is especially important. It is distributed heterogeneously and limitation of it may compromise agricultural production. An excess of phosphorus is a major quatic pollutant
\item Access to elements will govern patterns in economic and technological development
\item Soils are complex and take extremely long times to form. Soils are strongly influenced by climate.
\item A variety of activities compromise soils.
\end{itemize}
\section{11.23.20}
\label{sec:org30169fe}
\subsection{Tomato Agriculture}
\label{sec:org9018a5b}
\begin{itemize}
\item Tomato is one of the most valuable crops in the world
\item Originally from S America, transported to Europe by early 17th century, back to N America in 18th century
\item Annual production > 175M tons, 85B USD
\item Important plant model
\begin{itemize}
\item 15,000+ known varieties
\item Member of important Solanaceae family
\end{itemize}
\end{itemize}
\subsection{Ecosystems}
\label{sec:orga991697}
\begin{itemize}
\item The provisioning of ecosystem services is dependent upon functioning ecosystems
\item Biosphere: the total area on Earth where living things are found; the sum total of all biomes
\item Ecosystem: all of the organisms in a given area + the physical environment in which, and with which, they interact
\item Community: all the populations (plants, animals, other species) living and interacting in an area
\item Population: all the individuals of a species that live in the same geographic area and are able to interact and interbreed
\item Ecosystem ecologists: study how ecosystems work in relation to their biotic and abiotic components
\item Population ecologists: study how populations change over time and space
\item Community ecologists: investigate the factors that influence biodiversity, community structure, and the distribution and abundance of specie
\end{itemize}
\subsection{Energy and Matter}
\label{sec:orgc8162f2}
\begin{itemize}
\item All ecosystems function through:
\begin{itemize}
\item Matter cycles: movement of life's essential chemicals/nutrients through an ecosystem
\item Energy flow: the one-way passage of energy through an ecosystem
\end{itemize}
\item Earth is materially closed but energetically open
\item Biomass can't enter or leave the system, but energy can
\item Energy enters as sunlight through either heat or light
\item Photosynthesis: the chem eraction done by producers to convert energy of the sun using carbon dioxide and water -> sugar and oxygen
\end{itemize}
\subsection{Biomes}
\label{sec:orga7ccfd6}
\begin{itemize}
\item Biomes: specific portions of the biosphere determined by climate and identified by the predominant vegetation and organisms adapted to live there
\item Biomes are divided into three main categories
\begin{itemize}
\item Terrestrial
\item Marine
\item Freshwater
\end{itemize}
\end{itemize}
\subsection{Limiting Factors \& Distribution}
\label{sec:org18b44be}
\begin{itemize}
\item Limiting factor: the critical resource whose supply determines the population size of a given species in a given ecosystem
\item Range of tolerance: the rande, within upper and lower limits, of a limiting factor that can limit population size
\item Limiting factors determine the distribution and size of populations
\item Variability increases a population's range of toleration, expanding its distributoin and increasbing the chance that it will be able to adapt to changing conditions
\end{itemize}
\subsection{Review}
\label{sec:org0fcd097}
\begin{itemize}
\item Food insecurity is global and heterogeneous
\item There are 4 types of ecosystem services
\item Biophysical charactersistics of systems govern what and how much of a crop can be produced into a region
\end{itemize}
\subsection{Matter}
\label{sec:orge1e9a38}
\begin{itemize}
\item Matter cycles that move nutrients through ecosystems depend on living organisms and abiotic sinks of those resources
\item Biotic: the living, organic components of an ecosystem
\item Abiotic: the non-living compoenents of an ecosystem, important for nutrient cycling
\item BiogeochemistryL the ways in which an element or compound moves between its various living and nonliving forms and locations in the bioshphere
\item Elements required for life: Carbon, Hydrogen, Ditrogen, Oxygen, Phosphorus, Sulfur
\item Biogeochemical Cycles:
\begin{itemize}
\item Water cycle
\item Carbon cycle
\item Nitrogen cycle
\item Phosphorus Cycle
\end{itemize}
\end{itemize}
\subsection{Carbon Cycle}
\label{sec:org863f0d8}
\begin{itemize}
\item Photosynthesis: Plants and other photosynthetic organisms produce sugars
\begin{itemize}
\item Utilize sunlight and CO2, produce oxygen
\item Photosynthesizers known as producers
\end{itemize}
\item Cellular respiration: Organisms break down sugar to release energy
\begin{itemize}
\item Utilizes oxygen, produces CO2, known as consumers
\end{itemize}
\end{itemize}
\subsection{Nitrogen Cycle}
\label{sec:org01f0888}
\begin{itemize}
\item Nitrogen is the most abundant gas in the air but has a very tight bond, bond must be broken (fixed) before use by producers
\item Nitrogen cal also be released back into the atmosphere
\item Nitrogen Cycle: a continuous series of natural processes by which nitrogen passes from air -> soil -> organisms -> air/soil
\item Nitrogen enters ecosystem through nitrogen fixation
\item Nitrogen exits the ecosystem when other bacteria convert nitrate back to molecular Nitrogen
\item Humans are disrupting the nitrogen cycle
\begin{itemize}
\item Fertilizers and emissions are doubling availbility of nitrogen
\item While nitrogen is no longer a limitng factor for plant growth, the additional notrigen can disrupt the ecosystem
\end{itemize}
\end{itemize}
\subsection{Phosphorus Cycle}
\label{sec:orga5a6334}
\begin{itemize}
\item Different than other cycles because phosporus doesn't exist in the atmosphere
\item Only found in solid, liquid form
\item Phosphorus cycle: a series of natural processes by which phosphorus moves from rock -> soil, water -> living organisms -> soil
\end{itemize}
\section{11.20.20}
\label{sec:orge6cad35}
\subsection{What does it take to grow a tomato?}
\label{sec:orgad0cd63}
\begin{itemize}
\item Nursery (Water, energu, materials, presicides, etc)
\item Nursery to farm transport
\item Cultivation (Pesticides, land use, tillage, etc)
\item Farm to Packaging-House transport
\item Packaging (Energy, water, packaging)
\item Transportation (Freight ship route, truck route)
\end{itemize}
\subsection{Ecosystem Services}
\label{sec:org9733b6e}
\begin{itemize}
\item Provisioning Services (Prodicts obtained from the ecosystem)
\item Regulating Services (Benefits from regulation of ecosystems)
\item Cultural Services (non-material benefits obtained from ecosystems)
\item Life on earth depends on ecosystem services provided by natures
\item Recognizing the value of rhese services may motivate us to protect them
\end{itemize}
\subsection{Nutrition}
\label{sec:orgf7eba7f}
\begin{itemize}
\item Currently produce 1/3 more calories than needed
\item UN 2013, 842M people (12\% of the world) suffers from undernutrition (not enough calories)
\item Civil war and Climate change contribute to a significant increase in recent famine
\item Although we produce enough food to feed everyone, nearly 1B people don't have access to enough nutritious food
\item The rise of industrial agriculture and the Green Revolution helped fight hunger in the 20th century but came w some unintended consequences
\item Employing a variety of agricultural methods and addressing socioeconomic drivers of poverty necessary to fight hunger
\end{itemize}
\subsection{Food Security}
\label{sec:org1c9e0d4}
\begin{itemize}
\item Food security: having enough physical, social, and economic access to sufficient safe and nutritious food
\item Food insecurity is a problem due to
\begin{itemize}
\item Inadequate distribution of food
\item Inadequate funds to buy food
\end{itemize}
\item Undernourishment: When a person does not have enough to eat
\item Worldwide, 1/4 children experiences stunted growth due to undernutrition
\item Malnutrition: a state of poor health that results from a nutritional imbalance due to a lack of essential nutrients
\begin{itemize}
\item can serve as a prelude to many duseases
\item UN est that the cost of treating malnourishment in children under 2 is double of the cost to prevent it in the first place
\end{itemize}
\item Overnutrition: the consumption of too many calories
\begin{itemize}
\item considered a form of malnutrition
\item affects 1.5B people
\item increases susceptibility to diseases
\item problem of both the wealthy and poor
\end{itemize}
\item Protein deficiency -> Kwashiorkor
\item Calorie and protein deficiency -> wasting disease
\item Vitamin deficiency -> many diseases
\end{itemize}
\subsection{Food Deserts}
\label{sec:org123301b}
\begin{itemize}
\item 13/30 of athens census tracts are labeled as food deserts, 33\% of residents live 1+ mile from a grocery store
\end{itemize}
\section{11.11.20}
\label{sec:org45c0e5c}
\subsection{Cannabis \& Sustaibaility}
\label{sec:org456cad7}
\begin{itemize}
\item Now that cannabis legalization is sweeping North America, we need to better understand its impact on freshwater systems
\item 2/3 of Americans believe that marijuana should be legalized
\end{itemize}
\subsection{Cannabis \& The Economy}
\label{sec:org16635c5}
\begin{itemize}
\item Cannabis may be key to economic recovery, potentially post COVID, similar to how ending prohibition helped end the Great Depression
\item 10s to 100s of Millions made off of Marijauna tax revenue
\item California is the biggest producer with nevada as the runner up for marijuana
\end{itemize}
\subsection{Cannabis \& Society}
\label{sec:orgc760328}
\begin{itemize}
\item Many states are no decriminalizing Marijuana and allow for medical Marijuana use
\item Without legalization, marijuana feeds non-violent offenders into the prison system, perpetuates mass incarceration, and disproportionately affects POC
\item California was the first state to allow medical use of Marijuana, many states have created laws since then
\end{itemize}
\subsection{Cannabis and the Environment}
\label{sec:org67651f9}
\begin{itemize}
\item California case study, Pot takes up very significant amounts of water, no regulation
\item Groundwater use has triggered conflicts across areas of California
\item Water rights are a large concern in the Marijuana industry, especially for California because unlicenced growers often steal other's water
\item US DEA est. that 60\% of cannabis consumed nationwide is grown in California
\item Bulk of that comes from three upstate counties of the Emerald Triangle: Mendocino, Humboldt and Trinity.
\item This is because the conditions there are perfect for Cannabis growth but this comes with problems for the environment, waterways, and wildlife
\item Creek Diversions threaten fish habitats
\item Road building erodes soil, streams
\item 1 marijuana plant growing in a national forest uses 900 gallons of water per growing season
\item In 2017, 1.25 Million plants were found growing in CA national parks
\item Illegal marijuana growth therefore uses 1.1 Billion gallons of water
\end{itemize}
\subsection{Tristate Water Wars}
\label{sec:org2a08ebe}
\begin{itemize}
\item For 30 years, GA, AL, FL have fought over the sue of water in the Apalachicola-Chattahoochee- Flint River Basin (ACF) which is heavily infleunced by the US Army Corps of Engineers' operation of Lake Lanier's Buford Dam. Lanier lies within Chattahoochee's headwaters, north of Atlanta
\item 70: the number of attorneys on retainer by GA
\item 4 Million: Pages of documents produced by GA agencies, universities and non-profits requested by FL.
\item 660,000 emails give to GA by FL
\item 45 people deposed by both GA and FL
\end{itemize}
\section{11.02.20}
\label{sec:orgaa3e1eb}
\subsection{Triple Bottom Line}
\label{sec:orgb0b5374}
\begin{itemize}
\item An assessment of the cost of a good or service should include more than just the economic costs; it should also include the social and environmental cost
\item IPAT Equation:
I = P * A * T; I = Impact, P = Population size, A = Affluence(products/person), T = Tech Usage (impact/product)
\end{itemize}
\subsection{Assumptions of Mainstream Economics}
\label{sec:orgb878b4d}
\begin{itemize}
\item Environmental economists argue that mainstream economics will fail in the long run because it makes some assumptions that are inconsistent with the way nature operates
\end{itemize}
\begin{enumerate}
\item Assumption:
\label{sec:org56cdcb2}
\begin{itemize}
\item Natural and human resources are infinite, substitutes can be found as necessary
\item Economic growth will go on forever
\item Something that benefits/harms us today is more important than something that ight do so tomorrow
\end{itemize}
\item Impacts:
\label{sec:orge7c6557}
\begin{itemize}
\item Linear economic production models use inputs and produce waste without regard to sustainability; circular systems depend on renewable resources and see waste as a useful inp
\item Cradle to Cradle mentality creates sustainability whereas crade to grave increase the amount of overall waste
\end{itemize}
\end{enumerate}
\subsection{Market solutions}
\label{sec:orgbb199d9}
\begin{itemize}
\item Alternatve: Command and Control
\begin{itemize}
\item Command = estbalishment of performance standards by a govt authority that must be complied with
\item Control = negative consequences that could result from non-compliance
\end{itemize}
\item Performance Standards
\item Tradeable permits
\begin{itemize}
\item Important to consider the effect on environmental justice
\end{itemize}
\end{itemize}
Economic Incentives
\begin{itemize}
\item Seek to reduce or eliminate negative environmental externalities (such as pollution) by incorporating the external cost of production.
\item The general focus is prevention rather than remediation
\end{itemize}
\begin{itemize}
\item Payment for Ecosystem Services
\begin{itemize}
\item NYC protecting its water supply
\end{itemize}
\end{itemize}
\subsection{Environmental Policy}
\label{sec:org5f75faf}
\begin{itemize}
\item Environmental policy = A course of action adopted by a government or organization intended to improve the natural environment and public health and reduce human impact on the environment
\item Collective action undertaken to manage natural resources and human impacts on the environment.
\item Things like:
– Laws
– Regulations
– International agreements
– Funding decisions
\end{itemize}
\subsection{Why is Environmental Policy Challenging}
\label{sec:org0542174}
\begin{itemize}
\item Many environmental problems trasncend boundaries
\item Lots of WICKED problems, very complex with mulitple stakeholders
\item Lawmakers must juggle many factors
\begin{itemize}
\item Effectiveness of the policy
\item Negative tradeoffs
\item Cost burden (internal, external costs)
\item Flexibility of the policy to accomodate changes
\end{itemize}
\item Many times, voters and lawmakers don't agree that they are necessary
\end{itemize}
\subsection{History of Environmental Policy}
\label{sec:org85d263d}
\begin{itemize}
\item Before 1960's
\begin{itemize}
\item How best to use resources
\item Pollution not key objective
\item Primerily dealt with at the state level
\item Environmental problems addressed after the fact through litigation, favored the pollutor
\end{itemize}
\item Changes
\begin{itemize}
\item As industry, pollution inc, pollution crossed state lines
\item Massive outcry in the 60's and 70's lef to federal legislation
\item Performance standards let to a prevention-focused regulation
\end{itemize}
\end{itemize}
\subsection{Who Makes Environmentla Policy?}
\label{sec:org824913a}
\begin{itemize}
\item Elected Officials
\item Federal and State Agencies
\item Local departments: planning and zoning, public works, etc/
\item Courts
\item Corporations and other businesses
\end{itemize}
\subsection{NEPA}
\label{sec:org93bfda3}
\begin{itemize}
\item NEPA’s key feature is the Environmental Impact Statement (EIS)—a report that details the likely impacts (positive and negative) of a proposed action.
\item The goal of an EIS is to identify problems before they occur so that stakeholders can choose the most acceptable course of action.
\item The findings are made available to everyone (citizens, policy makers, and special interest groups)—this keeps the process transparent and everyone is given a chance to respon
\end{itemize}
\subsection{Policy Decision Making Process}
\label{sec:org9dfba1a}
\begin{itemize}
\item Identify problem -> Consider options -> Formulate Plan -> Adopt Law -> Implement Law
-> Evaluate effectiveness
\item Statutes:
\begin{itemize}
\item Provide policies, goals
\item Typically mandate an agency to promulgate regulations according to staturoy standards and enforce them
\item Often authorize states to enforce them
\item Often dictate funding allocations
\end{itemize}
\item Regulations:
\begin{itemize}
\item Regulation = rule = administrative law
\item The actual technical and programmatic standards for environmental protection
\item Standards usually in regulagtions instead of statutes because of ease of amendment
\end{itemize}
\item Court Decisions
\begin{itemize}
\item Rule on constitutionality of statute, regulation, or other deferal action
\item Rule on application of statue or regulation
\item Rule on meaning (language/intent)
\end{itemize}
\item Executive Orders
\begin{itemize}
\item Presidential directives to do something
\item Often involve internal affairs, Development of amendments to regulations
\end{itemize}
\end{itemize}
\subsection{Misc}
\label{sec:orgaec336b}
\begin{itemize}
\item Most environmental regulation passed between the 70's and 90's, no significant regulation since
\item Enforcement and Definitions absolutely essential
\item Trump and Environmental Policy
\begin{itemize}
\item Treaties
\end{itemize}
– Paris Climate Agreement
\begin{itemize}
\item Agency heads, federal judges
\item Agency directives –rules/regulations
\item Rule rollbacks
\item No new rules or policies
\item More state authority
\end{itemize}
\end{itemize}
\section{10.26.20}
\label{sec:orge68d662}
\begin{itemize}
\item Disease cases frim infected mosquitoes, ticks, and fleas have tripled in the last 13 years
\end{itemize}
\subsection{Malaria}
\label{sec:org1ed1891}
\begin{itemize}
\item Vector: Mosquito
\item Transmission: Bite from infected mosquitoes
\item Prevalence: Est 219M cases of Malaria, cases are mostly children w 660k Deaths
\item US Prevalence: An average of 1,500 reported cases of malaria in the U.S. each year
\end{itemize}
\subsection{Dengue Gever:}
\label{sec:orgbc57802}
\begin{itemize}
\item Vector: Asian tiger mosquito (in 36 US states)
\item Transmission: Bite from infected mosquito
\item Prevalence: 100M cases worldwide, endemic in the Americas
\item Occurs rarely, but there is a small risk for dengue outbreaks in the continental United States, mainly in the Southern US
\end{itemize}
\subsection{Chikungunya}
\label{sec:org246dd4a}
\begin{itemize}
\item Transmitted by mosquitoes
\item Mainly in Africa, Asia, Europe, Indian, and Pacific Oceans
\item First found in the Americas on Carribean islands in 2013
\item Beginning in 2014, reported in US travelers
\end{itemize}
\subsection{West Nile Virus}
\label{sec:orgbb46ca3}
\begin{itemize}
\item Vector: Mosquito
\item Transmission: Bite form infected mosquito
\item Prevalence: commonly found in Africa, Europe,Middle East, North America, West Asia
\item U.S. Prevalence: Between 1999 and 2012, about 37,000 cases of West Nile Virus were reported in the U.S. Over 1,500 people died as a result.
\end{itemize}
\subsection{Spread of Disease}
\label{sec:org969a9c9}
\begin{itemize}
\item Increased connectivity increases rate and spread of infectious diseases across the globe
\item Correlation between travel advisory and amount of travel to infected areas for Zika
\item Zika most likely to be found in the Southeast because of Zika-transmitting mosquito population residence
\item High poverty rates correlated with high risk of disease spread due to high population density, potential lack of good healthcare
\item Warmer average temps, longer growing seasons, changes in precipiation may lead to more standing water and conditions that may be better for disease spread
\item Warning temps could expose more than 1.3B people to Zika by 2050
\end{itemize}
\subsection{Climate Change and Health}
\label{sec:orgdadcc49}
\begin{itemize}
\item Without effective responses, climate change will:
\begin{itemize}
\item Water quality and quantity:  Contributing to a doubling of people living in water-stressed basins by 2050.
\item Food security: In some African countries, yields from rain-fed agriculture may halve by 2020.
\item Control of infectious disease: Increasing population at risk of malaria in Africa by 170 million by 2030, and at risk of dengue by 2 billion by 2080s.
\item Protection from disasters: Increasing exposure to coastal flooding by a factor of 10, and land area in extreme drought by a factor of 10-30
\end{itemize}
\item Rainfall: tranports and disseminates infectious agents
\item Flooding: sewage treatment plants overflow, water sources contaminated
\item Sea levels rise: Increased risk of severe flooding
\item Higher temps: increases growth and survival rates of infection
\item Drought: increases concentration of pathogens, hurts hygiene
\end{itemize}
\subsection{Health Outcomes from Climate Change}
\label{sec:orgbe469ef}
\begin{itemize}
\item Some expected impacts will be beneficial but most will be adverse.Expectations are mainly for changes in frequency or severity of familiar health risks
\item See Zika Climate Final for diagrams
\end{itemize}
\subsection{Poverty and Disease}
\label{sec:orgef5afd5}
\begin{itemize}
\item Diarrhea is related to temperature and precipiatation; Diarrhea increased 8\% for each 1 degree C temp increase
\item Health impacts of climate change unfairly distributed, hurt mortality of developing, low-income countries, especially in Africa
\end{itemize}
\subsection{Temperature Effects on Vectors and Pathogens}
\label{sec:org0b3fced}
\begin{itemize}
\item Vector:
\begin{itemize}
\item Survival inc/dec depending on species
\item Changes susceptibility of vectors to some pathogens
\item Changes in rate of vector population growth
\item Changes in feeding rate and host contact
\end{itemize}
\item Pathogen:
\begin{itemize}
\item Decreased incubation period at higher temps
\item Changes in transmission season
\item Changes in georgraphical distribution
\item Decreased viral replication
\end{itemize}
\end{itemize}
\subsection{Percipitation Effects on Vectors}
\label{sec:org1dd2322}
\begin{itemize}
\item Survival: increased rain may increase larval habitat
\item Excess rain can eliminate habitat by flooding•Low rainfall can create habitat as rivers dry into pools (dry season malaria)
\item Decreased rain can increase container-breeding mosquitoes by forcing increased water storage
\item Heavy rainfall events can synchronize vector host-seeking and virus transmission
\item Increased humidity increases vector survival and vice-versa
\end{itemize}
\subsection{IPCC}
\label{sec:orgc2868dd}
\begin{itemize}
\item Intergovernmental Panel on Climate Change, intl body for assessing the science related to climate change
\item Set up in 1988 by the World Meteorological Organization and the UN Environmental Programme
\item Provide policymakers w regular assessments about climate change, impacts and future risks, options for mitigation and adaptation
\end{itemize}
\section{10.14.20}
\label{sec:orgff5adce}
\begin{itemize}
\item Exam Review
\begin{itemize}
\item Taxonomoc group with the most known species: insects
\item Types of biodiversity
\begin{itemize}
\item Genetic
\item Species
\item Ecosystem
\end{itemize}
\item Biodiversity in the Southeast
\begin{itemize}
\item Describe SE biodoviersity using the terms ``richness,'' ``endemic,'' and ``hotspot''
\end{itemize}
\item Mussels: diversity, life history, and ecosystem service (nutrient cycling)
\item What is diversity?
\end{itemize}
\item Isolation \& Extinction Risk
\begin{itemize}
\item Hawaii's biodiversity is vulnerable to extinction - more than 90\% of native species on Hawaiian islands are endemic, one half of indigenous species face extinction
\end{itemize}
\end{itemize}
\subsection{Community Ecology}
\label{sec:orgdff844c}
\begin{itemize}
\item Mutualism - A symbiotic relationship between individuals where both species benefit
\item Parasitism - A symbiotic relationship between individuals of two species in which one benefits and the other is negatively affected (may or may not lead to death)
\item Commensalism - A symbiotic relationship between individuals of two species in which one is benefitted and the other is unaffected
\item All species contribute to theur ecosystem but some are more important than others
\item Keystone species influence community structure disporportionately to their abundance
\begin{itemize}
\item Role: create/modify habitats, influence interactions between other species
\item Removal of a keystone species may lead to a loss of biodiversity and changes in community structure within the ecosystem
\end{itemize}
\item Food web: complex and realistic representation of how species feed on each other in a community
\item Food chains: a linear representation of how different species in a community feed on each other
\item Producers and Consumers
\begin{itemize}
\item Producers: photosynthetic organisms that capture energy directly form the sun and convert it into food
\item Consumers: organisms that gain energy and nutruents by eating other organisms
\begin{itemize}
\item Animals, fungi, most bacteria, and protozoa
\end{itemize}
\end{itemize}
\item Trophic level - a level in a food chain or food web
\begin{itemize}
\item Primary consumer: a species that eats producers
\item Secondary consumer: a species that eats primary consumers
\item Tertiary consumer: a species that eats secondary consumers
\item Decomposers can be put practically anywhere on the food web
\end{itemize}
\item Conservation Status: IUCN Designations
\begin{itemize}
\item The International Union for Conservation of Nature established the Red List of Threatened Species in 1963
\end{itemize}
\item Single species conservation programs focus on an individual species, successfully protecting some high-profile species but are less often used for less visible or valued species
\item CITES
\begin{itemize}
\item Convention on International Trade in Endangered Species of Wild Flora and Fauna
\end{itemize}
\item Lacey Act: First law protecting wildlife
\end{itemize}
\section{10.12.20}
\label{sec:orgee7ac99}
\subsection{Definitions of Diversity}
\label{sec:org8d33892}
\begin{itemize}
\item Genetic Diversity: Variations in the genes among individuals of the same species
\item Species Diversity: The variety of species present in an area; includes the number of different species that are present as well as their relative abundance
\item Ecological Diversity: The variety of habitats, niches, trophic levels, and community interactions
\end{itemize}
\subsection{Robust Redhorse}
\label{sec:orge4f1965}
\begin{itemize}
\item Thought to be extinct until rediscovered in the Oconee in 1991
\item Extripated: Extinct in a local area
\end{itemize}
\subsection{Species Diversity}
\label{sec:org809cef7}
\begin{itemize}
\item Richness: number of different species
\item Evenness: relative abundance of each species
\item Diversity: combined richness and evenness
\end{itemize}
\subsection{Endemic Species}
\label{sec:org59d1e3e}
\begin{itemize}
\item Because areas w high ecological diversity offer many habitats and niches, they have a large number of endemic species
\item Endemic species: a species that is native to a particular area and not usually found elsewhere
\begin{itemize}
\item Most commonly found in small ecosystems
\end{itemize}
\end{itemize}
\subsection{Hotspots}
\label{sec:org2ce38b5}
\begin{itemize}
\item Biodiversity hotspots: areas that have high endemism and have lost at least 70\% of their original habitat
\item These areas contain a large number of endangered species (species at high risk of becoming extinct)
\item The Southeast US is a global hotspot of freshwater biodiversity supporting 2/3 of the country's fish species, over 90\% of the US total species of mussels and nearly half of the global total of crayfish species
\end{itemize}
\subsection{Habitat v Niche}
\label{sec:org73174b1}
\begin{itemize}
\item Habitat: the physical location of an species
\item Niche: the biotic and abiotic needs for a species to survive
\end{itemize}
\subsection{Biodiversity Loss}
\label{sec:orgc89bafe}
\begin{itemize}
\item As much as 20\% of the world's biodiversity may be lost in the next 30 years
\item 50-66\% of biodiversity may be lost by the end of the century
\item Current rate of extinction is 1500 times greater than pre-human background rate
\end{itemize}
\subsection{Causes of Biodiversity Loss}
\label{sec:org9737c74}
\begin{itemize}
\item Human actions are having significant impacts on biodiversity loss
\item Threats include:
\begin{itemize}
\item Habitat destruction
\item Invasive Species introduction
\item Pollution
\item Overharvesting
\item Climate change
\end{itemize}
\end{itemize}
\subsection{Value of Biodiversity}
\label{sec:org7db0dee}
\begin{itemize}
\item Provides key connections between species and their environment
\item Provides direct protection against disease
\item Provide food, fuel, building materials, and pharmaceuticals
\end{itemize}
\subsection{Ecosystem Services}
\label{sec:orgd5c3c89}
\begin{itemize}
\item Supportive Services:
\begin{itemize}
\item Purification of air and water
\item Carbon sequestration
\item Erosion Prevention
\item Habitats for animals and Plants
\end{itemize}
\item Provisioning Services: Food, resources, water, fuel
\item Regulating Services: Pollination, seed dispersal, protection, biological control
\item Cultural Services: Recreation, Spiritual Tourism, mental health
\item Human Wellbeing:
\begin{itemize}
\item Strong economic growth
\item Medicinal resources
\item Reduction in toxin exposure
\end{itemize}
\end{itemize}
\subsection{Isolation and Extinction Risk}
\label{sec:org2d8076a}
\begin{itemize}
\item Number of unique species increases with isolation
\begin{itemize}
\item Isolation and high endemism makes remote islands particularly vulnerable to species loss
\item Human impact contributes to isolation in the form of habitat fragmentation
\item Habitat fragmentation: destruction of part of an area that creates a patchwork of suitable and unsuitable havitat areas that may exclude some species altogether
\end{itemize}
\end{itemize}
\section{10.09.20}
\label{sec:org5b5c64d}
\subsection{Evolution and Resistance}
\label{sec:orge61ea6e}
\begin{itemize}
\item Evolution happens to populations, not individuals
\item Natural selection is the mechanism for evolution
\item Genetic drift more likely with low population size
\item The potential for antibiotic resistance to develop in bacteria is very high
\item Improper waste disposal
\end{itemize}
\subsection{Athens Water Quality}
\label{sec:org3d02cf2}
\begin{itemize}
\item 10/17 Athens watershed are impaired or unhealthy
\item Athens drinking water comes from:
\begin{itemize}
\item N Oconee River
\item Middle Oconee River
\item Cedar Creek
\end{itemize}
\item Athens had E. Coli outbreaks in water, showing prevalence of bacteria
\end{itemize}
\subsection{Gonnorhea \& Resistance}
\label{sec:org5e0d7f8}
\begin{itemize}
\item Gonorrhea treatment is done through antibiotics
\item Shown increase in resistance to every drug used to treat Gonnorhea
\item CDC currently recommending two-drug comination to preserve our last
highly effective antibiotic
\item Higher reported rates of Gonnorhea occur in SE US, on an overall upward
trend with younger populations
\end{itemize}
\subsection{Developing new Antibiotics}
\label{sec:org04e7cfa}
\begin{itemize}
\item First antibiotic developed by Alexander Fleming in 1982 after noticing the fungus penicillium could kill disease causing bacteria
\item Antibiotics aren't profitable for drug companies
\item Developing antibiotics are high risk, very expensive, and very difficult
\item Low return on investment, development void since 1990
\end{itemize}
\subsection{Post-antibiotic Era}
\label{sec:org65f7631}
\begin{itemize}
\item Currently:
\begin{itemize}
\item 80\% of gonnorhea infections now resistant to antibiotics- 440,000 new cases of resistant tuberculosis annually
\end{itemize}
\item In the future
\begin{itemize}
\item Strep throats to scraped knees could be deadly
\item Cost to treat drug resistant double that of the status quo
\end{itemize}
\item Davos Declaration
\begin{itemize}
\item Reducing the development of drug resistance.
\item Increasing investment in R\&D that meets global public health needs.
\item Improve access to high-quality antibiotics for all.
\item Signed by 98 companies, 11 industrial associations in 21 countries
\end{itemize}
\end{itemize}
\section{10.07.20}
\label{sec:orgc1d0399}
\subsection{Genetic Diversity \& Natural Selection}
\label{sec:orgf01bf75}
\begin{itemize}
\item Genetic diversity in a population is the raw material natural selection
\item The larger the amount of genetic diversity, the higher probability that some individuals from
that pool can survive changes to its environment
\item Phenotype = expressed gene
\item Natural selection acts directly on the phenotype, resulting in changes in allele frequencies
from parental to offspring generations
\end{itemize}
\section{10.05.20}
\label{sec:org569d57a}
\begin{itemize}
\item Following widespread usage of antibiotics on humans and animals, waste from livestock and humans
is generating antibiotic-resistance bacteria
\item These bacteria are getting back into the environment through out waste
\end{itemize}
\subsection{Antibiotic Resistance:}
\label{sec:orga3196be}
\begin{itemize}
\item A complex problem that involves helping many actors see the big picture and not just their
part of it
\item Issues where an action affects (or is affected by) the environment surrounding the issue,
either the natural environment or the competitive environment
\item Problem whose solutions are not Obvious
\end{itemize}
\subsection{Systems Thinking}
\label{sec:orgfde2500}
\begin{itemize}
\item Considers the whole rather than parts of the whole:
\begin{itemize}
\item Events
\item Patterns
\item Underlying Structure
\end{itemize}
\end{itemize}
\subsection{Cycle of Infection}
\label{sec:orgd09e613}
\begin{itemize}
\item Farm animals recieve antibiotics often, developing resistant bacteria in their gut
\item This can be transmitted through produce, waste, shared environments, etc.
\end{itemize}
\subsection{Bacteria}
\label{sec:orga836eab}
\begin{itemize}
\item Bacteria are single celled organisms that can grow in colonies
\item Many different kinds of bacteria can grow together in similar environments
\end{itemize}
\subsection{Explaining Resistance}
\label{sec:org9d42dbc}
\begin{itemize}
\item Antibiotics kill almost all antibiotic sensitive bacteria, leaving few sensitive and many unsensitive
\item Reproduction occurs with the mostly-unsensitive remaining bacteria, leaving to many unsensitive off-
spring. This increases the amonut of resistant bacteria as a whole.
\end{itemize}
\subsection{Genetic Variation}
\label{sec:org773d175}
\begin{itemize}
\item Variation in the susceptability of bacteria to antibiotics allows for the propogation of 
these genes in bacterial communities
\item Individuals of the same species have the same basic gene
\item Alleles: variants of genes that account for the diversity of traits seen in a populat
\item Adaptation: traits that promote the success of a species
\item An adaptive trait for one environmental condition does not mean that it is adaptive for all conditions
\end{itemize}
\subsection{Genetic Diversity}
\label{sec:org58e1beb}
\begin{itemize}
\item Within populations, biodiversity is measured by genetic diversity
\item Genetic diversity improves survival of a population
\item Outbreeding, through sexual reproduction of not closely related individuals, maximizes genetic 
diversity
\item Inbreeding, or mating between closely related individuals, results from small 
populations, and increases chances of genetic diseases (e.g., hemophilia, cystic fibrosis, etc.)
\end{itemize}
\subsection{Sources of Genetic Variation}
\label{sec:org8644622}
\begin{itemize}
\item Mutation: A change in the DNA sequence of sex cells that alter a gene
\begin{itemize}
\item Can be neutral, beneficial, or harmful
\end{itemize}
\item Genetic Recombination: The production of eggs and sperm that results in a shuffling of 
alleles, creating new combinations in offspring
\end{itemize}
\subsection{Natural Selection}
\label{sec:org3a7e1a2}
\begin{itemize}
\item Constant struggle of organisms to survive and mate
\item Organisms tend to produce more offspring that can survive
\item Individuals of the same species are not identical
\item Evidence of Natural Selection: Selective breeding (artificial selection) of dogs and cats
\item Natural selection results in changes in gene frequencies
\begin{itemize}
\item Some individuals will be able to obtain more resources and can produce more offspring
\begin{itemize}
\item Differential reproductive success results in changes to gene frequencies
\end{itemize}
\end{itemize}
\end{itemize}
\section{09.18.20}
\label{sec:org8d742d8}
\subsection{Hurricanes}
\label{sec:org1529e8c}
\subsubsection{How Hurricanes Form}
\label{sec:orgefb9908}
\begin{itemize}
\item Water evaporates over the ocean and forms clouds when it touches cold air
\item A column of low pressure develops at the center with winds around the column
\item Speed of the wind around it increases
\end{itemize}
\begin{itemize}
\item Categorized based on wind speed (1-5)
\item Hurrican development requires warm water and low wind shear
\begin{itemize}
\item Carribean has warm water all year but also high wind shear which isn't conducive to hurricanes
\end{itemize}
\end{itemize}
\subsubsection{Climate Change \& Hurricanes}
\label{sec:orgc9e0d60}
\begin{itemize}
\item Storm surge more dangerous (accoutns for 90\% of hurricane deaths)
\item 40\% increase with a 0.5 decree C inc in temperature
\item Increasing of North Atlantic hurricane season
\item Climate change is expected to shift the Bermuda high westward
\begin{itemize}
\item Bermuda High is a pressure system over the Atlantic
\item Has the ability to move hurricanes on the Atlantic
\end{itemize}
\end{itemize}
\subsubsection{Hurricane Harvey Intensification}
\label{sec:org2f8b512}
\begin{itemize}
\item Went from a tropical depression to a Cat 4 Hurricane in 57 hours
\item Soil in TX affected the amount of water maintained in the Earth
\item Huge economic impacts
\end{itemize}
\subsubsection{General Impacts}
\label{sec:org302450a}
\begin{itemize}
\item Storm Surge
\item Extreme Rainfall
\item Potential Wind Speed
\end{itemize}
\section{09.16.20}
\label{sec:orgf58539a}
\subsection{Heat Waves}
\label{sec:orgbd8d4aa}
\begin{itemize}
\item Heat extremes doubled in frequency from 1980-1999 to 2000-2019
\item Climate change affecting heat waves
\begin{itemize}
\item Shifting the frequency of hot and cold weather, heat waves are more frequent
\item Exacerbating heat inducing droughts, dry land leads to even hotter temps
\end{itemize}
\item Causes: Global warming ->
\begin{itemize}
\item Large scale global circulation change
\item Atmospheric Blocking increase
\item Air mass temp increase
\end{itemize}
\item Effects and Consequences
\begin{itemize}
\item Decreased human productivity
\item Increased tropical disease and death
\item Environmental racism
\item Crop productivity decreases
\item Lower biodiversity
\item Decreased water availability
\item Increased fire risk
\end{itemize}
\end{itemize}
\subsection{Wildfires}
\label{sec:org81e6c0a}
\begin{itemize}
\item Climate change is increasing the size, intensity, and frequency of wildfires
\item Wildfires create more cimate change through the increase of carbon expulsion through wildfires
\item Wildfires have global impacts due to smoke and temperature changes
\item Wildfire season has gotten longer due to climate change
\end{itemize}
\section{09.14.20}
\label{sec:org7393a6c}
\subsection{Coriolis Effect}
\label{sec:orge1f0bba}
\begin{itemize}
\item Deflection of an object's path due to the rotation of the Earth
\item North and south poles have different deflections of wind patterns
\item Little/no deflection at the equator
\end{itemize}
\subsection{Air circulation}
\label{sec:orgd98a86e}
\begin{itemize}
\item Hottest air at the equator, moves north or south, cools, then comes back into equator
\end{itemize}
\subsubsection{Cells}
\label{sec:org0e5ecb0}
\begin{itemize}
\item Hadley cells: 0-30 degrees North and South
\item Ferrell Cell: 30-60 degrees North
\item Polar cells: North and South poles
\item Northeast and Southeast trade winds (remember directions!)
\item Westerlies: bring rain and precipitation
\end{itemize}
\subsection{Surface Ocean Currents}
\label{sec:orgfff05cb}
\begin{itemize}
\item Ocean currents also affect the distribution of climates
\item Surface ocean currents generated by wind, Coriolis effect, heat, and continents
\item Heat redistribution from the Tropics
\begin{itemize}
\item Trade winds push warm surface waters west
\item Water reaches continents and flows north and south
\item water cools
\item Westerlies push cooler water east
\item Water reaches continents and flows to equator
\end{itemize}
\end{itemize}
\subsection{El Nino (Southern Oscillation)}
\label{sec:orgd9f6967}
\begin{itemize}
\item Recurring climate pattern involving changes in the termperature of waters in the central
and eastern tropical Pacific Ocean.
\item The ocean and atmosphere can interact to affect climate
\begin{itemize}
\item Water in the eastern pacific warms up
\item Sea level pressure drops but rises in the W pacific
\item Trade winds weaken
\item Upwelling in the Pacific is reduced
\item Warmer waters - increased rainfall in Peru
\item Cooler waters, drought in Australia/Indonesia
\end{itemize}
\item Critical because of its ability to change atmospheric circulation, temps, and percipitation
\item Significantly hurts fisheries and developing countries
\end{itemize}
\subsection{La Nina}
\label{sec:org48011e1}
\begin{itemize}
\item exacerbates normal conditions and leads to cooling in the Eastern pacific
\end{itemize}
\subsection{Heat Waves}
\label{sec:org80f3c74}
\begin{itemize}
\item Global warming has amplified the intensity, duration, and frequency of 
extreme heat and heat waves.
\end{itemize}
\section{09.11.20}
\label{sec:orgb1855c2}
\begin{itemize}
\item Northern latitudes experience greater seasonality in CO2 concentrations
\begin{itemize}
\item This is due to variation in photosynthetic activity by plants
\end{itemize}
\item Greenhouse effect
\begin{itemize}
\item Some incoming solar radiation is absorbed
\item Other amounts are reflected back into the atmosphere
\item Greenhouse gases capture and reradiate some heat over and over, warming the Earth
\item More gases, more heat
\end{itemize}
\item Albedo: measure of the reflectivity of a surface
\begin{itemize}
\item light surfaces have a higher albedo, darker surfaces have a lower albedo
\item surfaces with a low albedo release more heat into the atmosphere
\end{itemize}
\item Positive Feedback Loops
\begin{itemize}
\item applied to albedo:
\item temps rise -> more ice melting -> more water warming -> temps rise
\end{itemize}
\item Urban Heat Island Effect
\begin{itemize}
\item cities will be inc their population, inc energy and temperature
\item cities in particular have higher temperatures
\item tree cover -> cooler temperatures
\end{itemize}
\item Small changes in overall global temp can cause significant changes
in weather creating more extreme storms and more record temps
\begin{itemize}
\item roughly twice as many heat records
\item alterations in global jet streams
\item frost comes later and begins earlier
\end{itemize}
\item General climate change impacts:
\begin{itemize}
\item Health impacts
\item Crop productivity
\item Coastal erosion
\item Biodiversity
\item Water availability
\item Fire risk
\end{itemize}
\item Weather events getting more extreme with
\begin{itemize}
\item sea levels
\item wildfires
\end{itemize}
\item Need both adaptation and mitigation
\begin{itemize}
\item Adaptation: Responding to warming that has already happened
\item Mitigation: Preventing further warming by addressing climate change causes
\end{itemize}
\end{itemize}
\section{09.09.20}
\label{sec:org7ef4cee}
\subsection{The Earth's Atmoshphere}
\label{sec:org4bd5cff}
\begin{itemize}
\item Climate change is a serious environmental problem impacting species, ecosystems, and the globe
\item The atmosphere helps protect the Earth from the sun and keeps the temperature of the Earth cool
\item Atmosphere has a significant impact on climate
\item Earth's Atmosphere Composition
\begin{itemize}
\item Nitrogen (78\%)
\item Oxygen (21\%)
\item Other - Greenhouse Gases (1\%)
\end{itemize}
\end{itemize}
\subsection{The Keeling Curve}
\label{sec:org169e553}
\begin{itemize}
\item Curve developed to track atmospheric CO2 levels in Earth's atmosphere since 1952
\end{itemize}
\section{09.02.20}
\label{sec:orge790bf2}
\subsection{Demographic Transition Model}
\label{sec:org82f361c}
\begin{itemize}
\item Demographers use age structure diagrams to predict future growth potential of a population
\begin{itemize}
\item Pyramid structures indicate fast growth
\item House-shaped structures have moderate growth
\item Diamond structures have low/negative growth
\end{itemize}
\item Development leads to smaller families
\item Demographic transitions happen country by country
\item Industrialization might not lead to a demographic transition in all countries
\begin{itemize}
\item May not be linked to quality of life
\item Religion/Cultural beliefs
\item Social justice issue, improving the well-being of women and children key to dec. fertility
\end{itemize}
\end{itemize}
\subsection{Social Justice: Education for Women}
\label{sec:org3164520}
\begin{itemize}
\item Education of girls \& economic opportunities for women are correlated with lower birth rates
\item Education empowers women to take control over thri own fertility through: 
\begin{itemize}
\item Birth control
\item Marrying later
\item Delaying childbirth for career opportunities
\end{itemize}
\item Women earning more money is correlated to lower child mortality
\end{itemize}
\subsection{Environmental Impact}
\label{sec:orgb20c2f6}
\begin{itemize}
\item Slowing population growth is critical to sustainability and reducing our population impact
\item Our impact on the population is a result of (1) our population size and
(2) our consumption habits - both must be addressed
\item Ecological footprint: the land area needed to provide the resources for, and assimilate
the waste of, a person or population
\end{itemize}
\subsection{Sustainability}
\label{sec:org7dbe861}
\begin{itemize}
\item A dynamic process between the economy, society, and environment
\item Sustainable: The process or the activity can be mantained without exhaustion or collapse
\begin{itemize}
\item Intra \& Inter-generational issue
\item Capacity of a system to accomodate changes:
\begin{itemize}
\item rates of renewable resource use should not exceed regeneration rate
\item rates of non-renewable resource use should not exceed rate of renewable substitute dev
\item rates of pollution should not exceed ssimilative capacity of the environment
\end{itemize}
\end{itemize}
\item Sustainable development has three factors:
\begin{itemize}
\item Social equity
\item Economic efficiency
\item Environmental responsibility
\end{itemize}
\end{itemize}
\subsection{Worldviews}
\label{sec:orgfd6cf69}
\begin{itemize}
\item Culture influences our beliefs through:
\begin{itemize}
\item Knowledge
\item Beliefs
\item Values
\item Learned ways of life
\end{itemize}
\item Worldviews are affected by: 
\begin{itemize}
\item Environmental Ethics
\end{itemize}
\end{itemize}
\section{08.31.20}
\label{sec:org7518cc2}
\subsection{Human Populations}
\label{sec:org7211cea}
\begin{itemize}
\item 3 major sparks of growth
\begin{itemize}
\item Agricultural Revolution
\item Industrual Revolution
\item Green Revolution
\end{itemize}
\item With more food and technology, the population and need for more human labor increased
\item The human population is rapidly increasing and the impact of humans is due to:
\begin{itemize}
\item More humans overall
\item Greater growth / person
\end{itemize}
\item To address population growth, we need to pursue a variety of approaches that address factors
encouraging high birth rates
\item Zero population growth: the absence of population growth, occurs when birth rates = death rates
\begin{itemize}
\item Replacement fertility is reached
\end{itemize}
\end{itemize}
\subsection{Population Ecology}
\label{sec:org8a9f6a5}
\begin{itemize}
\item Analyze and categorize human populations using population ecology techniques
\item Population Ecology: a branch of biology dealing with the number of individuals
in a particular species in an area over time
\item Ecologists study populations to understand what makes them survive and thrive
\item Size, distribution, and growth rate is influenced by a variaty of factors and are important to 
understanding popilation ecology
\end{itemize}
\subsection{Monitoring Population Dynamics}
\label{sec:org2bb61c6}
\begin{itemize}
\item Population Dynamics: Changes over time in population size and composition
\item Important metrics:
\begin{itemize}
\item Minimum viable population - min number of individuals that would still allow population to persist or grow
\item Carrying Capacity (K) - the maximum population size that a particular environment can support indefinitely
\end{itemize}
\item Population Density - the overall desnity a particular populaiton can sustain
\end{itemize}
\subsection{Exponential Growth \& Populations}
\label{sec:org106636b}
\begin{itemize}
\item Exponential growth occurs in populations when growth is unrestricted. This is, overall, unsustainable
\item Growth which becomes progressively larger each breeding cycle
\item Produces a J curve when plotted
\end{itemize}
\subsection{Monitoring Population Growth}
\label{sec:org20266ee}
\begin{itemize}
\item Population growth rate - the rate at which a population of a species grows over time
\item Growth factors - factos which assist in the growth of a population
\item Resistance factors - factors which inhibit the growth of a population
\item Limiting factos: resources needed for survival but that may be in short supply
\end{itemize}
\subsection{Logistic Growth}
\label{sec:org77af78d}
\begin{itemize}
\item Occurs when a population nears carrying capacity (k) 
\begin{itemize}
\item Maximum sustainable population size
\item Determined by limiting factors
\end{itemize}
\end{itemize}
\subsection{Density-dependent/ Density-independent Factors}
\label{sec:org88d47d1}
\begin{itemize}
\item Density dependent factors increase as populations grow, typically biotic
\begin{itemize}
\item Disease
\item Competition
\item Predation
\end{itemize}
\item Density independent facts affect population growth regardless of population size
\begin{itemize}
\item Storm
\item Fire/Flood
\item Avalanche
\end{itemize}
\end{itemize}
\subsection{Regulation}
\label{sec:orgf309d28}
\begin{itemize}
\item Tendency for populations to decrease in size when above acertain level, and increase
in size below that level
\item Populations can only be regulated by density-dependent factors
\item Top down Regulation
\begin{itemize}
\item Predation
\item Disease
\end{itemize}
\item Bottom up Regulation
\begin{itemize}
\item Nutrients
\item Water
\item Sunlight
\end{itemize}
\end{itemize}
\section{08.28.20}
\label{sec:org5638252}
\subsection{What is Science?}
\label{sec:orgf687c9e}
\begin{itemize}
\item Science: a body of knowledge that allows us to understand the world around us
\item Science is based on empirical evidence
\item Science allows us to test our ideas and evaluate the evidence
\item Scientific knowledge, including facts, theories, and laws, is subject to change
\item Scientific claims change as new evidence is made available
\end{itemize}
\subsection{White-Nose Syndrome Case Study}
\label{sec:orga0407f8}
\subsubsection{About WNS}
\label{sec:orgaead707}
\begin{itemize}
\item White-Nose Syndrome
\begin{itemize}
\item 2007-2016, 6+ million bats dead as a result of White Nose Syndrome
\item The reason for the deaths was White-Nose Syndrome
\end{itemize}
\item Chytridiomycosis
\begin{itemize}
\item Infectious, fungal disease affecting amphibians
\item Helped understand white-nose syndrome with bats
\end{itemize}
\end{itemize}
\subsubsection{Science with WNS}
\label{sec:orga918d40}
\begin{itemize}
\item Scientific Method: the procedure used to empirically test a hypothesis
\begin{enumerate}
\item Observations generate questions
\item Choose a question to investigate
\item Consult literature
\item Develop a hypothesis and make a testable prediction
\item Design and carry out a study
\item Analyze data
\item Draw a conclusion
\end{enumerate}
\item Inferences: Conclusions drawn based on observations
\item Hypothesis: An inference that proposes possible explanation that includes previous knowledge/observation
\item Testing a Hypothesis: Hypotheses can be tested through an observational or experimental study
\item Scientific Studies: A fair test with results that could support or falsify the research prediction
\begin{itemize}
\item Experimental Studies: Conditions are manipulated intentionally
\begin{itemize}
\item Test Group: the group in an experimental study such that it differs from the control in only one way
\item Control Group: the group in an experimental study to which the test group's results are compared
\end{itemize}
\item Observational Studies: Gather real-world data without any intentional variable manipulation
\end{itemize}
\item Theory: A hypothesis that survives repeated testing by significant research can become a theory
\item Correlation v Causation
\begin{itemize}
\item Correlation: two things occuring together but not necessarily having a cause-effect relationship
\item Cause-Effect Relationship: the associationof a two variables that identifies one variable occurring
as a result of the other
\item Observational studies can derive correlation but not causation
\item Experimental studies can derive causational relationships
\end{itemize}
\item Policy: a formalized plan that addresses a desired outcome or goal
\begin{itemize}
\item policies need to be flexible, adapt to new findings, address the environmental problem, fit social need
and be economically viable in order to work effectively.
\end{itemize}
\end{itemize}
\subsection{Summary}
\label{sec:org2b532ac}
\begin{itemize}
\item Scientific knowledge, through reliable and durable, is never absolute pr certain
\item This knowledge, including facts, theories, and laws, is subject to change
\item Physical evidence, systematically collected and logically analyzed, helps scientists
understand environmental issues and guide policy decisions
\end{itemize}
\section{08.25.20}
\label{sec:org25e0157}
\subsection{Applied v Empirical Science}
\label{sec:orgb292c91}
\begin{itemize}
\item Applied Science = research whose findings are used to solve practical problems
\item Empirical science: A scientific approach that investigates the natural world through case studies
\end{itemize}
\subsection{Social Traps}
\label{sec:org9d44efe}
\begin{itemize}
\item Occurs when a large amount of people are using a shared resource
\item Seem good in the short term but are actually bad in the long term
\item 3 Types:
\begin{itemize}
\item Tragedy of the Commons: When resources are shared, individuals try to maximize personal
benefit which hurts the resource itself
\item Time delay: Collective decisions that are good today but gone tomorrow
\item Sliding reinforcer: related to the evolution of natural organisms and GMOs
\end{itemize}
\end{itemize}
\subsection{Beginning with Data Interpretation}
\label{sec:orgb26f0e4}
\begin{itemize}
\item Variables represent factors that can be manipulated, controlled, or merely measured for research
\item Variation = how much a variable changes
\item Independent var is controlled to see effects in the Dependent var
\item Graphs explore relationships with data and report this data
\end{itemize}
\subsection{Observational v Experimental Studies}
\label{sec:orgb5af828}
\begin{itemize}
\item Observational studies can observe a correlation but are unable to derive a causational reln.
\item Experimental studies have a control var (required) and are able to derive causactional rlns.
\end{itemize}
\section{08.24.20}
\label{sec:org8a8128b}
\subsection{Definitions}
\label{sec:org86df7d0}
\begin{itemize}
\item Ecology: the branch of science dealing with the relationships of living things to one another \& the environment
\item Environmental Science: The study of all aspects of the environment, including physical, chemical, and biological factos, particularly with respect to how these aspects affect humans, and vice versa
\item Environmental Ethics: Personal philosophy that influences how a person interacts with their natural environment and thus influences how one responds to environmental problems
\end{itemize}
\subsection{Ecology != Environmentalism}
\label{sec:org4248176}
\begin{itemize}
\item Distinguish between envrironmentalism \& ecology
\end{itemize}

\begin{center}
\begin{tabular}{ll}
Environmentalism & Ecology\\
\hline
Activism to protect the environment & Scientific study of living and non-living things\\
\end{tabular}
\end{center}
\end{document}
