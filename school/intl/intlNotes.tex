% Created 2020-09-23 Wed 22:16
% Intended LaTeX compiler: pdflatex
\documentclass[11pt]{article}
\usepackage[utf8]{inputenc}
\usepackage[T1]{fontenc}
\usepackage{graphicx}
\usepackage{grffile}
\usepackage{longtable}
\usepackage{wrapfig}
\usepackage{rotating}
\usepackage[normalem]{ulem}
\usepackage{amsmath}
\usepackage{textcomp}
\usepackage{amssymb}
\usepackage{capt-of}
\usepackage{hyperref}
\author{Sudhan Chitgopkar}
\date{\today}
\title{}
\hypersetup{
 pdfauthor={Sudhan Chitgopkar},
 pdftitle={},
 pdfkeywords={},
 pdfsubject={},
 pdfcreator={Emacs 26.3 (Org mode 9.1.9)}, 
 pdflang={English}}
\begin{document}

\tableofcontents

\section{Module 6}
\label{sec:orgf9ab1d2}
\subsection{Democratic Institutions}
\label{sec:org70637b3}
\subsubsection{Legislatures}
\label{sec:org32b400d}
\begin{itemize}
\item Forum for national political Debate
\item Where laws are proposed or passed
\item Bicameral (two houses)
\begin{itemize}
\item Senate \& House in US
\begin{itemize}
\item House of Lords \& Commons in the UK
\end{itemize}
\end{itemize}
\item Unicameral
\begin{itemize}
\item Single house more likely to be found in smaller and more centralized democracies
\item Sfound in Norway, South Koera
\end{itemize}
\end{itemize}
\subsubsection{Judiciary}
\label{sec:org1c4b729}
\begin{itemize}
\item Central to democracy's rule of law
\item Different types of courts and organizations of courts
\item Some countries have a constitutional court solely to interpret constitutional legality,
this is shown through judicial review
\item Not all countries have a supreme court that exercises judicial review
\end{itemize}
\subsubsection{Executvies}
\label{sec:org9a50996}
\begin{itemize}
\item Head of State
\begin{itemize}
\item Represents the government on the national/intl stage, mainly symbolid
\end{itemize}
\end{itemize}
\begin{itemize}
\item Head of Government
\begin{itemize}
\item Focuses on policy-making
\end{itemize}
\item President of the US is both head of state and government
\item Types of Executive systems
\begin{itemize}
\item Parliamentary
\item Presidential
\item Semi-Presidential
\end{itemize}
\end{itemize}
\begin{center}
\begin{tabular}{ll}
Presidential & Parliamentary\\
Limited government (Separation of powers) & High policy-making efficiency\\
Checks and Balances -> Gridlock & Fusion of divisions != Gridlock\\
Popularly-elected executive & Executive: leader of largest party\\
Fixed terms, no long term grip & Parties can hold power for long times\\
Elections are candidate-based & Single party loyalty (?)\\
\end{tabular}
\end{center}
\subsection{Electoral Systems}
\label{sec:org6d57082}
\subsubsection{Single-member District (SMD)}
\label{sec:org7540574}
\begin{itemize}
\item Also called the "first past the post" or "winner take all" system
\item Voting for candidates directly instead of for a party
\item Most likely to result in a two-party system
\begin{itemize}
\item Called Duverge's Law
\end{itemize}
\end{itemize}
\subsubsection{Proportional Representation (PR)}
\label{sec:org04b6c8f}
\begin{itemize}
\item Made of multi-member districts (mmd) - more than 1 person elected from ea. electoral district
\item People vote for parties moreso than individuals
\item Votes are ranked for parties
\item Proportion of the vote for a particular party wins the party certain amounts of seats
\item Tends to result in multiple parties winning seats, generally leading to electoral thresholds
\end{itemize}
\subsubsection{Mixed Electoral Systems}
\label{sec:org01b7f70}
\begin{itemize}
\item A comboination of SMD and PR systems
\item Ranked voting
\begin{itemize}
\item Alternative Vote - Australia
\item Single Transferrable Vote - Ireland
\end{itemize}
\end{itemize}
\subsubsection{Referenda and Initiatives}
\label{sec:org6a0145d}
\begin{itemize}
\item Some countries put political decisions in the hand of the people through a referendum
\item Can be seen as a cop-out by legislators and executives back to the people
\item Referenda can also be constitutionally-mandated as in Ireland
\item Initiatives are political decisions put to the people due to a petition
\item Certain number of people need to sign a petition before a vote
\end{itemize}
\section{Module 5}
\label{sec:org623af18}
\subsection{Political Economy}
\label{sec:orgf334352}
\begin{itemize}
\item Political Economy: The study of how politics and economics are related
\item Components:
\begin{itemize}
\item Markets
\item Property
\item Public goods
\item Taxation
\item Fiscal Policy
\item Regulation
\item Trade
\end{itemize}
\item Public Goods \& Social ExpendituresL
\begin{itemize}
\item Public goods: Those goods provided or secured by the state and are available for everyone
\item Social expenditure: The state's provision of public benefits or welfare
\begin{itemize}
\item All states have some kind of social expenditure
\end{itemize}
\end{itemize}
\item Taxation
\begin{itemize}
\item Mostly needed to fund state activities
\item Different kinds of taxes at different levels
\item Some countries provide goods and services mostly from revenues from taxation
\end{itemize}
\item Regulations
\begin{itemize}
\item Rules or orders that set the boundaries of a given procedure
\item Costs of compliance
\item Costs of monitoring
\item Costs of non-compliance
\end{itemize}
\item Trade \& Economic Development
\begin{itemize}
\item Free Trade: Trade among countries wherein no country restricts trade from any other country
\begin{itemize}
\item by levying import tariffs/duties
\item through imposition of quotas
\item by providing subsidies to its own domestic firms
\item by introducing other non-tariff barriers
\end{itemize}
\item Trade that is free from barriers is theorized to improve economic development/innovaiton
through the use of a comparative advantage
\end{itemize}
\end{itemize}
\subsection{Varieties of Capitalism}
\label{sec:org86979e4}
\begin{itemize}
\item Advantages of market systems
\begin{itemize}
\item very dynamic
\item high levels of productivity
\end{itemize}
\item Disadvantages of market systems
\begin{itemize}
\item Variability
\item Negative market swings can ahve a domino effect
\item Negative social externalities (inequality, unemployment, etc)
\end{itemize}
\item Political-Economic Systems
\begin{itemize}
\item Liberal Democracy
\item Social Democracy
\item Mercantile Democracy
\item Communism
\end{itemize}
\item Liberal Democracy: An ideology and political system that favors limited state role in society 
and the economy and places a high priorty on individual political and economic freedom
\item Social Democracy: A political-economic system where freedom and equality are balanced through 
state management of economy and provision of social expenditures
\begin{itemize}
\item features corporatism where government, forms, and workers have a tripartite relationship
\item often called a coordinated market economy
\end{itemize}
\item Mercantile Democracy: State controls economy
\begin{itemize}
\item State owns parts or all of industry
\item Heavy regulations, tariffs, and non-tariff barriers to foster and protect domestic industry
\item Little social expenditure, low taxes
\item Allows for rapid economic growth (Asian TIGER countries) and often export oriented
\end{itemize}
\item No single type of democracy is better than another- some simply align with certain interests
\end{itemize}
\section{Module 4}
\label{sec:org1890847}
\subsection{Nations \& Society}
\label{sec:org00af344}
\begin{itemize}
\item Goals of nation-building:
\begin{itemize}
\item Capacity
\item Legitimacy
\item Identity
\end{itemize}
\item Society: "A collection of people bound by shared institutions that define how relations
should be conducted
\item Types of Identity:
\begin{itemize}
\item Primordial (genetic)
\item Ascribed (given by others)
\item Socially constructed (develops over time)
\end{itemize}
\item Identity is not inherently political but can be politicized
\item Citizenship: An individual or group's relation to the state
\item Different states have different citizenship regimes
\begin{itemize}
\item Allowance of dual citizenship
\item Types of naturalization process
\end{itemize}
\item Identity as an Institution
\begin{itemize}
\item Identities comprise kinds of institutions
\item Identites are sticky
\item Politicization of identities increases probability of conflict
\end{itemize}
\item Ethnic conflict: Conflict between ethnic groups that struggle to acheive goals
at each other's expense
\item National Conflict: Conflict in which one or more groups within a country 
develops clear aspirations for political independence, clashing with others as a result
\end{itemize}
\subsection{Political Culture \& Ideology}
\label{sec:orga6f72f0}
\begin{itemize}
\item Political culture is very difficult to define and is relative
\begin{itemize}
\item can be considered an informal institution
\item may be rooted in culture or religion
\item developed from an early age
\end{itemize}
\item Political attitudes: how one sees the operations of the state and its institutions
\begin{itemize}
\item Radical, liberal, conservative, reactionary
\item Majority are around center
\item Liberal: Seek to change society through institutional adjustments
\item Constitution: Prefer continuity, resist change
\item Radicals and Reactionaries: generally outside instuitutions, may use violence
\end{itemize}
\item Attitudes are relative to political culture
\begin{itemize}
\item A liberal in the US = a conservative in France
\end{itemize}
\item Political ideologies: what one views as the fundamental goals of politics
\begin{itemize}
\item Communism -> Social Democracy -> Liberalism -> Fascism -> Anarchy
\item Here, liberalism supports political choice, not political attitudes
\item Social democracy supports greater state intervention
\item Communism, Facsism, and Anarchy are non-democratic (radical or reactionary)
\end{itemize}
\item Socialist definition
\begin{itemize}
\item Communist parties of the former societ bloc (non-democratic) described as socialist
\item Nazi (extreme right) stood for national socialist party
\item Social democrat parties of advanced democracies are democratic
\end{itemize}
\end{itemize}
\section{09.02.20}
\label{sec:org4f8e739}
\subsection{State Development}
\label{sec:orgf525be8}
\begin{itemize}
\item Europe v the New World
\begin{itemize}
\item Compare the state development of European, "old-world" countries and "new world" countries"
\begin{itemize}
\item Old world countries tend to be more imperialistic while new countries have a common exp
of being colonies
\item New world countries were composed of different types of people while 
Old world countries had a shared history
\end{itemize}
\end{itemize}
\item Feudalism: Geographic proximity and increasing power of feudal lords -> challenges between 
feudal properties were likely, so organization of resources and capabilities was key to survival
\item Feudalism led to increased collectivism, translating to:
\begin{itemize}
\item large, active labor organizations
\item large, state-provided social welfare
\item emphasis on production of higher quality goods instead of new innovation
\end{itemize}
\end{itemize}
\section{Module 3}
\label{sec:orge4acede}
\subsection{Institutions and States}
\label{sec:org00bee4a}
\subsubsection{Institutions}
\label{sec:orge046df2}
\begin{itemize}
\item Institution: Institutions are formal and informal rules 
that structure the relationship among individuals
\item Can have legal or social forces
\item Institutions are resistant to change but can change as a 
\begin{itemize}
\item response to outside forces
\item response to internal pressures
\item response to effects of other institutions
\end{itemize}
\end{itemize}
\subsubsection{The State}
\label{sec:org28077e8}
\begin{itemize}
\item An organization that maintains a legitimate monopoly of force over a certain territory
and its population
\item A set of political institutions sets policies for the territory and its population
\item Sovereignty: The ability for a state to carry out actions/policies within a territory
independently from external actors or internal rivals/challengers
\item Issues of autonomy and capcity: 
\begin{itemize}
\item Autonomy: the ability for the state to weild its power independently of the public
\item Capacity: the ability for the state to accrue and utilize sufficient resources to carry out
basic tasks and responsibilities
\end{itemize}
\end{itemize}
\subsubsection{Definitions}
\label{sec:orga2470a7}
\begin{enumerate}
\item General
\label{sec:org6646c37}
\begin{itemize}
\item State: governing structur's legitimate expression of sovereignty/main political organization 
of a country
\item Regime: Informal institutions that guide how a state operates
\item Government: Collection of actors in charge of carrying out political decisions of the regime
and in the interest of the state
\item Country: More generic; refers to the political collectivity of a soverieng territory
\item Nation: Refers to a group of people bound together by some trait who seek to establish 
to establish and express political interests
\item Nation != Country
\end{itemize}
\item Strength of States
\label{sec:org49ad7e5}
\begin{itemize}
\item Institutional Capabilities
\begin{itemize}
\item Strong States: Has good institutional foundations; these institutions function well
\item Weak States: Does not have good institutional foundations, its institutions do not function well
\item Failed States: Institutions so weak that they basically collapse and have no sovereignty
\end{itemize}
\item Organizational Structure
\begin{itemize}
\item Strong states maintain a fair amonut of centralized control
\item Weak states hand down authority to local institutions and are decentralized
\end{itemize}
\end{itemize}
\end{enumerate}
\subsection{Legitimacy \& Sovereignty}
\label{sec:org162b8a4}
\begin{itemize}
\item Legitimacy: a value whereby something or someone is recognized and accepted by a large 
portion of the population as right and proper (is highly subjective)
\item Types of legitimacy:
\begin{itemize}
\item Traditional legitimacy: embodies historical myths/legends and continues from past to present
\item Charismatic legitimacy: Built on the force of ideas and appeals embodied by a leader
\item Rational-Legal legitimacy: Based on a system of laws and procedures that are institutionalized
\end{itemize}
\item Sources of Legitimacy:
\begin{itemize}
\item Conferred by the ruler to a ruler, government, or state
\item Ascribed to a state or ruler by other states or rulers (prerequisity for intl. cooperation)
\item Ascribed to a state or ruler by organizations/non-state actors
\end{itemize}
\item Legitimacy can often be used to push for change
\end{itemize}
\section{08.26.20}
\label{sec:org93f33e3}
\subsection{Defining a Good Society}
\label{sec:org24c4701}
\begin{itemize}
\item Although observable, empirical assessments may differ from person to person,
depending upon factors that may distort individual observation.
\item Multiple factors contribute to whether a society is "good" or not, critical to comparing countries and 
political systems
\end{itemize}
\section{Module 2}
\label{sec:orge04327b}
\subsection{Video 1}
\label{sec:orgf94ae19}
\subsubsection{"Traditional Approach"}
\label{sec:org8275b03}
\begin{itemize}
\item Focus on a "formal-legal" aspects of political institutions
\item Mostly a categorizing exercise with little analysis
\item Many European ex-pats were these scholars
\end{itemize}
\subsubsection{Modern Era (1960s-1980s)}
\label{sec:orgb253cba}
\begin{itemize}
\item Scholars stop describing, start comparing
\item Behavioral Revolution - emphasis on individual, group behavior, not static institutions
\item Gave rise to "developmentalism" or "modernization theory" 
\begin{itemize}
\item Proposed that a state develops economically, political and social development follows
\item Functionalism (functions of differently societal elements lay foundation for growth)
\end{itemize}
\end{itemize}
\subsubsection{Development (1960s-1980s)}
\label{sec:org0fe916a}
\begin{itemize}
\item 5 stages each society goes through for development:
\item Traditional society (no mass production)
\item Preconditions for economic take-off (advent of industrialization and mass production)
\item Take-off (dynamic economic growth)
\item Drive to maturity (long era of econ growth, modern tech usage)
\item Age of high mass consumption (everyon is within driving distance of McDonalds (most places))
\end{itemize}
\subsubsection{Critiques of Behavioralims/Developmentalism}
\label{sec:org0c617aa}
\begin{itemize}
\item Enthocentric and ideologically driven
\item Creates dependency: capitalism creates a situation where underdeveloped countries depend
on developed countries
\item Developmentalist theories tried to be a one-size-fit-all theory which wasn't bale to be applied
to all individual case studies
\end{itemize}
\subsubsection{Post-Behavioralism (1990s-Present)}
\label{sec:org693ecf5}
\begin{itemize}
\item Development of middle-range theories instead of one single theory
\item Diversity of approaches (qualitative, quantitative, case sudies)
\item Takes culture and historical context into consideration
\item Rational choice theory applied
\item Political economy: the state can have a varying role in economic matters
\end{itemize}
\subsubsection{New Institutionalism (Past 25 years)}
\label{sec:org1f20f18}
\begin{itemize}
\item Institutions are the nexus of political action
\item Institutions are dynamic that interact over time w other variables
\item Institutions comprise the surrounding environment \& sentiment
\end{itemize}

\subsection{Video 2}
\label{sec:orgb6e0a6c}
\subsubsection{The Study of Comparative Politics}
\label{sec:org45ad862}
\begin{itemize}
\item Comparative politics implies a method of study or an approach to an analysis, not a single theory
\item greatest challenge is that events occur in real time with unreplicable environments
\item events in politics can not be replicated to test for validity
\end{itemize}
\subsubsection{Goals}
\label{sec:orgcac4e41}
\begin{itemize}
\item Goal: To assess which factors cause a certain outcome by comparing or contrasting cases
\item Cases: One of the group of things (events, states, actors, etc.) to be studied
\item Variable: a factor that changes over time or in different cases
\begin{itemize}
\item Independent var: causal var
\item Dependent var: outcome var
\end{itemize}
\item Causal relationships can be shown as:
\begin{itemize}
\item Cause -> effect
\item Independent var -> dependent var
\item Explanators var -> outcome
\item x var -> y var
\end{itemize}
\item Hypothesis: a possible answer that explains a causal effect
\end{itemize}
\subsubsection{Challenges}
\label{sec:org353ad5e}
\begin{itemize}
\item Goal: to determine causality, not just correlation

\item In comparative politics, the researcher may not be able to:
\begin{itemize}
\item have a constant
\item measure certain variables
\item anticipate certain events
\item disentangle one variable from others
\item Access to cases \& information
\begin{itemize}
\item Langauage barriers
\item Time \& funding
\item Sufficient cases (and selection bias)
\item IRB (Institutional Review Board)
\end{itemize}
\end{itemize}
\item Correlation: when var A occurs with var B, one is not caused by the other
\item Endogeneity: when it cannot be determined whether an outcome was caused by another factor
or the outcome caused that factor to occur
\end{itemize}
\subsection{Video 3}
\label{sec:org34ce630}
\subsubsection{Most Similar Systems Design (MSS)}
\label{sec:org52ade9a}
\begin{itemize}
\item A method in which as many independent vars as possible are held constant to explain a political
outcome: similar cases, different outcomes can help isolate a variable
\item Special Variation of MSS: Within-Case Comparison
\begin{itemize}
\item Single case analyzed over time or in different geographical areas
\item Breaks up a single case into subparts and allows for comparison
\end{itemize}
\end{itemize}
\subsubsection{Most-Different Systems Design (MDS)}
\label{sec:org9d72902}
\begin{itemize}
\item Looks at cases that are different from one another and observes why the same political outcome is
observed as a method of understanding how to isolate a single causal variable
\end{itemize}
\subsubsection{Overview}
\label{sec:org77d9844}
\begin{itemize}
\item Probable causal explanations (hypotheses): goal of these comparative approaches
\item Theories can be built from the strongest hypothesis
\item Theories can further be generalized based on the case
\end{itemize}
\end{document}