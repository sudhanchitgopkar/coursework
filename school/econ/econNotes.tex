% Created 2020-10-01 Thu 10:08
% Intended LaTeX compiler: pdflatex
\documentclass[11pt]{article}
\usepackage[utf8]{inputenc}
\usepackage[T1]{fontenc}
\usepackage{graphicx}
\usepackage{grffile}
\usepackage{longtable}
\usepackage{wrapfig}
\usepackage{rotating}
\usepackage[normalem]{ulem}
\usepackage{amsmath}
\usepackage{textcomp}
\usepackage{amssymb}
\usepackage{capt-of}
\usepackage{hyperref}
\author{Sudhan Chitgopkar}
\date{\today}
\title{}
\hypersetup{
 pdfauthor={Sudhan Chitgopkar},
 pdftitle={},
 pdfkeywords={},
 pdfsubject={},
 pdfcreator={Emacs 26.3 (Org mode 9.1.9)}, 
 pdflang={English}}
\begin{document}

\tableofcontents

\section{Chapter 5}
\label{sec:orgf876883}
\subsection{Introduction}
\label{sec:orgbeecfe5}
\begin{itemize}
\item Efficiency: Are we getting the most that we can out of our scarce resources?
\item Equity: Is what we're getting out of our resources fairly dstributed?
\end{itemize}
\subsection{Resource Allocation Methods}
\label{sec:org08c0535}
\begin{itemize}
\item Scarce resources might be allocated by
\begin{itemize}
\item Market price
\item Command (government, organizations and their hierarchical structures, rations, etc.)
\item Majority rule
\item Contest
\item First come, first served
\item Lottery
\item Force
\end{itemize}
\end{itemize}
\subsection{Demand and Consumer Surplus}
\label{sec:orgcd76caa}
\begin{itemize}
\item Demand, Willingness to Pay, and Value
\begin{itemize}
\item Value is what we get, price is what we pay
\item The value of one more unit of a good or service is its marginal benefit
\item The maxumum price that a person is willing to pay reveals marginal benefit
\item The demand curve is a marginal benefit curve
\end{itemize}
\item Individual Demand and Market Demand
\begin{itemize}
\item The relationship between the price of a good and the quantity demanded
\begin{itemize}
\item by one person: individual demand
\item by all buyers in the market: market demand
\end{itemize}
\item The market demand curve is the horizontal sum of individual demand curves
\end{itemize}
\item Consumer Surplus
\begin{itemize}
\item the excess of the benefit recieved from a good over the amount paid for it
\item Calculate as the marginal benefit of a good - price, summed over quantity bought
\item Market consumer surplus is the sum of individual consumer surplus
\end{itemize}
\end{itemize}
\subsection{Supply and Producer Surplus}
\label{sec:org13f6c9f}
\begin{itemize}
\item Supply and Marginal Cost
\begin{itemize}
\item To make a profit, firms must sell their output for a price > cost of production
\item Cost is what the producer gives up, price is what the producer recieves
\end{itemize}
\item Supply, Marginal Cost, and Minimum Supply-Price
\begin{itemize}
\item The cost of one more unit of a good or service is the marginal cost
\item The minimum price that a firm is willing to accept is its marginal cost
\item A supply curve is a marginal cost curve
\item The market supply curve is the horizontal sum of the individual supply
\end{itemize}
curves and is formed by adding the quantities supplied by all the producers at each price.
\item Producer surplus
\begin{itemize}
\item The excess of the amount recieved from a sale over the cost of production
\item Calculate as price - marginal cost, summed over quantity
\end{itemize}
\end{itemize}
\subsection{Is the Market Efficient?}
\label{sec:org7e03d2f}
\begin{itemize}
\item Efficiency of Competitive Equilibrium
\begin{itemize}
\item Resources are allocated efficienty when marginal social benefit = marginal social cost
\item If nobody other than producers and consumers are effected, the competitive equilibrium
can allocate resources efficiently
\end{itemize}
\end{itemize}
\subsection{Underproduction and Overproduction}
\label{sec:org971d01b}
\begin{itemize}
\item Market failure occurs upon an inefficient outcome (overproduction or underproduction)
\item Deadweight loss is the quantification of inefficiency by calculating the area of the 
full triangle before or after the equilibrium on a marginal social benefit \& cost curve
\end{itemize}
\subsection{Market Failure}
\label{sec:org6ae352a}
\begin{itemize}
\item Sources of Market Failure:
\begin{itemize}
\item Price and quantity regulations -> blocks price \& production, leads to underproduction
\item Taxes and subsidies -> taxes lead to underproduction, subsidies lead to overproduction
\item Externalities -> a cost/benefot affecting someone other than seller/buyer, leads to either
underproduction or overproduction
\item Public Goods and Common Resources
\begin{itemize}
\item Public goods: benefit everyone, nobody can be excluded. Nobody wants to pay for a public
good, leading to underproduction.
\item Common resouce: owned by nobody, but can be used by everyone. Leads to tragedy of the commons
and overproduction
\item Monopoly -> self-interest to produce profits results in underproduction
\item High Transaction costs -> leads to underproduction
\end{itemize}
\end{itemize}
\end{itemize}
\subsection{Fairness}
\label{sec:org1d30c82}
\begin{itemize}
\item Ideas of fairness can be divided into two rules
\begin{itemize}
\item Not fair if the result isn't fair
\begin{itemize}
\item Utilitarianism: greatest happiness for greatest number
\end{itemize}
\item Not far if the rules aren't fair
\end{itemize}
\end{itemize}
\subsubsection{It's not Fair if the Results aren't Fair}
\label{sec:org31ec18e}
\begin{itemize}
\item If everyone gets the same marginal utility from a given amount of income, and 
if the marginal benefit of income decreases as income increases, then taking a dollar from a 
richer person and giving it to a poorer person increases total benefit
\item Only when income is equally distributed has the greatest happiness been achieved
\item Utlitarianism ignores the cost of making income transfers
\item Recognizing these costs leads to the big tradeoff between efficiency and fairness
\end{itemize}
\subsubsection{It's not Fair if Rules aren't Fair}
\label{sec:orga46c5df}
\begin{itemize}
\item Symmetry principle: the requirement that people in similar situation be treated similarly
\item Nozick suggests that fairness is based on two rules
\begin{itemize}
\item The state must create and enforce laws that establish/protect private property
\item Private property may be transferred form one person to another only by voluntary exchange
\end{itemize}
\end{itemize}
\section{Chapter 4}
\label{sec:org7d3bac9}
\subsection{Introduction to Elasticity}
\label{sec:orgdcdbe3c}
\begin{itemize}
\item closeness of substitutes is critical to understanding elasticity of supply and demand
\end{itemize}
\subsection{Elasticity of Demand}
\label{sec:orge54ba89}
\subsubsection{Calculting Elasticity of Demand}
\label{sec:org5859a5c}
\begin{itemize}
\item Price elasticity of demand is a unit free measure of the responsiveness of quantity 
demanded to a change in price when all other influences stay the same
\item percentage change in quantity demanded/percentage change in price
\item percent change in price is calculated as change in price/average of two goods/services
\end{itemize}
\subsubsection{Inelastic and Elastic Demand}
\label{sec:org34f03c1}
\begin{itemize}
\item Demand can be inelastic, unit elastic, or elastic
\item Elasticity can range from 0 to infinity
\item If quantity demanded doesn't change when the price changes, price elasticity = 0 and the good
has perfectly inelastic demand (Vertical demand curve)
\item If price elasticity equals exactly one, the good has unit elastic demand
\item If price elasticity of demand is less than 1 then the good has inelastic demand
\item If price elasticity is greater than 1, then the good has an elastic demand
\item If the price elasticity is infinity, the good has a perfectly 
elastic demand (Horizontal demand curve)
\end{itemize}
\subsection{Factors Influencing Elasticity of Demand}
\label{sec:orgd940910}
\subsubsection{Closeness of substitutes}
\label{sec:org8e4a898}
\begin{itemize}
\item the closer the substitutes, the more elastic the demand for a good or service
\item necessities, such as food or housing, generally have an inelastic demands
\item luxuries, such as exotic vacations, generally have elastic demand
\end{itemize}
\subsubsection{Proportion of Income Spent on Good}
\label{sec:org5834dc7}
\begin{itemize}
\item The greater the portion of income consumers spend on a good, the larger the elasticity of demand
\end{itemize}
\subsubsection{Time Elapsed Since Price Change}
\label{sec:org8eea642}
\begin{itemize}
\item The more time consumers have to adjust to a price change or the longer the good can be stored
without losing its value, the more elastic the demand for the good
\end{itemize}
\subsection{Elasticity on a Linear Demand Curve \& Total Revenue Test}
\label{sec:orge03b0a1}
\begin{itemize}
\item At the midpoint of a linear demand curve, demand is unit elastic
\item At prices above the midpoint, demand is elastic
\item At prices below the midpoint, demand is inelastic
\end{itemize}
\subsubsection{Total Revenue and Elasticity}
\label{sec:org62a1803}
\begin{itemize}
\item Total revenue from the sale of a good or service = price of good * quantity sold
\item Raising the price doesn't always increase total revenue
\item If demand is elastic, a 1\% price cut increases quantity sold by >1\%, total revenue decreases
\item If demand is inelastic, a 1\% price cut increases the quantity <1\%, total revenue decreases
\item If demand is unit elastic a 1\% price cut increases the quantity sold by 1\%, total revenue same
\end{itemize}
\subsubsection{Total Revenue Test}
\label{sec:org9345157}
\begin{itemize}
\item a method of estimating the price elasticity of demand by
observing the change in total revenue that results from a price change
\item If a price cut increases total revenue, demand is elastic
\item If price cut decreases total revenue demand is inelastic
\item If a price cut doesn't change total revenue, demand is unit elastic
\item On a bell curve, increase shows elastic, decrease shows inelastic, and peak is unit elastic
\end{itemize}
\subsection{Income Elasticity and Cross Elasticity of Demand}
\label{sec:org1da2fae}
\subsubsection{Income Elasticity}
\label{sec:orgceaf117}
\begin{itemize}
\item Income elasticity of demand measures how the quantity demanded responds to a change in income
\begin{itemize}
\item \% change in quantity demanded/ \% change in income
\end{itemize}
\item If income elasticity is >1, demand is income elastic and the good is a normal good
\item If the income elasticity is 0<x<1, demand is income inelastic and the good is normal elastic
\item If income elasticity is <0, the good is an inferior good
\end{itemize}
\subsubsection{Cross Elasticity of Demand}
\label{sec:org7be6862}
\begin{itemize}
\item Measure of the responsiveness of demand to change in the price of a substitute/complement 
\begin{itemize}
\item \% change in quantity demanded/ \% change in price of substitute/complement
\end{itemize}
\item Cross elasticity of demand is:
\begin{itemize}
\item positive for a substitute
\item negative for a complement
\end{itemize}
\end{itemize}
\subsection{Elasticity of Supply}
\label{sec:org27750d1}
\begin{itemize}
\item Elasticity of supply: measures the responsiveness of quantity suppled to a change in price
\begin{itemize}
\item \% change in quantity supplied / \% change in price
\end{itemize}
\item Supply is perfectly inelastic when supply curve is vertical and elasticity = 0
\item Supply is unit elastic if the supply curve is linear and passes through the origin
\item Supply is perfectly elastic when the supply curve is elastic and the elasticity = infinity
\end{itemize}
\subsubsection{Factors Influencing Elasticity of Supply}
\label{sec:org1187c8d}
\begin{itemize}
\item Depends on
\begin{itemize}
\item Resource substitution possibilities
\begin{itemize}
\item The easier it is to substitute among resources used, the greater the elasticity of supply
\end{itemize}
\item Time frame for supply decision
\begin{itemize}
\item Momentary supply - perfectly inelastic for physical goods
\item Short-run supply is somewhat elastoc
\item Long-run supply is the most elastic
\end{itemize}
\end{itemize}
\end{itemize}
\section{Chapter 3}
\label{sec:org6fdb20b}
\subsection{Introduction}
\label{sec:org1c27e74}
\begin{itemize}
\item Markets are any arrangements that enable buyers and sellers to get information
and do business with each other
\item Competitive Market: many buyers and many sellers so no single buyer or seller can
influence prices
\end{itemize}
\subsection{Demand}
\label{sec:orgd609e39}
\begin{itemize}
\item Reflects the buyers' side of the market
\item If you demand something, you
\begin{itemize}
\item want it
\item can afford it
\item have a definite plan to buy it
\end{itemize}
\item Quantity demanded: amount that consumers plan to buy 
during a particular time @ a particular price
\item Law of Demand: other things remaining the same, the higher the price of a good, the smaller
the quantity demanded (and vice versa)
\item Substitution Effect: when the relative price of a good rises, people seek substitutes so
the quantity demanded decreases
\item When the price of a good rises relative to income, people cannot afford all the things
they previously bought so quantity demanded decreases
\item Demand Curve and Demand Schedule
\begin{itemize}
\item the term demand refers to the entire relationship between good and quantity demanded
\end{itemize}
\item Demand Curve: exhibits relationshit between quantity demanded and price when all other
consumers' planned purchases remain constant
\item Willingess and Ability to Pay
\begin{itemize}
\item The smaller the quantity available, the higher the price someone is willing to pay for
another unit
\item Willingness to pay measures marginal benefit
\end{itemize}
\item Changes in Demand: when some influence on buying plans other than price changes, there is a
shift in demand for that good
\item 6 factors influencing demand:
\begin{itemize}
\item Price of related goods
\begin{itemize}
\item substitutes - good that can be used in place of another
\item complement - good that is used in conjunction with another
\item If \$ substitute inc or \$ complement dec, demand of good inc
\item if \$ substitute dec or \$ complement inc, demand of good dec
\end{itemize}
\item Expected future prices
\begin{itemize}
\item if expected future price inc, current demand inc
\item if expected future price dec, current demand dec
\end{itemize}
\item Income
\begin{itemize}
\item normal good: a good for which demand inc as income inc
\item inferior good: a good for which demand dec as income inc
\item if expected future income increases/credit is easier to get, current demand inc
\end{itemize}
\item Population
\begin{itemize}
\item The higher the population, the higher the demand
\end{itemize}
\item Preferences
\begin{itemize}
\item People with the same income have different demands if they have different preferences
\end{itemize}
\end{itemize}
\end{itemize}
\subsection{Supply}
\label{sec:orgef63761}
\begin{itemize}
\item If a firm is a supplier, they
\begin{itemize}
\item have the resources and tech to produce it
\item can profit from producing it
\item has a definite plan to produce and sell it
\end{itemize}
\item Quantity supplied: the amount producers plan to sell during a given time at a particular price
\item Law of Supply: Other things remaning the same, the higher the price of a good, the greater the
quantity supplied (and vice versa).
\item Supply Curve and Supply Schedule
\begin{itemize}
\item Minimum supply price: As quantity produced inc, marginal cost inc.
\item The lowest price at which someone is willing to sell an additional unit rises
\item This lowest price is called the marginal cost
\end{itemize}
\item Changes in Supply
\begin{itemize}
\item Increases in supply shifts the curve to the right (and vice versa)
\end{itemize}
\item Factors that affect Supply
\begin{itemize}
\item Prices of factors of production
\begin{itemize}
\item If the price of an input inc, supply dec; curve shifts left
\end{itemize}
\item Prices of related goods produced
\begin{itemize}
\item denoted by substitute for production, not just substitute
\item supply of a good inc if price of a substitute dec
\item complements in production: goods that must be produced together (beef \& leather)
\item supply of a good inc if the price of a complement in production inc
\end{itemize}
\item Expected Future Prices
\begin{itemize}
\item If expected future price inc, current supply dec
\end{itemize}
\item Number of Suppliers
\begin{itemize}
\item as number of suppliers inc, supply inc
\end{itemize}
\item Technology
\begin{itemize}
\item Advances in technology lower the cost of making existing products
\item inc in technology means inc in supply
\end{itemize}
\item State of Nature
\begin{itemize}
\item natural forces and disasters can dec supply
\end{itemize}
\end{itemize}
\end{itemize}
\subsection{Equilibrium}
\label{sec:org6b503c0}
\begin{itemize}
\item Equilibrium: a situation in which opposing forces balance each other
\item Equilibrium Price: the price at which quantity demanded = quantity supplied
\item Equilibrium Quantity: quantity bought and sold at equilibrium cost
\item Price Regulation
\begin{itemize}
\item Price regulates buying and selling plans
\item Price adjusts when plans don't match
\end{itemize}
\item Price adjustments
\begin{itemize}
\item Surplus forces prices down
\item Shortage forces prices up
\end{itemize}
\item Increases in demand
\begin{itemize}
\item When demand increases without changes in supply, shortages occur
\item Price therefore increaes
\end{itemize}
\item Decrease in demand
\begin{itemize}
\item At the original price, there is a surplus
\item Price therefore falls
\end{itemize}
\item Increase in supply
\begin{itemize}
\item At the original price, there is a surplus
\item Price therefore falls
\end{itemize}
\item Decrease in supply
\begin{itemize}
\item At the original price, there is a shortage
\item Price therefore increases
\end{itemize}
\end{itemize}
\section{Chapter 2}
\label{sec:orgcdc68d9}
\subsection{Production Possibilities Frontier}
\label{sec:org5871fc5}
\begin{itemize}
\item PPF is the boundary between combinations of goods and services that can and can't be prodiced
\item Points outside the PPF are unattainable
\end{itemize}
\subsubsection{Production Efficiency}
\label{sec:org07a9c18}
\begin{itemize}
\item We can achieve production efficiency if we cannt make more of one good without making les
of another such good.
\item All points on the PPF are efficient, while all points within the PPF are inefficient
\end{itemize}
\subsection{Opportunity Cost on the PPF}
\label{sec:orgbe68917}
\begin{itemize}
\item Every choice/movement along the PPF is an opportunity cost
\item Opportunity Cost = Amnt given up/Amnt gained
\item Opportunity cost increases as we move along the PPF
\begin{itemize}
\item Because resources are not equally productive for all activities, the PPF bows outwards
\item The outward bow of the PPF means that as the quantity of each good increases, so does 
the opportunity cost
\end{itemize}
\end{itemize}
\subsection{Marginal Costs}
\label{sec:org5391c35}
\begin{itemize}
\item Marginal Cost: The opportunity cost of producing one more unit of that good
\item Marginal Cost curve slopes upward for the same reason that the PPF bows outward
\end{itemize}
\subsection{Marginal Benefits}
\label{sec:org709480a}
\begin{itemize}
\item Preferences: A description of a person's likes and dislikes
\item Marignal benefit: the benefit recieved from consuming one more unit of that good
\item Marginal benefot is measured by the amount that a person is willing to pay for one more unit
of a particular good or service
\item Principle of Decreasing Marginal Benefit: The more we have of any good, the smaller the marginal
benefit of that good
\end{itemize}
\subsection{Allocative Efficiency}
\label{sec:orgf726ac6}
\begin{itemize}
\item When we cannot produce more of any one good without giving up some other good that we value
more highly
\item Point at which marginal cost and marginal benefit curve meet
\end{itemize}
\subsection{Comparative \& Absolute advantage}
\label{sec:orgafbcd43}
\begin{itemize}
\item Comparative advantage: When a person can perform an activity at a lower opportunity cost than
anyone else
\item Absolute advantage: When a person is more productiv than others
\end{itemize}
\subsection{Economic Growth}
\label{sec:org9acf5c6}
\begin{itemize}
\item Two key factors:
\begin{itemize}
\item Technnological Change
\item Capital accumulation (growth of capital resources)
\end{itemize}
\item Economic growth is not free, investing in tech and capital costs production today but helps
production tomorrow through smart investment
\end{itemize}
\subsection{Cricular Flow Model}
\label{sec:org3e15994}
\begin{itemize}
\item Need:
\begin{itemize}
\item Firms (take input, make output)
\item Markets
\item Property Rights
\item Money
\end{itemize}
\end{itemize}
\section{Chapter 1}
\label{sec:org38a6294}
\subsection{Scarcity}
\label{sec:org8de81e1}
\begin{itemize}
\item all economic questions arise because we want more than we can get
\item inability to satisfy all wants because of scarcity
\item scarcity = limited resources
\end{itemize}
\subsection{Definition of Economics}
\label{sec:orgefed7c4}
\begin{itemize}
\item because we face scarcity, we must make choices
\item incentive = a reward that encourages an action or a penalty that discourages an action
\item economics is the social science that studies the choices that individuals, businesses, etc.
make as they cope with scarcity and the incentives that influence and reconcile those choices
\item Economics divides into two parts:
\begin{itemize}
\item Microeconomics = study of choices that individuals and businesses make \& how those choices
interact with markets and the influence of governments
\item Macroeconomics = the study of the performance of national and global economies
\end{itemize}
\end{itemize}
\subsection{6 Key Ideas}
\label{sec:orgb959a79}
\begin{itemize}
\item a choice is a tradeoff: ever choice is an exchange giving up one thing for another
\item making a rational choice: a rational choice compares costs and benefits, maximizing benefit
\item benefit = what you gain: the gain or pleasure something brings about, determined by preferences
\begin{itemize}
\item preferences = what a person likes, dislikes, and the intensity of those feelings
\end{itemize}
\item cost = what must be given up
\begin{itemize}
\item opportunity cost = highest val alternative that must be given up
\end{itemize}
\item choosing at the margin: the benefit of pursuing an incremental increase in some action
is marginal benefit of that action
\begin{itemize}
\item the opportunity cost of pursuing an incremental increase in some action is marginal cost
\item if marginal benefit > marginal cost, rational choice is to do more of that action
\end{itemize}
\item choices respond to incentives: a change in marginal cost/benefit changes our incentives \& choices
\end{itemize}
\subsection{Positive \& Normative}
\label{sec:orgee30ce9}
\begin{itemize}
\item economists distinguish between two types of statements: 
\begin{itemize}
\item positive statements: can be tested by checking the facts
\item normative statements: express an untestable opinion
\end{itemize}
\item economists as social scientists
\begin{itemize}
\item economists test economic models
\item economic model = a description of some aspect of the world w only the necessary features
\end{itemize}
\item economists as policy advisors
\end{itemize}
\subsection{Resources \& Highest Valued Use}
\label{sec:org4e0b75c}
\begin{itemize}
\item the scope of economics: 
\begin{itemize}
\item how do choices end up determining "what, how, and for whom" goods and services get produced
\end{itemize}
\item goods and services are produced using productive resources called factors of production
\begin{itemize}
\item land
\item labor
\item capital
\item entrepreneurship
\end{itemize}
\item who gets goods and services depends on income
\begin{itemize}
\item land earns rent, labor earns wages, capital earns interest, entrepreneruship earns profit
\end{itemize}
\item \textbf{\textbf{resources gravitate towards their highest value use}}
\end{itemize}
\subsection{Self Interest \& Social Interest}
\label{sec:orgdc30bbd}
\begin{itemize}
\item self interest = choices that are made because you think they are the best for you
\item social interest = choices that are best for society as a whole
\item social interest has two dimensions: 
\begin{itemize}
\item efficiency: resource use is efficient if it is not possible to make someone better off without
making someone else worse off (no waste to be eliminated)
\item fair shares/equity: refers to the fairness with which resource division occurs in a society
\end{itemize}
\item tension between self \& social interest: information revolution, climate change, globalization
\end{itemize}
\end{document}