% Created 2020-09-10 Thu 15:53
% Intended LaTeX compiler: pdflatex
\documentclass[11pt]{article}
\usepackage[utf8]{inputenc}
\usepackage[T1]{fontenc}
\usepackage{graphicx}
\usepackage{grffile}
\usepackage{longtable}
\usepackage{wrapfig}
\usepackage{rotating}
\usepackage[normalem]{ulem}
\usepackage{amsmath}
\usepackage{textcomp}
\usepackage{amssymb}
\usepackage{capt-of}
\usepackage{hyperref}
\author{Sudhan Chitgopkar}
\date{\today}
\title{}
\hypersetup{
 pdfauthor={Sudhan Chitgopkar},
 pdftitle={},
 pdfkeywords={},
 pdfsubject={},
 pdfcreator={Emacs 26.3 (Org mode 9.1.9)}, 
 pdflang={English}}
\begin{document}

\tableofcontents

\section{Chapter 3}
\label{sec:org3bcfca5}
\subsection{Introduction}
\label{sec:orgb8a3986}
\begin{itemize}
\item Markets are any arrangements that enable buyers and sellers to get information
and do business with each other
\item Competitive Market: many buyers and many sellers so no single buyer or seller can
influence prices
\end{itemize}
\subsection{Demand}
\label{sec:org713fffd}
\begin{itemize}
\item Reflects the buyers' side of the market
\item If you demand something, you
\begin{itemize}
\item want it
\item can afford it
\item have a definite plan to buy it
\end{itemize}
\item Quantity demanded: amount that consumers plan to buy 
during a particular time @ a particular price
\item Law of Demand: other things remaining the same, the higher the price of a good, the smaller
the quantity demanded (and vice versa)
\item Substitution Effect: when the relative price of a good rises, people seek substitutes so
the quantity demanded decreases
\item When the price of a good rises relative to income, people cannot afford all the things
they previously bought so quantity demanded decreases
\item Demand Curve and Demand Schedule
\begin{itemize}
\item the term demand refers to the entire relationship between good and quantity demanded
\end{itemize}
\item Demand Curve: exhibits relationshit between quantity demanded and price when all other
consumers' planned purchases remain constant
\item Willingess and Ability to Pay
\begin{itemize}
\item The smaller the quantity available, the higher the price someone is willing to pay for
another unit
\item Willingness to pay measures marginal benefit
\end{itemize}
\item Changes in Demand: when some influence on buying plans other than price changes, there is a
shift in demand for that good
\item 6 factors influencing demand:
\begin{itemize}
\item Price of related goods
\begin{itemize}
\item substitutes - good that can be used in place of another
\item complement - good that is used in conjunction with another
\item If \$ substitute inc or \$ complement dec, demand of good inc
\item if \$ substitute dec or \$ complement inc, demand of good dec
\end{itemize}
\item Expected future prices
\begin{itemize}
\item if expected future price inc, current demand inc
\item if expected future price dec, current demand dec
\end{itemize}
\item Income
\begin{itemize}
\item normal good: a good for which demand inc as income inc
\item inferior good: a good for which demand dec as income inc
\item if expected future income increases/credit is easier to get, current demand inc
\end{itemize}
\item Population
\begin{itemize}
\item The higher the population, the higher the demand
\end{itemize}
\item Preferences
\begin{itemize}
\item People with the same income have different demands if they have different preferences
\end{itemize}
\end{itemize}
\end{itemize}
\subsection{Supply}
\label{sec:org315b73d}
\begin{itemize}
\item If a firm is a supplier, they
\begin{itemize}
\item have the resources and tech to produce it
\item can profit from producing it
\item has a definite plan to produce and sell it
\end{itemize}
\item Quantity supplied: the amount producers plan to sell during a given time at a particular price
\item Law of Supply: Other things remaning the same, the higher the price of a good, the greater the
quantity supplied (and vice versa).
\item Supply Curve and Supply Schedule
\begin{itemize}
\item Minimum supply price: As quantity produced inc, marginal cost inc.
\item The lowest price at which someone is willing to sell an additional unit rises
\item This lowest price is called the marginal cost
\end{itemize}
\item Changes in Supply
\begin{itemize}
\item Increases in supply shifts the curve to the right (and vice versa)
\end{itemize}
\item Factors that affect Supply
\begin{itemize}
\item Prices of factors of production
\begin{itemize}
\item If the price of an input inc, supply dec; curve shifts left
\end{itemize}
\item Prices of related goods produced
\begin{itemize}
\item denoted by substitute for production, not just substitute
\item supply of a good inc if price of a substitute dec
\item complements in production: goods that must be produced together (beef \& leather)
\item supply of a good inc if the price of a complement in production inc
\end{itemize}
\item Expected Future Prices
\begin{itemize}
\item If expected future price inc, current supply dec
\end{itemize}
\item Number of Suppliers
\begin{itemize}
\item as number of suppliers inc, supply inc
\end{itemize}
\item Technology
\begin{itemize}
\item Advances in technology lower the cost of making existing products
\item inc in technology means inc in supply
\end{itemize}
\item State of Nature
\begin{itemize}
\item natural forces and disasters can dec supply
\end{itemize}
\end{itemize}
\end{itemize}
\subsection{Equilibrium}
\label{sec:orgac325e5}
\begin{itemize}
\item Equilibrium: a situation in which opposing forces balance each other
\item Equilibrium Price: the price at which quantity demanded = quantity supplied
\item Equilibrium Quantity: quantity bought and sold at equilibrium cost
\item Price Regulation
\begin{itemize}
\item Price regulates buying and selling plans
\item Price adjusts when plans don't match
\end{itemize}
\item Price adjustments
\begin{itemize}
\item Surplus forces prices down
\item Shortage forces prices up
\end{itemize}
\item Increases in demand
\begin{itemize}
\item When demand increases without changes in supply, shortages occur
\item Price therefore increaes
\end{itemize}
\item Decrease in demand
\begin{itemize}
\item At the original price, there is a surplus
\item Price therefore falls
\end{itemize}
\item Increase in supply
\begin{itemize}
\item At the original price, there is a surplus
\item Price therefore falls
\end{itemize}
\item Decrease in supply
\begin{itemize}
\item At the original price, there is a shortage
\item Price therefore increases
\end{itemize}
\end{itemize}
\section{Chapter 1}
\label{sec:orgc672c05}
\subsection{Scarcity}
\label{sec:org8005ab7}
\begin{itemize}
\item all economic questions arise because we want more than we can get
\item inability to satisfy all wants because of scarcity
\item scarcity = limited resources
\end{itemize}
\subsection{Definition of Economics}
\label{sec:orgb6eeeb2}
\begin{itemize}
\item because we face scarcity, we must make choices
\item incentive = a reward that encourages an action or a penalty that discourages an action
\item economics is the social science that studies the choices that individuals, businesses, etc.
make as they cope with scarcity and the incentives that influence and reconcile those choices
\item Economics divides into two parts:
\begin{itemize}
\item Microeconomics = study of choices that individuals and businesses make \& how those choices
interact with markets and the influence of governments
\item Macroeconomics = the study of the performance of national and global economies
\end{itemize}
\end{itemize}
\subsection{6 Key Ideas}
\label{sec:org584e961}
\begin{itemize}
\item a choice is a tradeoff: ever choice is an exchange giving up one thing for another
\item making a rational choice: a rational choice compares costs and benefits, maximizing benefit
\item benefit = what you gain: the gain or pleasure something brings about, determined by preferences
\begin{itemize}
\item preferences = what a person likes, dislikes, and the intensity of those feelings
\end{itemize}
\item cost = what must be given up
\begin{itemize}
\item opportunity cost = highest val alternative that must be given up
\end{itemize}
\item choosing at the margin: the benefit of pursuing an incremental increase in some action
is marginal benefit of that action
\begin{itemize}
\item the opportunity cost of pursuing an incremental increase in some action is marginal cost
\item if marginal benefit > marginal cost, rational choice is to do more of that action
\end{itemize}
\item choices respond to incentives: a change in marginal cost/benefit changes our incentives \& choices
\end{itemize}
\subsection{Positive \& Normative}
\label{sec:orgbc25aeb}
\begin{itemize}
\item economists distinguish between two types of statements: 
\begin{itemize}
\item positive statements: can be tested by checking the facts
\item normative statements: express an untestable opinion
\end{itemize}
\item economists as social scientists
\begin{itemize}
\item economists test economic models
\item economic model = a description of some aspect of the world w only the necessary features
\end{itemize}
\item economists as policy advisors
\end{itemize}
\subsection{Resources \& Highest Valued Use}
\label{sec:org43e93ca}
\begin{itemize}
\item the scope of economics: 
\begin{itemize}
\item how do choices end up determining "what, how, and for whom" goods and services get produced
\end{itemize}
\item goods and services are produced using productive resources called factors of production
\begin{itemize}
\item land
\item labor
\item capital
\item entrepreneurship
\end{itemize}
\item who gets goods and services depends on income
\begin{itemize}
\item land earns rent, labor earns wages, capital earns interest, entrepreneruship earns profit
\end{itemize}
\item \textbf{\textbf{resources gravitate towards their highest value use}}
\end{itemize}
\subsection{Self Interest \& Social Interest}
\label{sec:org4a4ef83}
\begin{itemize}
\item self interest = choices that are made because you think they are the best for you
\item social interest = choices that are best for society as a whole
\item social interest has two dimensions: 
\begin{itemize}
\item efficiency: resource use is efficient if it is not possible to make someone better off without
making someone else worse off (no waste to be eliminated)
\item fair shares/equity: refers to the fairness with which resource division occurs in a society
\end{itemize}
\item tension between self \& social interest: information revolution, climate change, globalization
\end{itemize}
\end{document}