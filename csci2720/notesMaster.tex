% Created 2021-01-19 Tue 11:29
% Intended LaTeX compiler: pdflatex
\documentclass[11pt]{article}
\usepackage[utf8]{inputenc}
\usepackage[T1]{fontenc}
\usepackage{graphicx}
\usepackage{grffile}
\usepackage{longtable}
\usepackage{wrapfig}
\usepackage{rotating}
\usepackage[normalem]{ulem}
\usepackage{amsmath}
\usepackage{textcomp}
\usepackage{amssymb}
\usepackage{capt-of}
\usepackage{hyperref}
\author{Sudhan Chitgopkar}
\date{\today}
\title{}
\hypersetup{
 pdfauthor={Sudhan Chitgopkar},
 pdftitle={},
 pdfkeywords={},
 pdfsubject={},
 pdfcreator={Emacs 27.1 (Org mode 9.5)}, 
 pdflang={English}}
\begin{document}

\tableofcontents

\section{01.19.20 (C++ Ch. 2)}
\label{sec:orgb874e81}
A look at some basic C++ code
\begin{verbatim}
#include <iostream> //enables program to output data

//main function begins program execution
int main () {
  //cout currently a function as a part of the std namespace
  std::cout << "Welcome to C++!\n";
  //above << is an insertion operator, overloaded from the bitwise left-shift

  return 0;
}
\end{verbatim}

A look at some higher level C++ code
\begin{verbatim}
#include <iostream>

int main () {

  int num1{0}; //list initialization
  int num2 = 0; //regular initialization
 //Note, there is no difference between list and regular initialization for primitive types. When using UDTs, though, list-initialization may be better than regular initialization

  int sum{0}

  std::cin >> num1;
  std::cin >> num2;

  sum = num1 + num2;

  std::cout << sum << std::endl;
  //endl is helpful because it flushes the buffer, which the newline character does not
  return 0;
}
\end{verbatim}
\end{document}
