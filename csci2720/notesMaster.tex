% Created 2021-01-21 Thu 11:35
% Intended LaTeX compiler: pdflatex
\documentclass[11pt]{article}
\usepackage[utf8]{inputenc}
\usepackage[T1]{fontenc}
\usepackage{graphicx}
\usepackage{grffile}
\usepackage{longtable}
\usepackage{wrapfig}
\usepackage{rotating}
\usepackage[normalem]{ulem}
\usepackage{amsmath}
\usepackage{textcomp}
\usepackage{amssymb}
\usepackage{capt-of}
\usepackage{hyperref}
\author{Sudhan Chitgopkar}
\date{\today}
\title{}
\hypersetup{
 pdfauthor={Sudhan Chitgopkar},
 pdftitle={},
 pdfkeywords={},
 pdfsubject={},
 pdfcreator={Emacs 27.1 (Org mode 9.5)}, 
 pdflang={English}}
\begin{document}

\tableofcontents \clearpage\section{01.21.20 (C++ Ch. 3)}
\label{sec:orgcda033b}
\subsection{Objects and Object Sizes}
\label{sec:orgaed776a}
\begin{itemize}
\item An objects size will always be the sum of its data members. The size will not be affected by any methods that are called upon it.
\item Because of this, objects can quickly become very large in size.
\end{itemize}
\subsection{UML Diagrams}
\label{sec:orgdd05fb0}
\begin{itemize}
\item Classes are listed as individual boxes
\begin{itemize}
\item top box = class name
\item middle compartment =
\end{itemize}
\end{itemize}
\subsection{Constructors}
\label{sec:orgc248836}
\begin{itemize}
\item Explicit constructors can be used to prevent implicit typecasting, as seen below:
\end{itemize}
\begin{verbatim}
class Student {
  Student (int s) {

  } //constructor
} //Student

int main () {
  Student s {15}; //allowed, completes correctly
  Student c {'C'}; //typecasts automatically, should not occur
  //Note, () can be used in place of {} to construct objects
}
\end{verbatim}

\begin{itemize}
\item Ex. list initialization with an explicit constructor
\end{itemize}
\begin{verbatim}
explicit Account (std::string accountName) //explicit constructor
  : name{accountName} {
  //insert constructor code here
  }
\end{verbatim}
\section{01.19.20 (C++ Ch. 3)}
\label{sec:org529f060}
A look at class creation
\begin{verbatim}
#include <iostream>
using namespace std;

//defining the class
class GradeBook {
  //holds all public vars, functions
  public:
  //public function
  void displayMessage() {
    cout << "Welcome to your Gradebook" << endl;
  } //displayMesage
} //GradeBook

//main method
int main () {
  //creates a GradeBook object
  GradeBook myGradeBook;
  //calls above created function on object
  myGradeBook.displayMessage();
}
\end{verbatim}

\begin{itemize}
\item Class functions and vars are, by default, private. The public keyword must be used to denote any public parts of a class.

\item Move implementations to a header file for use in main methods while separating out each file.

\item When using header files, use quotation marks around them to indicate that they're a file on your machine. Use angle brackets around things to include form the C std lib.

\item The purpose of const functions is to prevent the function from modifying the values of data members or objects.
\end{itemize}

\section{01.19.20 (C++ Ch. 2)}
\label{sec:orgb2b6948}
A look at some basic C++ code
\begin{verbatim}
#include <iostream> //enables program to output data

//main function begins program execution
int main () {
  //cout currently a function as a part of the std namespace
  std::cout << "Welcome to C++!\n";
  //above << is an insertion operator, overloaded from the bitwise left-shift

  return 0;
}
\end{verbatim}

A look at some higher level C++ code
\begin{verbatim}
#include <iostream>

int main () {

  int num1{0}; //list initialization
  int num2 = 0; //regular initialization
 //No difference between list & regular initializtion with primitive types.
 //List initialization should be used for UDTs.

  int sum{0}

  std::cin >> num1;
  std::cin >> num2;

  sum = num1 + num2;

  std::cout << sum << std::endl;
  //endl is helpful because it flushes the buffer
  //newline character does not
  return 0;
}
\end{verbatim}

A look at a common mistake
\begin{verbatim}
#include <iostream>

int main () {
  int x {5};

  if(x > 10); {
    std::cout << x "> 10" << std::endl;
  }
  //still prints output because of semicolon after if statement

  return 0;
}
\end{verbatim}
\end{document}
