% Created 2021-02-23 Tue 12:16
% Intended LaTeX compiler: pdflatex
\documentclass[11pt]{article}
\usepackage[utf8]{inputenc}
\usepackage[T1]{fontenc}
\usepackage{graphicx}
\usepackage{grffile}
\usepackage{longtable}
\usepackage{wrapfig}
\usepackage{rotating}
\usepackage[normalem]{ulem}
\usepackage{amsmath}
\usepackage{textcomp}
\usepackage{amssymb}
\usepackage{capt-of}
\usepackage{hyperref}
\usepackage[margin=1in]{geometry}
\author{Sudhan Chitgopkar}
\date{\today}
\title{Data Structures}
\hypersetup{
 pdfauthor={Sudhan Chitgopkar},
 pdftitle={Data Structures},
 pdfkeywords={},
 pdfsubject={},
 pdfcreator={Emacs 27.1 (Org mode 9.5)}, 
 pdflang={English}}
\begin{document}

\maketitle
\section*{02.23.21 (Stacks \& Queues)}
\label{sec:org5ccb063}
\begin{itemize}
\item Stacks operate in LIFO (Last in, First out)
\end{itemize}
\subsection*{Standard Operations}
\label{sec:orgc3f33bc}
\begin{itemize}
\item MakeEmpty
\item IsEmpty
\item IsFull
\item Push (Itemtype newItem) - Adds an item to the top of a stack, throws an exception if the stack is full
\item Pop - Removes an item from the top of the stack
\item ItemType Top - Returns a copy of the item at the top of the stack
\end{itemize}
\subsection*{Array-Based Stack}
\label{sec:org8d11118}
\begin{itemize}
\item Constructor initializes top = -1
\begin{itemize}
\item This is because push increments top first, leaving an index empty if top is set to 0
\end{itemize}
\item Top increments top by 1 and adds an item to the top index]
\item Pop decrements the top value by 1
\item IsEmpty checks if top = -1
\item IsFull checks if top = MAX\textsubscript{ITEMS}
\end{itemize}
\section*{02.22.21 (Sorted Linked Lists)}
\label{sec:org5e2e50f}
\subsection*{Searching}
\label{sec:org6fac6d0}
\begin{itemize}
\item We can use a modified linear search rather than a full linear search
\item The search can terminate after any value greater than the element sought becuase the list is sorted
\item If asked whether a binary search is possible on a sorted linked list, just answer no even though a binary search is possible (inefficiently) with sorted linked lists
\end{itemize}
\subsection*{Insertion}
\label{sec:orgb946ece}
\begin{itemize}
\item Begin with two pointers, one for search and one trailing
\item Compare item to insert with items already in the list, looking for insertion location
\item There also exist special cases for inserting in the first node
\begin{itemize}
\item Case 1: Adding a node to the beginning of the list
\item Case 2: Inserting a node to an empty list
\end{itemize}
\end{itemize}
\section*{02.22.21 (Unsorted Linked Lists)}
\label{sec:org58bca7f}
\subsection*{Unsorted Linked Lists}
\label{sec:org3aad7e4}
\begin{itemize}
\item MakeEmpty is the same as the destructor
\begin{itemize}
\item Involves going through each node and deleting it
\end{itemize}
\item Insertion occurs at the beginning of the linked list instead of at the end to allow for O(1) time, as opposed to the O(n) time
\item Search operations require a sequential search as the data is unsorted
\item Delete operations require two pointers
\begin{itemize}
\item Need to begin with a search operation for the node to delete
\item Use one pointer to search, one trailing pointer
\item Delete node pointed to by search when found, connect trailing pointer to node following search pointer
\item Special case occurs when the first node is deleted
\end{itemize}
\item Complexity of Operations
\begin{itemize}
\item Search: O(n)
\item Insert: O(1)
\item Delete: O(n)
\begin{itemize}
\item Searching is O(n)
\item Deleting is O(1)
\end{itemize}
\end{itemize}
\end{itemize}
\section*{02.08.21 - 02.09.21}
\label{sec:org9b8015d}
\subsection*{List ADT}
\label{sec:org6ce5670}
\begin{itemize}
\item List consists of a domain and a set of operations
\item Domain is composed of homogeneous elements
\item Elements should be able to be compared (comparison operation must be defined)
\item All elements except for the first and last elements have a predecessor and successor
\item Sorted lists are always implied to be sorted in ascending order
\item List operations
\begin{itemize}
\item Constructor
\item Transformer: (changes state)
\begin{itemize}
\item MakeEmpty, InsertItem,DeleteItem
\end{itemize}
\item Observer: (obsesrve state)
\begin{itemize}
\item IsFull, LengthIs, Min, Max, Average
\end{itemize}
\item Iterator: GetNextItem (note that a ResetList is generally called before GetNextItem)
\end{itemize}
\end{itemize}
\subsection*{Generic Data Type}
\label{sec:org39a4988}
\begin{itemize}
\item A generic data type is a type for which the operations are defined but the types of the items being manipulated are not defined
\end{itemize}
\subsection*{Efficient Operations in Unsorted Lists}
\label{sec:org5e8147b}
\begin{itemize}
\item When dealing with unsorted, array-based lists of a data-type,
\begin{itemize}
\item insert operations should always be done at the end of the list through insert(length)
\begin{itemize}
\item this is because inserting at the beginning requires us to push all items to the right one unit which is O(n), not O(1)
\end{itemize}
\item delete operations should (1) find the element, (2) delete the element, and (3) move the last term to the index of the deleted term
\begin{itemize}
\item this allows us to prevent moving through the array and refactoring it post-deletion
\end{itemize}
\end{itemize}
\begin{itemize}
\item Operation Complexity:
\begin{itemize}
\item Search: O(n)
\item Insert: O(1)
\item Delete: O(n)
\end{itemize}
\end{itemize}
\end{itemize}
\subsection*{Efficient Operations in Sorted Lists}
\label{sec:orge0fc630}
\begin{itemize}
\item GetItem, PutItem, and DeleteItem implementations change
\item For GetItem, we no longer need to conduct a linear search with O(n) time. Instead, we can use a binary search with O(log n) time.
\item When using a linked-list, it is not possible to run a binary search due to the lack of indices in linked lists. Here, we will use a modified linear search instead that terminates after reaching the item or an itwm with a value greater than it.
\item PutItem requires us to find the first index with a value greater than the item to be inserted. All elements including and after that index are shifted to the right. Finally, the necessary item is inserted at its proper location. This is O(n) because the search is O(n), the shift is O(n), and they are not nested loops
\end{itemize}
\section*{02.01.21 (Recursion)}
\label{sec:orgc18c614}
\subsection*{Recursion Trees}
\label{sec:org353d357}
\begin{itemize}
\item To convert recurrences into a tree,
\begin{itemize}
\item each node represents the cost incurred at various levels of recursion
\item Sum up the costs of all levels
\end{itemize}
\item Complexity of a recursive function is determined by the amount of recursive calls
\item To solve a recurrence relationship, we find a closed form for it or use a master method
\end{itemize}
\section*{01.28.21 (Algorithm Analysis)}
\label{sec:org250f1dc}
\subsection*{Experimental Analysis}
\label{sec:org0a5a70a}
\begin{itemize}
\item Algorithms = step-by-step procedure for solving a problem in a finite amount of time
\item Experimentation Steps:
\begin{itemize}
\item Write a program implementing the algorithm
\item Run the program with inputs of varying size, composition
\item Plot the results
\end{itemize}
\item Limitations of Experiments:
\begin{itemize}
\item Implementing the algorithm may be difficult
\item Results may not indicate running time on other inputs
\item Algorithm comparison is difficult
\end{itemize}
\item For this reason, theoretical results are preferred
\end{itemize}
\subsection*{Theoretical Analysis}
\label{sec:orgefded95}
\begin{itemize}
\item Theoretical Analysis
\begin{itemize}
\item Use a high level description instead of an implementation
\item Characterizes running time as a function of input size, n
\item Takes into account all possible inputs
\item Allows for algorithm comparison independent of hardware/software
\end{itemize}
\item Primitive Operations
\begin{itemize}
\item Count the amount of primitive/basic operations
\item These operations are
\begin{itemize}
\item identifiable in pseudocode
\item generally independent of programming language
\item want to focus on large operations such as loops
\end{itemize}
\end{itemize}
\item Asymptotic Complexity
\begin{itemize}
\item simply can be understood as Big-O
\item Generally fives us an idea of how rapidly the space/time requirements grow as problem size increases
\end{itemize}
\item Rate of Growth
\begin{itemize}
\item Because lower order terms become relatively insignificant for large n, we consider the actual function and its highest order term to have the same rate of growth
\end{itemize}
\end{itemize}
\section*{01.26.21 (ADTs \& Big-O)}
\label{sec:orgc88cc69}
\subsection*{Abstract Data Types}
\label{sec:orgfb63bf8}
\begin{itemize}
\item Abstract Data Type (ADT): A data type whose properties are
\item Require a domain and an operation, implementation not relevant at this point
\item When implementation is considerd, an ADT becomes a data structure
\end{itemize}
\subsection*{Data from 3 Different Levels}
\label{sec:org3f0fbe2}
\begin{itemize}
\item Application (user) level - modeling real life data in a specific context (ex. Library of Congress)
\item Logical (ADT) level - considering abstract understanding of necessary requirements (ex. Domain: Collection of Books, Operations: Check-in, Check-out, etc.)
\item Implementation level - considering how to carry out operations upon the domain
\end{itemize}
\subsection*{Basic Types of ADT Operations}
\label{sec:org1357ebf}
\begin{itemize}
\item Constructor - creates a new instance of an ADT
\item Transformer - changes the state of one or more of the data values of an instance
\item Observer - allows us to observe the state of 1+ data value without changing them
\item Iterator - allows us to process all the components in a data structure sequentially
\end{itemize}
\subsection*{Composite Data Type}
\label{sec:org3264127}
\begin{itemize}
\item Composite data types are types which
\begin{itemize}
\item Store a collection of individual data components under one variable name
\item Allow the individual data components to be accessed
\end{itemize}
\item Examples include arrays and classes
\end{itemize}
\subsection*{Accessing Functions}
\label{sec:org987d042}
\begin{itemize}
\item Accessing fucntions give the position of className[Index]
\item Address(Index) = BaseAddress + Index * SizeOfElement
\item Consider a base address of 6000 with a constant element size of 1 byte. Find the address of the 10th cell of this array.
\begin{itemize}
\item 6000 + (10 * 1) = 6010;
\end{itemize}
\end{itemize}
\subsection*{Order of Magnitude of a Function}
\label{sec:org256c7e9}
\begin{itemize}
\item Order of magnitude (Big-O notation) expresses computing time of a problem as the term in a function that increases the most rapodly relative to the size of the problem
\item Consider two algorithms, A and B. They are both used to solve the same class of problems.
\begin{itemize}
\item A has time complexity 5,000n
\item B has time complexity 1.1\textsuperscript{n}
\end{itemize}
\item Here, A is more efficient because it is linear, rather than exponential - which is preferable for large n
\item Order of growth and time complexity are inverses (larger growth rate = slower time to execute)
\item All functions are monotonic (continue increasing indefinitely)
\end{itemize}
\section*{01.25.21 (File I/O)}
\label{sec:orgb09367f}
\begin{itemize}
\item File I/O ex.
\begin{verbatim}
#include <fstream>

int main () {
  //opens file
  ifstream inClientFile("clients.dat", ios::in);

  //exits if file can't be opened
  if (!inClientFile) {
    cerr << "File could not be opened" << endl;
    exit(1);
  } //if

  //var declarations
  int account;
  string name;
  double balance;

 // displays each record in the file
 while (inClientFile >> account >> name >> balance) {
   outputLine(account,name,balance);
 } //while

}
\end{verbatim}
\end{itemize}
\section*{01.25.21 (C++ Ch. 9)}
\label{sec:org8379185}
\subsection*{Pass by Reference}
\label{sec:org0410a2f}
\begin{itemize}
\item When dealing with very large objects, don't pass by copy due to the large overhead of copying. Instead, pass by reference
\item When passing by reference, use const if you don't want to modify the data members
\end{itemize}
\subsection*{Destructors}
\label{sec:org1638f30}
\begin{itemize}
\item Name of destructor is className\textasciitilde{}
\item Called implicitly when an object is destroyed
\item Takes no parameters, returns no value
\item No return type allowed in signature, not even void
\item Only one destructor allowed per class
\item Must be public
\item Destructors are called once a variable exits its scope
\item Static variables are destroyed after local variables, with global variables destroyed last
\item Objects are also destroyed in reverse order from their construction
\end{itemize}
\subsection*{Const Objects}
\label{sec:orgf3e5358}
\begin{itemize}
\item const objects must use const methods only
\item non-const objects may use both non-const and const methods
\end{itemize}
\section*{01.21.21 (C++ Ch. 9)}
\label{sec:orgf923c7d}
\subsection*{Encapsulation}
\label{sec:org6cbbb65}
\begin{itemize}
\item Header files should not contain source code, it should only include prototypes in order to ensure proper information-hiding
\item Source code should be placed in a different cpp file, which pulls from the prototypes in the header file
\end{itemize}
\subsection*{Include Guards}
\label{sec:org3afec6c}
\begin{itemize}
\item Consider the following classes: Student, Course, and Main
\begin{itemize}
\item Student uses Course
\item Main uses Student and Course
\item The main method would then look like:
\end{itemize}
\begin{verbatim}
  #include "student.h"
  #include "course.h"
\end{verbatim}
\begin{itemize}
\item student.h compiles properly, but an error is thrown when course.h tries to be included because it has already been included through Student.
\item To fix this, use header guards, as follows:
\end{itemize}
\begin{verbatim}
  #ifndef FILENAME_H
  #define FILENAME_H
\end{verbatim}
\item Include guards ensure that a prototype is not defined twice
\item The header guard should be put in header files that are used in multiple places
\end{itemize}
\subsection*{Writing Classes}
\label{sec:org739ab81}
\begin{itemize}
\item Begin by including the necessary header file
\item All methods and constructors must be preceded by the header file name and the scope resolution operator (::)
\end{itemize}
\subsection*{Constructors \& Default Constructors}
\label{sec:orgd3ceba9}
\begin{itemize}
\item Constructors can call other methods and do data-checking
\item Constructors can be called explicit with multiple parameters when the parameters are impossible to typecast, as follows:
\end{itemize}
\begin{verbatim}
int main () {
  explicit Time t (x = 0, y = 0, z = 0);
} //main
\end{verbatim}
\section*{01.21.21 (C++ Ch. 3)}
\label{sec:orgb8cf92e}
\subsection*{Objects and Object Sizes}
\label{sec:org3cb155f}
\begin{itemize}
\item An objects size will always be the sum of its data members. The size will not be affected by any methods that are called upon it.
\item Because of this, objects can quickly become very large in size.
\end{itemize}
\subsection*{UML Diagrams}
\label{sec:orga2a197c}
\begin{itemize}
\item Classes are listed as individual boxes
\begin{itemize}
\item top box = class name
\item middle compartment = data members : data type
\item bottom compartment - methods and parameters
\begin{itemize}
\item - = private
\item + = public
\item \# = protected
\end{itemize}
\end{itemize}
\end{itemize}
\subsection*{Constructors}
\label{sec:orgf2b16fd}
\begin{itemize}
\item Explicit constructors can be used to prevent implicit typecasting, as seen below:
\end{itemize}
\begin{verbatim}
class Student {
  Student (int s) {

  } //constructor
} //Student

int main () {
  Student s {15}; //allowed, completes correctly
  Student c {'C'}; //typecasts automatically, should not occur
  //Note, () can be used in place of {} to construct objects
}
\end{verbatim}

\begin{itemize}
\item Ex. list initialization with an explicit constructor
\end{itemize}
\begin{verbatim}
explicit Account (std::string accountName) //explicit constructor
  : name{accountName} {
  //insert constructor code here
  }
\end{verbatim}
\section*{01.19.21 (C++ Ch. 3)}
\label{sec:org12b832f}
A look at class creation
\begin{verbatim}
#include <iostream>
using namespace std;

//defining the class
class GradeBook {
  //holds all public vars, functions
  public:
  //public function
  void displayMessage() {
    cout << "Welcome to your Gradebook" << endl;
  } //displayMesage
} //GradeBook

//main method
int main () {
  //creates a GradeBook object
  GradeBook myGradeBook;
  //calls above created function on object
  myGradeBook.displayMessage();
}
\end{verbatim}

\begin{itemize}
\item Class functions and vars are, by default, private. The public keyword must be used to denote any public parts of a class.

\item Move implementations to a header file for use in main methods while separating out each file.

\item When using header files, use quotation marks around them to indicate that they're a file on your machine. Use angle brackets around things to include form the C std lib.

\item The purpose of const functions is to prevent the function from modifying the values of data members or objects.
\end{itemize}

\section*{01.19.21 (C++ Ch. 2)}
\label{sec:orgb7cf8f1}
A look at some basic C++ code
\begin{verbatim}
#include <iostream> //enables program to output data

//main function begins program execution
int main () {
  //cout currently a function as a part of the std namespace
  std::cout << "Welcome to C++!\n";
  //above << is an insertion operator, overloaded from the bitwise left-shift

  return 0;
}
\end{verbatim}

A look at some higher level C++ code
\begin{verbatim}
#include <iostream>

int main () {

  int num1{0}; //list initialization
  int num2 = 0; //regular initialization
 //No difference between list & regular initializtion with primitive types.
 //List initialization should be used for UDTs.

  int sum{0}

  std::cin >> num1;
  std::cin >> num2;

  sum = num1 + num2;

  std::cout << sum << std::endl;
  //endl is helpful because it flushes the buffer
  //newline character does not
  return 0;
}
\end{verbatim}

A look at a common mistake
\begin{verbatim}
#include <iostream>

int main () {
  int x {5};

  if(x > 10); {
    std::cout << x "> 10" << std::endl;
  }
  //still prints output because of semicolon after if statement

  return 0;
}
\end{verbatim}
\end{document}
