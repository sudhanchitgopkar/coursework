% Created 2021-01-19 Tue 12:06
% Intended LaTeX compiler: pdflatex
\documentclass[11pt]{article}
\usepackage[utf8]{inputenc}
\usepackage[T1]{fontenc}
\usepackage{graphicx}
\usepackage{grffile}
\usepackage{longtable}
\usepackage{wrapfig}
\usepackage{rotating}
\usepackage[normalem]{ulem}
\usepackage{amsmath}
\usepackage{textcomp}
\usepackage{amssymb}
\usepackage{capt-of}
\usepackage{hyperref}
\author{Sudhan Chitgopkar}
\date{\today}
\title{}
\hypersetup{
 pdfauthor={Sudhan Chitgopkar},
 pdftitle={},
 pdfkeywords={},
 pdfsubject={},
 pdfcreator={Emacs 27.1 (Org mode 9.5)}, 
 pdflang={English}}
\begin{document}

\tableofcontents

\section{01.19.20 (C++ Ch. 3)}
\label{sec:org1430752}
A look at class creation
\begin{verbatim}
#include <iostream>
using namespace std;

//defining the class
class GradeBook {
  //holds all public vars, functions
  public:
  //public function
  void displayMessage() {
    cout << "Welcome to your Gradebook" << endl;
  } //displayMesage
} //GradeBook

//main method
int main () {
  //creates a GradeBook object
  GradeBook myGradeBook;
  //calls above created function on object
  myGradeBook.displayMessage();
}
\end{verbatim}

Class functions and vars are, by default, private. The public keyword must be used to denote any public parts of a class.

Move implementations to a header file for use in main methods while separating out each file.

When using header files, use quotation marks around them to indicate that they're a file on your machine. Use angle brackets around things to include form the C std lib.
\section{01.19.20 (C++ Ch. 2)}
\label{sec:orgbc569ad}
A look at some basic C++ code
\begin{verbatim}
#include <iostream> //enables program to output data

//main function begins program execution
int main () {
  //cout currently a function as a part of the std namespace
  std::cout << "Welcome to C++!\n";
  //above << is an insertion operator, overloaded from the bitwise left-shift

  return 0;
}
\end{verbatim}

A look at some higher level C++ code
\begin{verbatim}
#include <iostream>

int main () {

  int num1{0}; //list initialization
  int num2 = 0; //regular initialization
 //No difference between list & regular initializtion with primitive types.
 //List initialization should be used for UDTs.*/

  int sum{0}

  std::cin >> num1;
  std::cin >> num2;

  sum = num1 + num2;

  std::cout << sum << std::endl;
  //endl is helpful because it flushes the buffer, which the newline character does not
  return 0;
}
\end{verbatim}

A look at a common mistake
\begin{verbatim}
#include <iostream>

int main () {
  int x {5};

  if(x > 10); {
    std::cout << x "> 10" << std::endl;
  }
  //still prints output because of semicolon after if statement

  return 0;
}
\end{verbatim}
\end{document}
