% Created 2020-10-29 Thu 23:42
% Intended LaTeX compiler: pdflatex
\documentclass[11pt]{article}
\usepackage[utf8]{inputenc}
\usepackage[T1]{fontenc}
\usepackage{graphicx}
\usepackage{grffile}
\usepackage{longtable}
\usepackage{wrapfig}
\usepackage{rotating}
\usepackage[normalem]{ulem}
\usepackage{amsmath}
\usepackage{textcomp}
\usepackage{amssymb}
\usepackage{capt-of}
\usepackage{hyperref}
\author{Sudhan Chitgopkar}
\date{\today}
\title{}
\hypersetup{
 pdfauthor={Sudhan Chitgopkar},
 pdftitle={},
 pdfkeywords={},
 pdfsubject={},
 pdfcreator={Emacs 27.1 (Org mode 9.4)}, 
 pdflang={English}}
\begin{document}

\tableofcontents

\section{Chapter 13}
\label{sec:org2d4ba7c}
\subsection{Introduction}
\label{sec:org5d3e04c}
\begin{itemize}
\item A monopoly is a market:
\begin{itemize}
\item that produces a good or service with no close substitutes
\item that has one supplier protected by barriers to entry
\item no threat of competition
\item Barries to entry prevent new firms from tntering the market and can be natural,    ownership, legal
\end{itemize}
\end{itemize}
\subsection{Types of Barriers to Entry}
\label{sec:org34d83c5}
\begin{itemize}
\item Natural barriers to entry create natural monopoly. A natural monopoly is a market in which econmies of scale enable one firm to supply the entire market at the lowest possible cost.
\begin{itemize}
\item Over the relevant part of the market, you never finish exploiting economies of scale.
\item In a natural monopoly, economies of scale are so powerful that they are still being achieved even when the entire market demand is met
\item The Long Run Average Cost curve is still sloping downward when it meets the demand curve
\end{itemize}
\item Ownership barriers to entry: occurs when one firm owns a significant portuion of a key resource
\item Legal barriers create a legal monopoly
\begin{itemize}
\item Legal monpoly is a market where competition and entry are restricted by the granting of a
\begin{itemize}
\item public franchise
\item government license
\item patent/copyright
\end{itemize}
\end{itemize}
\end{itemize}
\subsection{How a Single-Price Monpoly Chooses Price and Quantity}
\label{sec:orgd5369b2}
\begin{itemize}
\item Monpoly price-setting strategies
\item Monopolies are price-setters.
\item For a monpoly firm to determine the quantity it sells, it must choose the appropriate price
\item There are two types of monopoly price-setting strategies:
\begin{itemize}
\item single-price monopoly: a form that must sell each unit of its output for the same price to all its customers
\item price duscrimination is the practice of selling different units of a good or service for different prices
\end{itemize}
\item A single price monopoliy maximizes profit by choosing quantity where marginal revenue = marginal cost
\begin{itemize}
\item but a monopolist's marginal revenue depends on quantity
\item to sell a larger quantity, a monopoly must set a lower price
\item this is because the demand for the monpoly's output is the market demand
\end{itemize}
\item Determining Marginal Revenue: TR = P * Q;
\begin{itemize}
\item For a single price monpoly, marginal revenue is less than the price at each level of output
\item MR < P
\end{itemize}
\item If demand is elastic: A fall in the price brings an increase in total revenue, MR > 0
\item If demand is inelastic: A fall in price brings a decrease in total revenue, MR < 0
\item If demand is unit elastic: A fall in price does not change total revenue, MR = 0
\begin{itemize}
\item Total revenue is maximized when MR = 0
\end{itemize}
\item A single price monopoly never produces an output at which demand is inelastic
\item Price and Output Decision
\begin{itemize}
\item The monpoly chooses the profit-maximizing quantity where MR = MC.
\item The monopoly sets the price at the highest price at which the profit-maximizing quantity will sell.
\item Find intersection of MR and MC, move up all the way to the demand curve
\item Monopoly earns a profut of (P-ATC) * Q
\end{itemize}
\item The monopoly can earn an economic profit, even in the long run, because abrriers to entry protect the form from market entry by competitor firms. A monopoly that incurs an economic lass can shut down termporarily in the short run or exit the market in the long run.
\end{itemize}
\subsection{Comparison of a Single Price Monopoly to Perfect Competition}
\label{sec:org1da1229}
\begin{itemize}
\item Perfect competition: Equilibrium occurs where the quantity demanded = quantity supplied.
\item Because price exceeds marginal social cost, MSB > MSC and deadweight loss occurs
\item Redistribution of surplus: some of the lost consumer surplus goes to the monopoly as producer surplus
\item Rent seeking: Any surplus - consumer surplus, producer surplus, or economic profit, is called economic rent.
\item Rent seeking is the pursuit of wealth by caputring economic rent
\item Rent seekers pursue their goals in two main ways
\begin{itemize}
\item Buy a monopoly
\item Create a monopoly
\end{itemize}
\item Rent seeking costs can shoft the ATC cruve upward, causing producer surplus to disappear
\item The deadweight loss would increase to the larger gray area
\end{itemize}
\subsection{Price Discrimination}
\label{sec:orgb0f2c3b}
\begin{itemize}
\item Practice of selling different units of a good/service for different prices
\begin{itemize}
\item To be able to price discriminate, a monopoly must
1: identify aand separate different buyer types
2: Sell a product that can't be resold
\end{itemize}
\item Methods of discrimination:
\begin{itemize}
\item Among groups of buyers
\item Among units of a good
\end{itemize}
\item By price discriminating, a monopoly caputes consumer surplus and converts it into producer surplus
\item More producer surplus = economic profit
\item Calculations:
\begin{itemize}
\item Economic profit = TR - TC
\item Producer surplus = TR - TVC
\item Economic Profit = Producer Surplus - TFC
\end{itemize}
\item Perfect price discrimination: occurs if a firm is able to sell each unit of out[ut for the highest price someone is willing to pay
\item Marginal revenue = price, Demand = MR
\item Consumer surplus is 0, there is no deadweight loss, producer surplus is maximized.
\item Economic profit attracts even more rent seeking, increases the amount of inefficiency.
\end{itemize}
\subsection{Monopoly Regulation}
\label{sec:org59a8524}
\begin{itemize}
\item Regulation: rules administrated by a government agency to influence prices, quantities, entry, and other aspects of economic activity.
\item Two theories about how regulation works are social interest theory and capute theory:
\begin{itemize}
\item Social Interest theory: political and regulatory process relentlessly seeks out inefficiency and reulates to eliminate deadweight loss
\item Caputre theory: regulatino serves the self interest of the producer who caputes the regulator and maximizes economic profit
\end{itemize}
\item Rate of Return Regulation
\begin{itemize}
\item Firm must justify its price by showing that its return on capital doesn't exceed a certain rate
\end{itemize}
\item Price cap regulation: imposition of a price ceiling, incentive to operate efficiently
\item Efficient regulation of a natural monopoly
\begin{itemize}
\item Marginal cost pricing rule sets price = marginal cost.
makes quantity demanded the efficient quantity
\item Average cost exceeds price, firm incurs economic loss
\item Natural monopoly might charge a fixed fee to voer its fixed costs then charge a price = marginal cost in order to get around economic losses
\end{itemize}
\item Average cost pricing rule
\begin{itemize}
\item the firm produces the quantity at which price = average cost and to set price = average cost. firm breaks even but produces less than efficient quantity.
\end{itemize}
\end{itemize}
\section{Chapter 12}
\label{sec:orge6d402e}
\subsection{Introduction}
\label{sec:org68a0bf0}
\begin{itemize}
\item Perfect competition is a market in which
\begin{itemize}
\item Many firms sell identical products to many buyers
\item There are no restrictions to entry into the industry
\item Established firms have no advantages over the new ones
\item Sellers and buyers are well informed about prices
\end{itemize}
\item Perfect Competition arises when
\begin{itemize}
\item The firms minimum efficient scale is small relative to market demand
\begin{itemize}
\item There is enough demand foe many firms to enter the market and explit all economies of scale
\end{itemize}
\item Each firm is percieved to produce a good/service that has no unique characteristics
\end{itemize}
\item Price takers
\begin{itemize}
\item In perfect competition, each firm is a price taker
\item A price taker is a firm that can't influence the price of a good or service
\item Each firm's output is a perfect substitute for that of other firms so demand for each output is perfectly elastic
\end{itemize}
\end{itemize}
\subsection{Cost, Revenue, and Profit}
\label{sec:org70bae7d}
\subsubsection{Economic Profit and Cost}
\label{sec:org87fc40f}
\begin{itemize}
\item The goal of each firm is to maximize economic profit, which equals total revenue - total cost
\item total cost is the opportunity cost of production which is the value of the best alternative use of resources that a firm uses in production
\item The opportunity cost includes normal profit, which is the profit an entreprenier can expect to recieve on average
\end{itemize}
\subsubsection{Economic Profit and Revenue}
\label{sec:org93a39ad}
\begin{itemize}
\item A firm's total revenue = price, P, multiplied by quantity sold, Q; revenue = P * Q
\item A firm's marginal revenue is the change in total revenue that results from a one-unit increase in quantity sold
\begin{itemize}
\item can be complicated because price and quantity may be interdependent
\item does not exist when everyone is a price taker
\end{itemize}
\end{itemize}
\section{Chapter 11}
\label{sec:orgeb7e27d}
\subsection{Introduction}
\label{sec:org5cd2ea4}
\begin{itemize}
\item Decision timeframes
\begin{itemize}
\item The firm makes many decisions to achieve profit maximization
\item 2 time frames: short run and long run
\end{itemize}
\item The short run: a timeframe in which the quantity of one or more resources used in production is fixed. For most firms, the capital, called the firm's plant, is fixed in the short run. Other reousrces used by the firm (labor, raw materials, energy) can be changed in the short run
\item The long run: a timeframe in which the quantities of all resources, including plant size, can be varied
\item A sunk cost is a cost incurred by the firm that can't be changed
\item Sunk costs are irrelevant to a firm's current decisions because the sunk cost can't be changed
\end{itemize}
\subsection{Short Run Tech Constraint}
\label{sec:org3d815af}
\begin{itemize}
\item To increase output in hte short run, a firm must increase amount of labor employed
\item Marginal Product of Labor: change in the total product that results from a one-unit increase in the quantity of labor emplyed w all other inputers remaining the same
\item Average product of labor = total product / quantity of labor employed
\item Total product = total output
\item As quantity of labor employed increases,
\begin{itemize}
\item Total product increases
\item Marginal product increases initially but then decreases
\item Average product eventually decreases
\item Increasing marginal returns initially
\item Diminishing marginal returns eventually
\end{itemize}
\item Increasing Marginal Returns
\begin{itemize}
\item increased marginal returns arise from increased specialization and division of labor
\item Diminishing marginal returns arises because each additional worker has less access to capital and less space in which to work
\end{itemize}
\item Law of Diminishing Returns: A firm uses more of a variable input within a given quantity of fixed inputs,the marginal product of variable input eventually diminishes
\item Average Product Curve
\begin{itemize}
\item When marginal product exceeds average product, average product increases
\item When marginal product is below average product, average product decreases
\item Average product will fall eventually because we know marginal product will fall over time as a result of the law of diminishing returns
\item When marginal product = average product, average product is at its maximum
\end{itemize}
\end{itemize}
\subsection{Costs in the Short Run}
\label{sec:org7f42b32}
\begin{itemize}
\item Three cost concepts and types of cost curves:
\begin{itemize}
\item Total Cost
\item Marginal Cost
\item Average cost
\end{itemize}
\end{itemize}
\begin{itemize}
\item Total Cost (TC)
\begin{itemize}
\item Cost of all resources used
\item Total Fixed Cost = cost of the firms fixed inputs, fixed costs don't change with output
\item Total Variable Cost = total cost of firm's variable inputs, var costs change with output
\item TC = Total fixed cost + total variable cost
\end{itemize}
\item Average Cost
\begin{itemize}
\item Average fixed cost = total fixed cost/unit of output
\item Average variable cost = total variable cost/unit of output
\item Average total cost = total cost/unit of output
\item Average cost = Average fixed cost + Average variable cost
\end{itemize}
\item Marginal Cost (MC)
\begin{itemize}
\item Marginal cost is the increase in total cost that results from a one unit increase in total product
\item Over the output range with increasing marignal returns, marginal cost falls as output increases
\item Over the output range with diminishing marginal returns, marginal cost rises as output increases
\end{itemize}
\item The average fixed cost shows that average fixed cost falls as output increases
\item The average variable cost is U shaped
\begin{itemize}
\item As output increases, average variable cost falls to a minimum then increases
\end{itemize}
\item Average Total Cost curve is also U-shaped because it is the sum of AVC and AFC
\item Average total cost falls at first because of
\begin{itemize}
\item Decreased fixed cost with more output
\item Specialization and division of labor
\end{itemize}
\item ATF will keep falling for some time after AVC starts rising because of AFC continuing to fall. Eventually, AVC gets larger and AFC falls only a little, bringing ATC higher
\item Marginal Cost curve passes through the minimums of AVC and ATC curves
\item For outputs over which AVC is falling, MC is below AVC
\item For outputs over which AVC is rising, MC is above AVC
\item For the output at minimum AVC, MC = AVC
\item For outputs over which ATC is rising, MC is above ATC; For outputs over which ATC is decreasing, MC is below ATC
\item For output at minimum ATC, MC = ATC
\end{itemize}
\subsubsection{Summary}
\label{sec:orgef1e93e}
\begin{itemize}
\item ATC curve is the vertical sum of AFC and AVC
\item The U shape of ATC comes because of
\begin{itemize}
\item Spreading total fixed cost over a larger output (AFC slopes downward)
\item Eventually diminishing returns from the U shaped AVC curve
\end{itemize}
\end{itemize}
\subsection{Shifts in the Cost Curves}
\label{sec:orga3c1dfa}
\begin{itemize}
\item Two factors can shoft a firm's cost curves
\begin{itemize}
\item Technology: relationship between inputs and outputs
\begin{itemize}
\item Technology can change both product and cost curves
\item Increase in productivity shifts product curve upward, cost curve downward
\item Tech advance usually results in using more capital and less labor, fixed costs     increase and var costs decrease
\end{itemize}
\item Prices of Factos of Production
\begin{itemize}
\item An increase in the price of a factor of production increases costs and shifts cost curves
\item An increase in fixed cost shifts TC and ATC curves upward
\item An increase in variable cost shifts the TC, ATC, AVC, and MC
\end{itemize}
\end{itemize}
\end{itemize}
\subsection{Costs in the Long Run}
\label{sec:org8c3de3d}
\begin{itemize}
\item In the long run, all inputs are variable and all costs are variable
\end{itemize}
\subsubsection{The Production Function}
\label{sec:org78ae09a}
\begin{itemize}
\item The Behavior of long-run cost depends on the firm's production function
\item The production function is the relationship between maximum output attainable and the quantities of both capital and labor
\item As the size of the plant increases, the output that a given quantity of labor can produce increases
\item For each plant, as the quantity of labor increases, diminishing returns occur
\end{itemize}
\subsubsection{Diminishing Marginal Product of Capital}
\label{sec:org93b21d6}
\begin{itemize}
\item The marginal product of capital is the increase in output resulting from a one unit increase in the amount of capital employed, holding the amount of labor employed constant
\item A firm's production exhibits
\begin{itemize}
\item Diminishing marginal returns to labor for a given plant
\item Diminishing marginal returns to caputal for a given amount of labor
\end{itemize}
\item For each plant, diminishing marginal product of labor creates a set of short run, U-shaped curves for MC, AVC, and ATC
\end{itemize}
\subsubsection{Short Run Cost and Long Run Cost}
\label{sec:orgda63e3d}
\begin{itemize}
\item The average cost of producing a given output varies and depends on the firm's plant
\item The larger the plant, the greater the output at which ATC is a minimum
\item The long run average cost curve is the relationship between lowest attainable average total cost and output when both plant and labor are varies
\item It is made up from the lowest ATC for each output level
\end{itemize}
\subsubsection{Economies and Diseconomies of Scale}
\label{sec:org6ad9eb6}
\begin{itemize}
\item Economies of scale are features of a firm's technology that leads to falling long run average cost (LRAC) (High initial costs mean that serving few customers/having low output makes it more costly to have a smaller firm than a bigger one)
\item Diseconomies of scale are features of a firm's technology that lead to rising LRAC as output increases (Management costs make it so difficult to manage a firm that a big firm has higher costs than a smaller one)
\item Constant returns to scale are features of a firm's technology that lead to constant LRAC as output increases
\item Economies of scale is when LRAC is falling, Diseconomies of scale is when LRAC is rising, Constant returns means LRAC is constant
\end{itemize}
\subsubsection{Minimum Efficient Scale}
\label{sec:org5b475e1}
\begin{itemize}
\item A firm experiences economies of scale up to some output level
\item Beyond that output level, it moves into constant returns to scale or diseconomies of scale
\item Minimum efficient scale = smallest output quantity at which LRAC is at its lowest level
\end{itemize}
\section{Chapter 6}
\label{sec:orgbb0b8d3}
\subsection{Price Ceiling}
\label{sec:org1201df0}
\begin{itemize}
\item Price ceiling or price cap is a regulation that makes it illegal to charge higher than a 
specified level
\item Price ceilings applied to a housing market is called a rent ceiling
\item If the rent ceiling is above equilibrium rent, it has no effect. 
A rent ceiling set below the equilibrium creates
\begin{itemize}
\item A housing shortage
\item Increased search activity
\item Black Market
\item Occurs because the legal price cannot eliminate the shortage and other mechanisms take over
\end{itemize}
\item Increased search activity: the time spent looking for someone with whom to do business activity
\begin{itemize}
\item Opp. cost of housing = rent (regulated) + opp cost of search activity (unregulated)
\item The opportunity cost of housing can exceed unregulated rent (cost is higher than equilibrium)
\end{itemize}
\item A Black Market: An illegal market that operates alongside a legal market in which a price ceiling
or other restriction has been imposed.
\item Rent Ceiling Inefficiency
\begin{itemize}
\item A rent ceiling below equilibrium leads to inefficient underproduction
\item Rent ceiling decreases quantity suppled to less than efficient quantity
\item Marginal social benefit exceeds Marginal cost and deadweight loss occurs
\end{itemize}
\item Are Rent Ceilings Fair
\begin{itemize}
\item According to fair rules, rent ceilings are unfair because they block voluntary exchange
\item According to fair results, a rent ceiling is unfair because it doesn't usually benefit the poor
\item Allocation methods:
\begin{itemize}
\item Lower willingness to pay search costs
\item Lottery, doesn't help the poor more than others
\item First come, first served
\item Discrimination
\end{itemize}
\end{itemize}
\end{itemize}
\subsection{Price Floor}
\label{sec:orgf140123}
\begin{itemize}
\item A price floor is a regulation that makes it illegal to trade at a price lower than a specific lvl
\item Price floor applied to labor market = minimum wage
\item Price floors below the equilibrium have no effect
\item If minimum wage is above equilibrium wage, quantity of labor supplied exceeds quantity demanded
by employers, creating a suprlus of labor
\item Because the legal wage rate can't eliminate surplus, this causes unemployment
\item Inefficency of a Minimum Wage
\begin{itemize}
\item Supply of labor measures the social cost of labor to workers
\item The demand for labor measures its marginal social benefit
\item A minimum wage above equilivium wage decreases the quantity of labor emplyed
\item Deadweight loss arises with potential loss from increased job search costs
\end{itemize}
\item Ultimately, both this price floor and price ceilings lead to underproduction
\item Is Minimum Wage Fair?
\begin{itemize}
\item Currently 7.25, same since 2009
\item Many economists believe that min wage rates increase unemployment of young, low-skilled workers
\end{itemize}
\end{itemize}
\subsection{Taxes}
\label{sec:org56c9dae}
\subsubsection{Tax Incidence}
\label{sec:orgc28b6d6}
\begin{itemize}
\item Tax incidence is the division of the burden of a tax between buyers and sellers
\item When an item is taxes, the price might rise by the full amount of the tax, by a lesser amount,
or not at all
\item If market price rises by the full amount of the tax, the buyer pays the tax
\item If the market rises by a lesser amount than the tax, the buyer and seller share the tax burden
\item If the market price doesn't change, sellers pay the tax
\end{itemize}
\subsubsection{Equivalence of a Tax on Buyers and Sellers}
\label{sec:orgf837b16}
\begin{itemize}
\item The effect of a tax is the same, regardless of which side of the market the tax is imposed upon
\item Demand decreases (moves down), Supply decreases (moves up), overall always decreasing quantity
\item Price paid by buyers is always higher than price recieved by sellers
\item Price paid by buyers is always on the original demand curve, price paid by sellers is 
always on the original supply curve
\item With no tax, marginal social benefit = marginal social cost, maximizing surplus
\item Taxes decrease quantity, raising buyer's price and lowering seller's cost
\item Tax revenue takes part of the total surplus
\end{itemize}
\subsubsection{Tax Incidence and Elasticity}
\label{sec:orgcd72ed2}
\begin{itemize}
\item The more inelastic the demand, the larger the buyers' share of the tax
\begin{itemize}
\item Perfectly inelastic: buyer pays full tax
\item Perfectly elastic: seller pays full tax
\end{itemize}
\item The more inelastic the supply, the larger the sellers' share of the tax
\begin{itemize}
\item Perfectly inelastic supply: seller pays the full tax
\item Perfectly elastic: buyer pays the full tax
\end{itemize}
\end{itemize}
\subsubsection{Taxes in Practice}
\label{sec:org9ba1de5}
\begin{itemize}
\item Taxes are usually levied on goods and services w inelastic demand or inelastic supply
\item Alcohol, tobacco, and gasoline have inelastic demand, so buyers pay most of the tax
\item Labor has inelastic supply, so sellers usually pay most of the tax
\end{itemize}
\subsubsection{Taxes and Fairness}
\label{sec:org0d8eebb}
\begin{itemize}
\item Benefits Principle: People should pay taxes equal to the benefits they recieve from the govt
\item Ability-to-Pay Principle: People should pay taxes based on how easily they can bear the tax
\end{itemize}
\subsection{Quotas and Subsidies}
\label{sec:org8e4e045}
\begin{itemize}
\item Quota: an upper limit to the quantity of a good that may be produced during a specified period
\item Subsidy: a payment made by the government to a producer
\item Quotas help protect producers to create a profit when the market isn't doing well
\item Quotas make production inefficient and producers have an incentive to cheat
\end{itemize}
\subsection{Markets for Illegal Goods}
\label{sec:org67ddfd4}
\subsubsection{Penalties}
\label{sec:org733b8ed}
\begin{itemize}
\item Penalties on sellers has the same effect of a tax on the seller
\item Supply of the good decreases to penalty * cost of being caught + marginal cost 
\begin{itemize}
\item Supply + Cost of Breaking the Law
\end{itemize}
\item Penalty on buyers = Demand - cost of breaking the law
\item Opportunity cost increases
\item Penalties on both buyers and sellers is the intersection of S+CBL and D-CBL
\item The new market price is P(c), buyer pays P(b) and seller gets P(s)
\end{itemize}
\subsubsection{Legalizing and Taxing Drugs}
\label{sec:org72a7990}
\begin{itemize}
\item An illegal good can be legalized and taxed
\item A high enough tax rate decreases consumption to the level that occurs when trade is illegal
\end{itemize}
\section{Chapter 5}
\label{sec:org2251df8}
\subsection{Introduction}
\label{sec:org95c7ff2}
\begin{itemize}
\item Efficiency: Are we getting the most that we can out of our scarce resources?
\item Equity: Is what we're getting out of our resources fairly dstributed?
\end{itemize}
\subsection{Resource Allocation Methods}
\label{sec:orgd4497de}
\begin{itemize}
\item Scarce resources might be allocated by
\begin{itemize}
\item Market price
\item Command (government, organizations and their hierarchical structures, rations, etc.)
\item Majority rule
\item Contest
\item First come, first served
\item Lottery
\item Force
\end{itemize}
\end{itemize}
\subsection{Demand and Consumer Surplus}
\label{sec:org1489397}
\begin{itemize}
\item Demand, Willingness to Pay, and Value
\begin{itemize}
\item Value is what we get, price is what we pay
\item The value of one more unit of a good or service is its marginal benefit
\item The maxumum price that a person is willing to pay reveals marginal benefit
\item The demand curve is a marginal benefit curve
\end{itemize}
\item Individual Demand and Market Demand
\begin{itemize}
\item The relationship between the price of a good and the quantity demanded
\begin{itemize}
\item by one person: individual demand
\item by all buyers in the market: market demand
\end{itemize}
\item The market demand curve is the horizontal sum of individual demand curves
\end{itemize}
\item Consumer Surplus
\begin{itemize}
\item the excess of the benefit recieved from a good over the amount paid for it
\item Calculate as the marginal benefit of a good - price, summed over quantity bought
\item Market consumer surplus is the sum of individual consumer surplus
\end{itemize}
\end{itemize}
\subsection{Supply and Producer Surplus}
\label{sec:org5e788a0}
\begin{itemize}
\item Supply and Marginal Cost
\begin{itemize}
\item To make a profit, firms must sell their output for a price > cost of production
\item Cost is what the producer gives up, price is what the producer recieves
\end{itemize}
\item Supply, Marginal Cost, and Minimum Supply-Price
\begin{itemize}
\item The cost of one more unit of a good or service is the marginal cost
\item The minimum price that a firm is willing to accept is its marginal cost
\item A supply curve is a marginal cost curve
\item The market supply curve is the horizontal sum of the individual supply
\end{itemize}
curves and is formed by adding the quantities supplied by all the producers at each price.
\item Producer surplus
\begin{itemize}
\item The excess of the amount recieved from a sale over the cost of production
\item Calculate as price - marginal cost, summed over quantity
\end{itemize}
\end{itemize}
\subsection{Is the Market Efficient?}
\label{sec:org8bb5566}
\begin{itemize}
\item Efficiency of Competitive Equilibrium
\begin{itemize}
\item Resources are allocated efficienty when marginal social benefit = marginal social cost
\item If nobody other than producers and consumers are effected, the competitive equilibrium
can allocate resources efficiently
\end{itemize}
\end{itemize}
\subsection{Underproduction and Overproduction}
\label{sec:org467564e}
\begin{itemize}
\item Market failure occurs upon an inefficient outcome (overproduction or underproduction)
\item Deadweight loss is the quantification of inefficiency by calculating the area of the 
full triangle before or after the equilibrium on a marginal social benefit \& cost curve
\end{itemize}
\subsection{Market Failure}
\label{sec:org3b183bf}
\begin{itemize}
\item Sources of Market Failure:
\begin{itemize}
\item Price and quantity regulations -> blocks price \& production, leads to underproduction
\item Taxes and subsidies -> taxes lead to underproduction, subsidies lead to overproduction
\item Externalities -> a cost/benefot affecting someone other than seller/buyer, leads to either
underproduction or overproduction
\item Public Goods and Common Resources
\begin{itemize}
\item Public goods: benefit everyone, nobody can be excluded. Nobody wants to pay for a public
good, leading to underproduction.
\item Common resouce: owned by nobody, but can be used by everyone. Leads to tragedy of the commons
and overproduction
\item Monopoly -> self-interest to produce profits results in underproduction
\item High Transaction costs -> leads to underproduction
\end{itemize}
\end{itemize}
\end{itemize}
\subsection{Fairness}
\label{sec:org788b3f6}
\begin{itemize}
\item Ideas of fairness can be divided into two rules
\begin{itemize}
\item Not fair if the result isn't fair
\begin{itemize}
\item Utilitarianism: greatest happiness for greatest number
\end{itemize}
\item Not far if the rules aren't fair
\end{itemize}
\end{itemize}
\subsubsection{It's not Fair if the Results aren't Fair}
\label{sec:orgcc0dd71}
\begin{itemize}
\item If everyone gets the same marginal utility from a given amount of income, and 
if the marginal benefit of income decreases as income increases, then taking a dollar from a 
richer person and giving it to a poorer person increases total benefit
\item Only when income is equally distributed has the greatest happiness been achieved
\item Utlitarianism ignores the cost of making income transfers
\item Recognizing these costs leads to the big tradeoff between efficiency and fairness
\end{itemize}
\subsubsection{It's not Fair if Rules aren't Fair}
\label{sec:org6bac4b4}
\begin{itemize}
\item Symmetry principle: the requirement that people in similar situation be treated similarly
\item Nozick suggests that fairness is based on two rules
\begin{itemize}
\item The state must create and enforce laws that establish/protect private property
\item Private property may be transferred form one person to another only by voluntary exchange
\end{itemize}
\end{itemize}
\section{Chapter 4}
\label{sec:org3068a69}
\subsection{Introduction to Elasticity}
\label{sec:org6690a26}
\begin{itemize}
\item closeness of substitutes is critical to understanding elasticity of supply and demand
\end{itemize}
\subsection{Elasticity of Demand}
\label{sec:orge484f5f}
\subsubsection{Calculting Elasticity of Demand}
\label{sec:orgae6f2a6}
\begin{itemize}
\item Price elasticity of demand is a unit free measure of the responsiveness of quantity 
demanded to a change in price when all other influences stay the same
\item percentage change in quantity demanded/percentage change in price
\item percent change in price is calculated as change in price/average of two goods/services
\end{itemize}
\subsubsection{Inelastic and Elastic Demand}
\label{sec:org1e6cc8e}
\begin{itemize}
\item Demand can be inelastic, unit elastic, or elastic
\item Elasticity can range from 0 to infinity
\item If quantity demanded doesn't change when the price changes, price elasticity = 0 and the good
has perfectly inelastic demand (Vertical demand curve)
\item If price elasticity equals exactly one, the good has unit elastic demand
\item If price elasticity of demand is less than 1 then the good has inelastic demand
\item If price elasticity is greater than 1, then the good has an elastic demand
\item If the price elasticity is infinity, the good has a perfectly 
elastic demand (Horizontal demand curve)
\end{itemize}
\subsection{Factors Influencing Elasticity of Demand}
\label{sec:orgdd87a22}
\subsubsection{Closeness of substitutes}
\label{sec:org1d3566a}
\begin{itemize}
\item the closer the substitutes, the more elastic the demand for a good or service
\item necessities, such as food or housing, generally have an inelastic demands
\item luxuries, such as exotic vacations, generally have elastic demand
\end{itemize}
\subsubsection{Proportion of Income Spent on Good}
\label{sec:org403aad6}
\begin{itemize}
\item The greater the portion of income consumers spend on a good, the larger the elasticity of demand
\end{itemize}
\subsubsection{Time Elapsed Since Price Change}
\label{sec:orgd740296}
\begin{itemize}
\item The more time consumers have to adjust to a price change or the longer the good can be stored
without losing its value, the more elastic the demand for the good
\end{itemize}
\subsection{Elasticity on a Linear Demand Curve \& Total Revenue Test}
\label{sec:orgfd5a895}
\begin{itemize}
\item At the midpoint of a linear demand curve, demand is unit elastic
\item At prices above the midpoint, demand is elastic
\item At prices below the midpoint, demand is inelastic
\end{itemize}
\subsubsection{Total Revenue and Elasticity}
\label{sec:orgc858a79}
\begin{itemize}
\item Total revenue from the sale of a good or service = price of good * quantity sold
\item Raising the price doesn't always increase total revenue
\item If demand is elastic, a 1\% price cut increases quantity sold by >1\%, total revenue decreases
\item If demand is inelastic, a 1\% price cut increases the quantity <1\%, total revenue decreases
\item If demand is unit elastic a 1\% price cut increases the quantity sold by 1\%, total revenue same
\end{itemize}
\subsubsection{Total Revenue Test}
\label{sec:org57e9c81}
\begin{itemize}
\item a method of estimating the price elasticity of demand by
observing the change in total revenue that results from a price change
\item If a price cut increases total revenue, demand is elastic
\item If price cut decreases total revenue demand is inelastic
\item If a price cut doesn't change total revenue, demand is unit elastic
\item On a bell curve, increase shows elastic, decrease shows inelastic, and peak is unit elastic
\end{itemize}
\subsection{Income Elasticity and Cross Elasticity of Demand}
\label{sec:orge2e7fac}
\subsubsection{Income Elasticity}
\label{sec:org89906e4}
\begin{itemize}
\item Income elasticity of demand measures how the quantity demanded responds to a change in income
\begin{itemize}
\item \% change in quantity demanded/ \% change in income
\end{itemize}
\item If income elasticity is >1, demand is income elastic and the good is a normal good
\item If the income elasticity is 0<x<1, demand is income inelastic and the good is normal elastic
\item If income elasticity is <0, the good is an inferior good
\end{itemize}
\subsubsection{Cross Elasticity of Demand}
\label{sec:org32898af}
\begin{itemize}
\item Measure of the responsiveness of demand to change in the price of a substitute/complement 
\begin{itemize}
\item \% change in quantity demanded/ \% change in price of substitute/complement
\end{itemize}
\item Cross elasticity of demand is:
\begin{itemize}
\item positive for a substitute
\item negative for a complement
\end{itemize}
\end{itemize}
\subsection{Elasticity of Supply}
\label{sec:orgb661af1}
\begin{itemize}
\item Elasticity of supply: measures the responsiveness of quantity suppled to a change in price
\begin{itemize}
\item \% change in quantity supplied / \% change in price
\end{itemize}
\item Supply is perfectly inelastic when supply curve is vertical and elasticity = 0
\item Supply is unit elastic if the supply curve is linear and passes through the origin
\item Supply is perfectly elastic when the supply curve is elastic and the elasticity = infinity
\end{itemize}
\subsubsection{Factors Influencing Elasticity of Supply}
\label{sec:org9c67332}
\begin{itemize}
\item Depends on
\begin{itemize}
\item Resource substitution possibilities
\begin{itemize}
\item The easier it is to substitute among resources used, the greater the elasticity of supply
\end{itemize}
\item Time frame for supply decision
\begin{itemize}
\item Momentary supply - perfectly inelastic for physical goods
\item Short-run supply is somewhat elastoc
\item Long-run supply is the most elastic
\end{itemize}
\end{itemize}
\end{itemize}
\section{Chapter 3}
\label{sec:org66a34f3}
\subsection{Introduction}
\label{sec:orgbde0aeb}
\begin{itemize}
\item Markets are any arrangements that enable buyers and sellers to get information
and do business with each other
\item Competitive Market: many buyers and many sellers so no single buyer or seller can
influence prices
\end{itemize}
\subsection{Demand}
\label{sec:org0c1dfa2}
\begin{itemize}
\item Reflects the buyers' side of the market
\item If you demand something, you
\begin{itemize}
\item want it
\item can afford it
\item have a definite plan to buy it
\end{itemize}
\item Quantity demanded: amount that consumers plan to buy 
during a particular time @ a particular price
\item Law of Demand: other things remaining the same, the higher the price of a good, the smaller
the quantity demanded (and vice versa)
\item Substitution Effect: when the relative price of a good rises, people seek substitutes so
the quantity demanded decreases
\item When the price of a good rises relative to income, people cannot afford all the things
they previously bought so quantity demanded decreases
\item Demand Curve and Demand Schedule
\begin{itemize}
\item the term demand refers to the entire relationship between good and quantity demanded
\end{itemize}
\item Demand Curve: exhibits relationshit between quantity demanded and price when all other
consumers' planned purchases remain constant
\item Willingess and Ability to Pay
\begin{itemize}
\item The smaller the quantity available, the higher the price someone is willing to pay for
another unit
\item Willingness to pay measures marginal benefit
\end{itemize}
\item Changes in Demand: when some influence on buying plans other than price changes, there is a
shift in demand for that good
\item 6 factors influencing demand:
\begin{itemize}
\item Price of related goods
\begin{itemize}
\item substitutes - good that can be used in place of another
\item complement - good that is used in conjunction with another
\item If \$ substitute inc or \$ complement dec, demand of good inc
\item if \$ substitute dec or \$ complement inc, demand of good dec
\end{itemize}
\item Expected future prices
\begin{itemize}
\item if expected future price inc, current demand inc
\item if expected future price dec, current demand dec
\end{itemize}
\item Income
\begin{itemize}
\item normal good: a good for which demand inc as income inc
\item inferior good: a good for which demand dec as income inc
\item if expected future income increases/credit is easier to get, current demand inc
\end{itemize}
\item Population
\begin{itemize}
\item The higher the population, the higher the demand
\end{itemize}
\item Preferences
\begin{itemize}
\item People with the same income have different demands if they have different preferences
\end{itemize}
\end{itemize}
\end{itemize}
\subsection{Supply}
\label{sec:orgbf8cedf}
\begin{itemize}
\item If a firm is a supplier, they
\begin{itemize}
\item have the resources and tech to produce it
\item can profit from producing it
\item has a definite plan to produce and sell it
\end{itemize}
\item Quantity supplied: the amount producers plan to sell during a given time at a particular price
\item Law of Supply: Other things remaning the same, the higher the price of a good, the greater the
quantity supplied (and vice versa).
\item Supply Curve and Supply Schedule
\begin{itemize}
\item Minimum supply price: As quantity produced inc, marginal cost inc.
\item The lowest price at which someone is willing to sell an additional unit rises
\item This lowest price is called the marginal cost
\end{itemize}
\item Changes in Supply
\begin{itemize}
\item Increases in supply shifts the curve to the right (and vice versa)
\end{itemize}
\item Factors that affect Supply
\begin{itemize}
\item Prices of factors of production
\begin{itemize}
\item If the price of an input inc, supply dec; curve shifts left
\end{itemize}
\item Prices of related goods produced
\begin{itemize}
\item denoted by substitute for production, not just substitute
\item supply of a good inc if price of a substitute dec
\item complements in production: goods that must be produced together (beef \& leather)
\item supply of a good inc if the price of a complement in production inc
\end{itemize}
\item Expected Future Prices
\begin{itemize}
\item If expected future price inc, current supply dec
\end{itemize}
\item Number of Suppliers
\begin{itemize}
\item as number of suppliers inc, supply inc
\end{itemize}
\item Technology
\begin{itemize}
\item Advances in technology lower the cost of making existing products
\item inc in technology means inc in supply
\end{itemize}
\item State of Nature
\begin{itemize}
\item natural forces and disasters can dec supply
\end{itemize}
\end{itemize}
\end{itemize}
\subsection{Equilibrium}
\label{sec:org18b692b}
\begin{itemize}
\item Equilibrium: a situation in which opposing forces balance each other
\item Equilibrium Price: the price at which quantity demanded = quantity supplied
\item Equilibrium Quantity: quantity bought and sold at equilibrium cost
\item Price Regulation
\begin{itemize}
\item Price regulates buying and selling plans
\item Price adjusts when plans don't match
\end{itemize}
\item Price adjustments
\begin{itemize}
\item Surplus forces prices down
\item Shortage forces prices up
\end{itemize}
\item Increases in demand
\begin{itemize}
\item When demand increases without changes in supply, shortages occur
\item Price therefore increaes
\end{itemize}
\item Decrease in demand
\begin{itemize}
\item At the original price, there is a surplus
\item Price therefore falls
\end{itemize}
\item Increase in supply
\begin{itemize}
\item At the original price, there is a surplus
\item Price therefore falls
\end{itemize}
\item Decrease in supply
\begin{itemize}
\item At the original price, there is a shortage
\item Price therefore increases
\end{itemize}
\end{itemize}
\section{Chapter 2}
\label{sec:orgb6a23ac}
\subsection{Production Possibilities Frontier}
\label{sec:orge856255}
\begin{itemize}
\item PPF is the boundary between combinations of goods and services that can and can't be prodiced
\item Points outside the PPF are unattainable
\end{itemize}
\subsubsection{Production Efficiency}
\label{sec:org791733c}
\begin{itemize}
\item We can achieve production efficiency if we cannt make more of one good without making les
of another such good.
\item All points on the PPF are efficient, while all points within the PPF are inefficient
\end{itemize}
\subsection{Opportunity Cost on the PPF}
\label{sec:org361e7d0}
\begin{itemize}
\item Every choice/movement along the PPF is an opportunity cost
\item Opportunity Cost = Amnt given up/Amnt gained
\item Opportunity cost increases as we move along the PPF
\begin{itemize}
\item Because resources are not equally productive for all activities, the PPF bows outwards
\item The outward bow of the PPF means that as the quantity of each good increases, so does 
the opportunity cost
\end{itemize}
\end{itemize}
\subsection{Marginal Costs}
\label{sec:org9371e65}
\begin{itemize}
\item Marginal Cost: The opportunity cost of producing one more unit of that good
\item Marginal Cost curve slopes upward for the same reason that the PPF bows outward
\end{itemize}
\subsection{Marginal Benefits}
\label{sec:org1b1ed96}
\begin{itemize}
\item Preferences: A description of a person's likes and dislikes
\item Marignal benefit: the benefit recieved from consuming one more unit of that good
\item Marginal benefot is measured by the amount that a person is willing to pay for one more unit
of a particular good or service
\item Principle of Decreasing Marginal Benefit: The more we have of any good, the smaller the marginal
benefit of that good
\end{itemize}
\subsection{Allocative Efficiency}
\label{sec:org487032b}
\begin{itemize}
\item When we cannot produce more of any one good without giving up some other good that we value
more highly
\item Point at which marginal cost and marginal benefit curve meet
\end{itemize}
\subsection{Comparative \& Absolute advantage}
\label{sec:orge89d7ef}
\begin{itemize}
\item Comparative advantage: When a person can perform an activity at a lower opportunity cost than
anyone else
\item Absolute advantage: When a person is more productiv than others
\end{itemize}
\subsection{Economic Growth}
\label{sec:org2b87b2f}
\begin{itemize}
\item Two key factors:
\begin{itemize}
\item Technnological Change
\item Capital accumulation (growth of capital resources)
\end{itemize}
\item Economic growth is not free, investing in tech and capital costs production today but helps
production tomorrow through smart investment
\end{itemize}
\subsection{Cricular Flow Model}
\label{sec:org69c3186}
\begin{itemize}
\item Need:
\begin{itemize}
\item Firms (take input, make output)
\item Markets
\item Property Rights
\item Money
\end{itemize}
\end{itemize}
\section{Chapter 1}
\label{sec:orgddd45d3}
\subsection{Scarcity}
\label{sec:orge878466}
\begin{itemize}
\item all economic questions arise because we want more than we can get
\item inability to satisfy all wants because of scarcity
\item scarcity = limited resources
\end{itemize}
\subsection{Definition of Economics}
\label{sec:org82f02c1}
\begin{itemize}
\item because we face scarcity, we must make choices
\item incentive = a reward that encourages an action or a penalty that discourages an action
\item economics is the social science that studies the choices that individuals, businesses, etc.
make as they cope with scarcity and the incentives that influence and reconcile those choices
\item Economics divides into two parts:
\begin{itemize}
\item Microeconomics = study of choices that individuals and businesses make \& how those choices
interact with markets and the influence of governments
\item Macroeconomics = the study of the performance of national and global economies
\end{itemize}
\end{itemize}
\subsection{6 Key Ideas}
\label{sec:org8da50cb}
\begin{itemize}
\item a choice is a tradeoff: ever choice is an exchange giving up one thing for another
\item making a rational choice: a rational choice compares costs and benefits, maximizing benefit
\item benefit = what you gain: the gain or pleasure something brings about, determined by preferences
\begin{itemize}
\item preferences = what a person likes, dislikes, and the intensity of those feelings
\end{itemize}
\item cost = what must be given up
\begin{itemize}
\item opportunity cost = highest val alternative that must be given up
\end{itemize}
\item choosing at the margin: the benefit of pursuing an incremental increase in some action
is marginal benefit of that action
\begin{itemize}
\item the opportunity cost of pursuing an incremental increase in some action is marginal cost
\item if marginal benefit > marginal cost, rational choice is to do more of that action
\end{itemize}
\item choices respond to incentives: a change in marginal cost/benefit changes our incentives \& choices
\end{itemize}
\subsection{Positive \& Normative}
\label{sec:org2b06d1a}
\begin{itemize}
\item economists distinguish between two types of statements: 
\begin{itemize}
\item positive statements: can be tested by checking the facts
\item normative statements: express an untestable opinion
\end{itemize}
\item economists as social scientists
\begin{itemize}
\item economists test economic models
\item economic model = a description of some aspect of the world w only the necessary features
\end{itemize}
\item economists as policy advisors
\end{itemize}
\subsection{Resources \& Highest Valued Use}
\label{sec:org676f19d}
\begin{itemize}
\item the scope of economics: 
\begin{itemize}
\item how do choices end up determining ``what, how, and for whom'' goods and services get produced
\end{itemize}
\item goods and services are produced using productive resources called factors of production
\begin{itemize}
\item land
\item labor
\item capital
\item entrepreneurship
\end{itemize}
\item who gets goods and services depends on income
\begin{itemize}
\item land earns rent, labor earns wages, capital earns interest, entrepreneruship earns profit
\end{itemize}
\item \textbf{\textbf{resources gravitate towards their highest value use}}
\end{itemize}
\subsection{Self Interest \& Social Interest}
\label{sec:orga7067bd}
\begin{itemize}
\item self interest = choices that are made because you think they are the best for you
\item social interest = choices that are best for society as a whole
\item social interest has two dimensions: 
\begin{itemize}
\item efficiency: resource use is efficient if it is not possible to make someone better off without
making someone else worse off (no waste to be eliminated)
\item fair shares/equity: refers to the fairness with which resource division occurs in a society
\end{itemize}
\item tension between self \& social interest: information revolution, climate change, globalization
\end{itemize}
\end{document}
