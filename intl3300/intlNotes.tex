% Created 2020-11-23 Mon 12:17
% Intended LaTeX compiler: pdflatex
\documentclass[11pt]{article}
\usepackage[utf8]{inputenc}
\usepackage[T1]{fontenc}
\usepackage{graphicx}
\usepackage{grffile}
\usepackage{longtable}
\usepackage{wrapfig}
\usepackage{rotating}
\usepackage[normalem]{ulem}
\usepackage{amsmath}
\usepackage{textcomp}
\usepackage{amssymb}
\usepackage{capt-of}
\usepackage{hyperref}
\author{Sudhan Chitgopkar}
\date{\today}
\title{}
\hypersetup{
 pdfauthor={Sudhan Chitgopkar},
 pdftitle={},
 pdfkeywords={},
 pdfsubject={},
 pdfcreator={Emacs 27.1 (Org mode 9.4)}, 
 pdflang={English}}
\begin{document}

\tableofcontents

\section{Module 14}
\label{sec:org4d771bb}
\subsection{Globalization}
\label{sec:org39cab36}
\begin{itemize}
\item The process of expanding and intensifying linkages among states, societies, and economies
\item Globalization is relatively new due to the depth of interconnection and the connection of institutions and societies
\begin{itemize}
\item Insitutiton types:
\begin{itemize}
\item Multinational Corporations
\item NGO's
\item IGO's
\end{itemize}
\end{itemize}
\item Political Globalization
\begin{itemize}
\item Will state sovereignty dimish as states become hollowed out
\item Less conflict and increast transparency, fragmentation? weakened democracies?
\end{itemize}
\item Societal Globalization
\begin{itemize}
\item Societies and cultures are not homogenous or static
\item Global multiculturalism? Increased democratization?
\item Spread of civil society?
\item Too much information?
\end{itemize}
\item Economic Globalization
\begin{itemize}
\item Rules of the Game: Bretton Woods Insitutions and the Washington Consensus
\item Road to greater world-wide economic growth and prosperity?
\item Greater dependence and inequality throughout the world
\item Increased outsourcing, worker displacement, and even tax avoidance through off-shoring
\end{itemize}
\end{itemize}
\section{Module 12}
\label{sec:orgac1f7b9}
\subsection{Political Violence}
\label{sec:org8cbf651}
\begin{itemize}
\item Political Violence: Violence ouside of state control that is politically motivates
\begin{itemize}
\item A challenge to the legitimacy and authority of state/state institutions
\item Can be domestic, international, or both
\item Can include: revolutions, civil wars, riots, strikes
\end{itemize}
\item Contention: Forms of collective political struggle which may or may not inclue violence
\item Roots of political violence
\begin{itemize}
\item Institutional - Institutions can constrain or enable activity
\begin{itemize}
\item Winner take all systmes tend to
\begin{itemize}
\item result in two party rule
\item increase marginalization of some groups
\item increase the likelihood of conflict
\end{itemize}
\end{itemize}
\item Ideational - Based in the power of ideas and preferences for political/economic organization, or religion/culture/history
\begin{itemize}
\item Ideas shape:
\begin{itemize}
\item World views
\item Problem identification
\item Formation of solutions to problems
\item Means to achieve these solutions
\end{itemize}
\end{itemize}
\item Individual
\begin{itemize}
\item Relative deprivation (usually economic or sociopolitlcal)
\item Marginalization (humiliation, lack of dignity, equality, etc)
\item \{art of a mass movement (collective) or absed on personal experiences
\end{itemize}
\end{itemize}
\item Responses to Political Violence
\begin{itemize}
\item Balancing freedom/privacy and security
\item Problematic in a democracy
\begin{itemize}
\item Danger of a surveillance state
\end{itemize}
\end{itemize}
\end{itemize}
\subsection{Social Movements}
\label{sec:org67436ec}
\begin{itemize}
\item Social Movements Definition
\begin{itemize}
\item Kriesi: An organized, sustained, self-conscious challenge to existing authorities on behalf of constitutencies whose goals are not effectively taken into account by authorities
\item Tarrow: Collective challenges to elites, authorities, etc by people with a common purpose and solidarity in sustained interactions with elites, opponents, and authorities
\item Have loose organization with spatial and temporal limitations
\begin{itemize}
\item Still true with social media?
\end{itemize}
\item Networks with weak ties
\begin{itemize}
\item Emphaized by physical separation of the internet
\end{itemize}
\item Can become interest groups/pol parties but often lack the singlular mission and collective cohesion to do social
\end{itemize}
\item A social movement comprises:
\begin{itemize}
\item Mobilitzation/prganization of members
\item Establishment of a community of believers
\item Segmented, polycentric leadership
\item Evolution through stages (may not move linearly)
\end{itemize}
\item Stages of Social Movements
\begin{itemize}
\item Spontaneous response from the concerned
\item Organizational structure emerges
\item Interest groups become institutionalized, can influence policy
\item Advocates become policy-makers
\end{itemize}
\item Social movements and the Internet
\begin{itemize}
\item Makes it easier for connection and spread
\item Weaker ties because of the isolated nature of the internet
\end{itemize}
\end{itemize}
\section{Module 11}
\label{sec:org950fb9f}
\subsection{Non-Democratic Systems}
\label{sec:org16bc1b5}
\begin{itemize}
\item Often have the same institutions as democratic systems but the institutions don't have to answer to the people
\item Sources of Non-democratic rule
\begin{itemize}
\item Many argue that without a middle class, there can be no democracy
\item Modernization can bring democratization but can also bring instability that increases chances of non-democracy
\begin{itemize}
\item + inflation, unemployment, changing employment sectors
\item social instabiltiy
\item weak institutions
\end{itemize}
\end{itemize}
\item The Resourse Trap
\begin{itemize}
\item Presence of natural resources can encourage leaders/elites to siphon wealth for themselves while keeping public in check through
\begin{itemize}
\item Violence/terror
\item Corporatism
\end{itemize}
\end{itemize}
\item Rent-seeking behavior by elites can lead to kleptocracy with attention on gaining wealth, non controlling population. Situation can backfire if resources are depleted or if market prices of resources fall
\item Populism: an appeal by certain leaders to the population for support in working against current institutions structure in the name of interests of the people
\item Civil society: societal associations organized around interests of a group of people
\item Civil society in non-democracies are
\begin{itemize}
\item co-opted by the state
\item prohibited by the state
\item regulated by the state
\end{itemize}
\item Social and Political culture
\begin{itemize}
\item religion critical in shaping political culture
\item cultures can be heterogeneous making it difficult for consesus rule, seeds of competitions for resources and conflict amongst groups
\item cultures with a dominant group in control can lead to exclusion aof the minortiy group
\end{itemize}
\item Maintenance of the State
\begin{itemize}
\item Coercian and surveillance
\item Cooptation
\begin{itemize}
\item corporatism
\item clientelims/patronage
\item neopatrimonialism
\item nepotism
\end{itemize}
\item Personality cults (often based on charismatic/traditional legitimacy)
\end{itemize}
\end{itemize}
\subsection{Types of Non-Democratic Systems}
\label{sec:org4f4f4ad}
\subsubsection{Models of Non-democratic Rule}
\label{sec:org169d269}
\begin{itemize}
\item Military
\begin{itemize}
\item Common over the past 50 years in Latin America, Africa, some of Asia
\item Usually via coup or military intervention where regimes are weak
\item Most civili liberties and political parties suspended
\end{itemize}
\item Personal/Monarchical
\begin{itemize}
\item Leaders draw on traditional and charismatic legitimacy
\item Weak/non-existent ideology
\item State and society are subjects of the leader
\item Patrimonialism: clientelism for a select few
\end{itemize}
\item One-Party
\begin{itemize}
\item Single political party monopolizes politics
\item Often uses corporatism for popular control
\item Party mobilizes citizens through propoganda and indoctrination
\end{itemize}
\item Illiberal/Hybrid
\begin{itemize}
\item Partially free - restricts individual liberties and combines democratic and non-democratic insitutions
\item Weak rule of law
\item Executives hold inordinate amount of power
\end{itemize}
\item Theocracy
\begin{itemize}
\item Rule by God
\item Fundamentalism fusion of the state and religion where the state is surbordinate to religious tenets
\end{itemize}
\end{itemize}
\section{Module 10}
\label{sec:org54be16d}
\subsection{Developing Countries}
\label{sec:org27353de}
\begin{itemize}
\item Economic Development
\begin{itemize}
\item Economic development can be indicated by GDP and GDP per capita
\item GDP/Capita doesn't account for income inequality, which is accounted for by the Gini Index
\end{itemize}
\item Sub-national Variations
\begin{itemize}
\item Components of Index:
\begin{itemize}
\item Life expectancy at birth
\item Expected years of schooling
\item GNI (Gross National Income) per capita
\end{itemize}
\item Looking at national indeces is not as specific or accurate, many times
\end{itemize}
\item First, Second, Third World
\begin{itemize}
\item Used during the cold war but no longer used by political scientists
\item West = first world
\item Second world = communist
\item Third world = everyone else
\end{itemize}
\item Middle and Lower Income Country
\begin{itemize}
\item Middle-Income Country: Historically less-developed country that experienced significant economic growth and democratization
\item Lower-Income Country: A country that lacks significant economic development or political institutionalization, or both
\item Not static definitions, imply a path along which a country is moving
\end{itemize}
\item Roots of Middle \& Lower Income Countrues
\begin{itemize}
\item Imperialism: projecting power outside the state in order to gain resources
\item Colonialism: A greater degree of physical occupation by the imperial power (not just a skeleton crew in charge of shipping good back home)
\end{itemize}
\item Historical Roots of Middle and Lower Income Countries
\begin{itemize}
\item Spain \& Portugal: Central/South America
\item Britain \& France: North America
\item Japan: Chunks of Asia
\item Various European Powers: N Afrrica -> Most of Africa -> Middle East -> Asia
\end{itemize}
\item What Imperialism Meant:
\begin{itemize}
\item Imposition of imperial power's institutions obliterating original institutions
\item Borders reflected imperial country's strategic interests, not geographic or demographic realities
\item Imperial power's state was trying to ``civilize'' or ``modernize'' territory
\item Bureaucratic structures established by imperial powers (National language, legal code, administrative capacities)
\end{itemize}
\end{itemize}
\subsection{Developing Countries (cont.)}
\label{sec:orgbb0447d}
\subsubsection{Imperialism \& Culture}
\label{sec:org9199563}
\begin{itemize}
\item Identity shift: before imperialism, people held identities based on clans, religion, econ status, gender, etc. The imperial powers constructed and imposed new national/ethnic identities for conquered territories based on ascribed racial characteristics. These identities were used for classification and admin purposes, distribution of resources to indigenous populations
\item Religion transplant
\begin{itemize}
\item Catholocism in Latin America
\item Islam in North Africa, Mediterranean
\item Protestantism in North America
\end{itemize}
\item Gender roles transplanted as well, diminished status of women and took out matriarchies
\item Legacies of imperialism: Belgium -> Hutus \& Tutsis -> Rwandan Genocide
\end{itemize}
\subsubsection{Imperialism \& Culture}
\label{sec:orgcc7dceb}
\begin{itemize}
\item Production \& Dependence
\begin{itemize}
\item Traditional socieities turned into cash socieities, stemming from th resource needs of imperial powers
\item Significant mercantilist economies
\item Massive companies acted as states in territories (Dutch East India Company
\end{itemize}
\end{itemize}
\subsubsection{Development in the Modern Era}
\label{sec:org6ad1d1f}
\begin{itemize}
\item Third wave of democratization
\item Colonies gained independence in the 1960s under harsh conditions to take full autonomy over
\end{itemize}
\section{10.09.20}
\label{sec:org4b6ee80}
Donald Trump, during his time as president of the United States has upheld American democracy by abiding by the systems of checks and balances in the branches of government and actively contributing to passing policy that he promised during the 2016 election cycle. Democracies are based largely upon governments following the will of the people and the argument can be made that when Donald Trump was elected in 2016, the people wanted the policies he promised.
\section{Module 8}
\label{sec:org10fce48}
\subsection{Developed Democracies}
\label{sec:orge0ea5ca}
\begin{itemize}
\item Post-modern or post-material values: values things more than just the ability to live
from day to day; Increasing quality of life through high mass consumption and self-
acutalization
\item Began and most prevalent in Europe, least prevalent in Africa w NA,SA in between
\item Economic development often supports democratic development but does not always relate
\item Human Development Index shows that just because a country is rich doesn't mean wealth
is evenly distributed
\item Gini Index can be used to show inequality, Social democracies have low gini scores
\begin{itemize}
\item US is an anomaly here
\item COVID shows that death rate is extremely low for mercantile democracy
\end{itemize}
\end{itemize}
\subsection{The EU \& Germany}
\label{sec:orge887202}
\subsubsection{History of the EU}
\label{sec:orgd66f62e}
\begin{itemize}
\item Began as the European Coal and Steel Community (1951)
\item Started with France, Germany, Italy, Benelux
\item Added European Economic Community \& European Atomic Energy Community (1957) Treaty of Rome
\item Brought the whole thing together with the Single European Act (1986)
\item Single European Act -> Maastrict Treaty (1991)
\item EU declared (1993)
\item Currently 27 member states
\end{itemize}
\subsubsection{Structure of the EU}
\label{sec:orgf1eb84e}
\begin{itemize}
\item European Council - heads of govt of each member
\item Council of the EU - Main legislative body, crafts legislation and budget w parliament
\item EU Parliament - elected by citizens of respective member states, shares duties w council
\item Court of Justice of the EU - hears cases brought by/against member states, EU citizens, companies, etc.
\item Court of Auditors - controls EU budget
\item European Central Bank - forms EU econ and monetary policy, manages Euro
\end{itemize}
\subsubsection{Germany \& The EU}
\label{sec:org8d94617}
\begin{itemize}
\item Largest pop of EU members
\item Highest GDP of EU members
\item Second highest employment rate, third lowest unemployment rate with a national min wage
\end{itemize}
\section{Module 7}
\label{sec:orga9415b0}
\subsection{Comparing the US \& UK}
\label{sec:org87ac53c}
\begin{center}
\begin{tabular}{ll}
UK & US\\
Parliamentary system & Presdential system\\
Single Member District Majoritarian System & SMD + Majoritarian system\\
No single written constitution & constitution est. 1787\\
Limited local govt. some devolved authority & Federal system w state powers\\
3 Branches of Govt. + Crown (Ceremonial) & 3 Branches of gove\\
No checks and balances, sep. of powers & Sep. of powers\\
House of Commons, House of Lords & House of Rep, Senate\\
\end{tabular}
\end{center}
\begin{itemize}
\item Both are democracies
\end{itemize}
\section{Module 6}
\label{sec:orgeb7af02}
\subsection{Democratic Institutions}
\label{sec:org3a56ed8}
\subsubsection{Legislatures}
\label{sec:org630efd7}
\begin{itemize}
\item Forum for national political Debate
\item Where laws are proposed or passed
\item Bicameral (two houses)
\begin{itemize}
\item Senate \& House in US
\begin{itemize}
\item House of Lords \& Commons in the UK
\end{itemize}
\end{itemize}
\item Unicameral
\begin{itemize}
\item Single house more likely to be found in smaller and more centralized democracies
\item Sfound in Norway, South Koera
\end{itemize}
\end{itemize}
\subsubsection{Judiciary}
\label{sec:org7517810}
\begin{itemize}
\item Central to democracy's rule of law
\item Different types of courts and organizations of courts
\item Some countries have a constitutional court solely to interpret constitutional legality,
this is shown through judicial review
\item Not all countries have a supreme court that exercises judicial review
\end{itemize}
\subsubsection{Executvies}
\label{sec:orgd373df6}
\begin{itemize}
\item Head of State
\begin{itemize}
\item Represents the government on the national/intl stage, mainly symbolic
\end{itemize}
\end{itemize}
\begin{itemize}
\item Head of Government
\begin{itemize}
\item Focuses on policy-making
\end{itemize}
\item President of the US is both head of state and government
\item Types of Executive systems
\begin{itemize}
\item Parliamentary
\item Presidential
\item Semi-Presidential
\end{itemize}
\end{itemize}
\begin{center}
\begin{tabular}{ll}
Presidential & Parliamentary\\
Limited government (Separation of powers) & High policy-making efficiency\\
Checks and Balances -> Gridlock & Fusion of divisions != Gridlock\\
Popularly-elected executive & Executive: leader of largest party\\
Fixed terms, no long term grip & Parties can hold power for long times\\
Elections are candidate-based & Single party loyalty (?)\\
\end{tabular}
\end{center}
\subsection{Electoral Systems}
\label{sec:org8d7d6ac}
\subsubsection{Single-member District (SMD)}
\label{sec:orgf38b416}
\begin{itemize}
\item Also called the ``first past the post'' or ``winner take all'' system
\item Voting for candidates directly instead of for a party
\item Most likely to result in a two-party system
\begin{itemize}
\item Called Duverge's Law
\end{itemize}
\end{itemize}
\subsubsection{Proportional Representation (PR)}
\label{sec:orgfd9c831}
\begin{itemize}
\item Made of multi-member districts (mmd) - more than 1 person elected from ea. electoral district
\item People vote for parties moreso than individuals
\item Votes are ranked for parties
\item Proportion of the vote for a particular party wins the party certain amounts of seats
\item Tends to result in multiple parties winning seats, generally leading to electoral thresholds
\end{itemize}
\subsubsection{Mixed Electoral Systems}
\label{sec:org1c6dc65}
\begin{itemize}
\item A comboination of SMD and PR systems
\item Ranked voting
\begin{itemize}
\item Alternative Vote - Australia
\item Single Transferrable Vote - Ireland
\end{itemize}
\end{itemize}
\subsubsection{Referenda and Initiatives}
\label{sec:orgc0f3531}
\begin{itemize}
\item Some countries put political decisions in the hand of the people through a referendum
\item Can be seen as a cop-out by legislators and executives back to the people
\item Referenda can also be constitutionally-mandated as in Ireland
\item Initiatives are political decisions put to the people due to a petition
\item Certain number of people need to sign a petition before a vote
\end{itemize}
\section{Module 5}
\label{sec:org2097c92}
\subsection{Political Economy}
\label{sec:org82d22c1}
\begin{itemize}
\item Political Economy: The study of how politics and economics are related
\item Components:
\begin{itemize}
\item Markets
\item Property
\item Public goods
\item Taxation
\item Fiscal Policy
\item Regulation
\item Trade
\end{itemize}
\item Public Goods \& Social ExpendituresL
\begin{itemize}
\item Public goods: Those goods provided or secured by the state and are available for everyone
\item Social expenditure: The state's provision of public benefits or welfare
\begin{itemize}
\item All states have some kind of social expenditure
\end{itemize}
\end{itemize}
\item Taxation
\begin{itemize}
\item Mostly needed to fund state activities
\item Different kinds of taxes at different levels
\item Some countries provide goods and services mostly from revenues from taxation
\end{itemize}
\item Regulations
\begin{itemize}
\item Rules or orders that set the boundaries of a given procedure
\item Costs of compliance
\item Costs of monitoring
\item Costs of non-compliance
\end{itemize}
\item Trade \& Economic Development
\begin{itemize}
\item Free Trade: Trade among countries wherein no country restricts trade from any other country
\begin{itemize}
\item by levying import tariffs/duties
\item through imposition of quotas
\item by providing subsidies to its own domestic firms
\item by introducing other non-tariff barriers
\end{itemize}
\item Trade that is free from barriers is theorized to improve economic development/innovaiton
through the use of a comparative advantage
\end{itemize}
\end{itemize}
\subsection{Varieties of Capitalism}
\label{sec:org48f99a0}
\begin{itemize}
\item Advantages of market systems
\begin{itemize}
\item very dynamic
\item high levels of productivity
\end{itemize}
\item Disadvantages of market systems
\begin{itemize}
\item Variability
\item Negative market swings can ahve a domino effect
\item Negative social externalities (inequality, unemployment, etc)
\end{itemize}
\item Political-Economic Systems
\begin{itemize}
\item Liberal Democracy
\item Social Democracy
\item Mercantile Democracy
\item Communism
\end{itemize}
\item Liberal Democracy: An ideology and political system that favors limited state role in society 
and the economy and places a high priorty on individual political and economic freedom
\item Social Democracy: A political-economic system where freedom and equality are balanced through 
state management of economy and provision of social expenditures
\begin{itemize}
\item features corporatism where government, forms, and workers have a tripartite relationship
\item often called a coordinated market economy
\end{itemize}
\item Mercantile Democracy: State controls economy
\begin{itemize}
\item State owns parts or all of industry
\item Heavy regulations, tariffs, and non-tariff barriers to foster and protect domestic industry
\item Little social expenditure, low taxes
\item Allows for rapid economic growth (Asian TIGER countries) and often export oriented
\end{itemize}
\item No single type of democracy is better than another- some simply align with certain interests
\end{itemize}
\section{Module 4}
\label{sec:orgfc7a63e}
\subsection{Nations \& Society}
\label{sec:org455347c}
\begin{itemize}
\item Goals of nation-building:
\begin{itemize}
\item Capacity
\item Legitimacy
\item Identity
\end{itemize}
\item Society: ``A collection of people bound by shared institutions that define how relations
should be conducted
\item Types of Identity:
\begin{itemize}
\item Primordial (genetic)
\item Ascribed (given by others)
\item Socially constructed (develops over time)
\end{itemize}
\item Identity is not inherently political but can be politicized
\item Citizenship: An individual or group's relation to the state
\item Different states have different citizenship regimes
\begin{itemize}
\item Allowance of dual citizenship
\item Types of naturalization process
\end{itemize}
\item Identity as an Institution
\begin{itemize}
\item Identities comprise kinds of institutions
\item Identites are sticky
\item Politicization of identities increases probability of conflict
\end{itemize}
\item Ethnic conflict: Conflict between ethnic groups that struggle to acheive goals
at each other's expense
\item National Conflict: Conflict in which one or more groups within a country 
develops clear aspirations for political independence, clashing with others as a result
\end{itemize}
\subsection{Political Culture \& Ideology}
\label{sec:org705baff}
\begin{itemize}
\item Political culture is very difficult to define and is relative
\begin{itemize}
\item can be considered an informal institution
\item may be rooted in culture or religion
\item developed from an early age
\end{itemize}
\item Political attitudes: how one sees the operations of the state and its institutions
\begin{itemize}
\item Radical, liberal, conservative, reactionary
\item Majority are around center
\item Liberal: Seek to change society through institutional adjustments
\item Constitution: Prefer continuity, resist change
\item Radicals and Reactionaries: generally outside instuitutions, may use violence
\end{itemize}
\item Attitudes are relative to political culture
\begin{itemize}
\item A liberal in the US = a conservative in France
\end{itemize}
\item Political ideologies: what one views as the fundamental goals of politics
\begin{itemize}
\item Communism -> Social Democracy -> Liberalism -> Fascism -> Anarchy
\item Here, liberalism supports political choice, not political attitudes
\item Social democracy supports greater state intervention
\item Communism, Facsism, and Anarchy are non-democratic (radical or reactionary)
\end{itemize}
\item Socialist definition
\begin{itemize}
\item Communist parties of the former societ bloc (non-democratic) described as socialist
\item Nazi (extreme right) stood for national socialist party
\item Social democrat parties of advanced democracies are democratic
\end{itemize}
\end{itemize}
\section{09.02.20}
\label{sec:org0997094}
\subsection{State Development}
\label{sec:orgd5e213b}
\begin{itemize}
\item Europe v the New World
\begin{itemize}
\item Compare the state development of European, ``old-world'' countries and ``new world'' countries``
\begin{itemize}
\item Old world countries tend to be more imperialistic while new countries have a common exp
of being colonies
\item New world countries were composed of different types of people while 
Old world countries had a shared history
\end{itemize}
\end{itemize}
\item Feudalism: Geographic proximity and increasing power of feudal lords -> challenges between 
feudal properties were likely, so organization of resources and capabilities was key to survival
\item Feudalism led to increased collectivism, translating to:
\begin{itemize}
\item large, active labor organizations
\item large, state-provided social welfare
\item emphasis on production of higher quality goods instead of new innovation
\end{itemize}
\end{itemize}
\section{Module 3}
\label{sec:org6424f1b}
\subsection{Institutions and States}
\label{sec:org9174f6f}
\subsubsection{Institutions}
\label{sec:org3a2daf4}
\begin{itemize}
\item Institution: Institutions are formal and informal rules 
that structure the relationship among individuals
\item Can have legal or social forces
\item Institutions are resistant to change but can change as a 
\begin{itemize}
\item response to outside forces
\item response to internal pressures
\item response to effects of other institutions
\end{itemize}
\end{itemize}
\subsubsection{The State}
\label{sec:orge9fcf0b}
\begin{itemize}
\item An organization that maintains a legitimate monopoly of force over a certain territory
and its population
\item A set of political institutions sets policies for the territory and its population
\item Sovereignty: The ability for a state to carry out actions/policies within a territory
independently from external actors or internal rivals/challengers
\item Issues of autonomy and capcity: 
\begin{itemize}
\item Autonomy: the ability for the state to weild its power independently of the public
\item Capacity: the ability for the state to accrue and utilize sufficient resources to carry out
basic tasks and responsibilities
\end{itemize}
\end{itemize}
\subsubsection{Definitions}
\label{sec:org56b5be9}
\begin{enumerate}
\item General
\label{sec:orge80e931}
\begin{itemize}
\item State: governing structur's legitimate expression of sovereignty/main political organization 
of a country
\item Regime: Informal institutions that guide how a state operates
\item Government: Collection of actors in charge of carrying out political decisions of the regime
and in the interest of the state
\item Country: More generic; refers to the political collectivity of a soverieng territory
\item Nation: Refers to a group of people bound together by some trait who seek to establish 
to establish and express political interests
\item Nation != Country
\end{itemize}
\item Strength of States
\label{sec:org05fd08c}
\begin{itemize}
\item Institutional Capabilities
\begin{itemize}
\item Strong States: Has good institutional foundations; these institutions function well
\item Weak States: Does not have good institutional foundations, its institutions do not function well
\item Failed States: Institutions so weak that they basically collapse and have no sovereignty
\end{itemize}
\item Organizational Structure
\begin{itemize}
\item Strong states maintain a fair amonut of centralized control
\item Weak states hand down authority to local institutions and are decentralized
\end{itemize}
\end{itemize}
\end{enumerate}
\subsection{Legitimacy \& Sovereignty}
\label{sec:orgfa6b981}
\begin{itemize}
\item Legitimacy: a value whereby something or someone is recognized and accepted by a large 
portion of the population as right and proper (is highly subjective)
\item Types of legitimacy:
\begin{itemize}
\item Traditional legitimacy: embodies historical myths/legends and continues from past to present
\item Charismatic legitimacy: Built on the force of ideas and appeals embodied by a leader
\item Rational-Legal legitimacy: Based on a system of laws and procedures that are institutionalized
\end{itemize}
\item Sources of Legitimacy:
\begin{itemize}
\item Conferred by the ruler to a ruler, government, or state
\item Ascribed to a state or ruler by other states or rulers (prerequisity for intl. cooperation)
\item Ascribed to a state or ruler by organizations/non-state actors
\end{itemize}
\item Legitimacy can often be used to push for change
\end{itemize}
\section{08.26.20}
\label{sec:org66c490e}
\subsection{Defining a Good Society}
\label{sec:orgb3eb779}
\begin{itemize}
\item Although observable, empirical assessments may differ from person to person,
depending upon factors that may distort individual observation.
\item Multiple factors contribute to whether a society is ``good'' or not, critical to comparing countries and 
political systems
\end{itemize}
\section{Module 2}
\label{sec:orgcaa9a09}
\subsection{Video 1}
\label{sec:orgf485a52}
\subsubsection{``Traditional Approach''}
\label{sec:org4d2b211}
\begin{itemize}
\item Focus on a ``formal-legal'' aspects of political institutions
\item Mostly a categorizing exercise with little analysis
\item Many European ex-pats were these scholars
\end{itemize}
\subsubsection{Modern Era (1960s-1980s)}
\label{sec:org76203f3}
\begin{itemize}
\item Scholars stop describing, start comparing
\item Behavioral Revolution - emphasis on individual, group behavior, not static institutions
\item Gave rise to ``developmentalism'' or ``modernization theory'' 
\begin{itemize}
\item Proposed that a state develops economically, political and social development follows
\item Functionalism (functions of differently societal elements lay foundation for growth)
\end{itemize}
\end{itemize}
\subsubsection{Development (1960s-1980s)}
\label{sec:org78a047a}
\begin{itemize}
\item 5 stages each society goes through for development:
\item Traditional society (no mass production)
\item Preconditions for economic take-off (advent of industrialization and mass production)
\item Take-off (dynamic economic growth)
\item Drive to maturity (long era of econ growth, modern tech usage)
\item Age of high mass consumption (everyon is within driving distance of McDonalds (most places))
\end{itemize}
\subsubsection{Critiques of Behavioralims/Developmentalism}
\label{sec:org98f02b6}
\begin{itemize}
\item Enthocentric and ideologically driven
\item Creates dependency: capitalism creates a situation where underdeveloped countries depend
on developed countries
\item Developmentalist theories tried to be a one-size-fit-all theory which wasn't bale to be applied
to all individual case studies
\end{itemize}
\subsubsection{Post-Behavioralism (1990s-Present)}
\label{sec:orgca437cf}
\begin{itemize}
\item Development of middle-range theories instead of one single theory
\item Diversity of approaches (qualitative, quantitative, case sudies)
\item Takes culture and historical context into consideration
\item Rational choice theory applied
\item Political economy: the state can have a varying role in economic matters
\end{itemize}
\subsubsection{New Institutionalism (Past 25 years)}
\label{sec:org079dae9}
\begin{itemize}
\item Institutions are the nexus of political action
\item Institutions are dynamic that interact over time w other variables
\item Institutions comprise the surrounding environment \& sentiment
\end{itemize}

\subsection{Video 2}
\label{sec:org9a2ba86}
\subsubsection{The Study of Comparative Politics}
\label{sec:org6380d94}
\begin{itemize}
\item Comparative politics implies a method of study or an approach to an analysis, not a single theory
\item greatest challenge is that events occur in real time with unreplicable environments
\item events in politics can not be replicated to test for validity
\end{itemize}
\subsubsection{Goals}
\label{sec:org8580d43}
\begin{itemize}
\item Goal: To assess which factors cause a certain outcome by comparing or contrasting cases
\item Cases: One of the group of things (events, states, actors, etc.) to be studied
\item Variable: a factor that changes over time or in different cases
\begin{itemize}
\item Independent var: causal var
\item Dependent var: outcome var
\end{itemize}
\item Causal relationships can be shown as:
\begin{itemize}
\item Cause -> effect
\item Independent var -> dependent var
\item Explanators var -> outcome
\item x var -> y var
\end{itemize}
\item Hypothesis: a possible answer that explains a causal effect
\end{itemize}
\subsubsection{Challenges}
\label{sec:orgf0a5973}
\begin{itemize}
\item Goal: to determine causality, not just correlation

\item In comparative politics, the researcher may not be able to:
\begin{itemize}
\item have a constant
\item measure certain variables
\item anticipate certain events
\item disentangle one variable from others
\item Access to cases \& information
\begin{itemize}
\item Langauage barriers
\item Time \& funding
\item Sufficient cases (and selection bias)
\item IRB (Institutional Review Board)
\end{itemize}
\end{itemize}
\item Correlation: when var A occurs with var B, one is not caused by the other
\item Endogeneity: when it cannot be determined whether an outcome was caused by another factor
or the outcome caused that factor to occur
\end{itemize}
\subsection{Video 3}
\label{sec:org8bf7b4b}
\subsubsection{Most Similar Systems Design (MSS)}
\label{sec:org99473de}
\begin{itemize}
\item A method in which as many independent vars as possible are held constant to explain a political
outcome: similar cases, different outcomes can help isolate a variable
\item Special Variation of MSS: Within-Case Comparison
\begin{itemize}
\item Single case analyzed over time or in different geographical areas
\item Breaks up a single case into subparts and allows for comparison
\end{itemize}
\end{itemize}
\subsubsection{Most-Different Systems Design (MDS)}
\label{sec:orgbbb144e}
\begin{itemize}
\item Looks at cases that are different from one another and observes why the same political outcome is
observed as a method of understanding how to isolate a single causal variable
\end{itemize}
\subsubsection{Overview}
\label{sec:orgb4d6bcd}
\begin{itemize}
\item Probable causal explanations (hypotheses): goal of these comparative approaches
\item Theories can be built from the strongest hypothesis
\item Theories can further be generalized based on the case
\end{itemize}
\end{document}
