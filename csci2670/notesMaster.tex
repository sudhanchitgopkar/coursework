% Created 2021-01-20 Wed 22:47
% Intended LaTeX compiler: pdflatex
\documentclass[11pt]{article}
\usepackage[utf8]{inputenc}
\usepackage[T1]{fontenc}
\usepackage{graphicx}
\usepackage{grffile}
\usepackage{longtable}
\usepackage{wrapfig}
\usepackage{rotating}
\usepackage[normalem]{ulem}
\usepackage{amsmath}
\usepackage{textcomp}
\usepackage{amssymb}
\usepackage{capt-of}
\usepackage{hyperref}
\author{Sudhan Chitgopkar}
\date{\today}
\title{}
\hypersetup{
 pdfauthor={Sudhan Chitgopkar},
 pdftitle={},
 pdfkeywords={},
 pdfsubject={},
 pdfcreator={Emacs 27.1 (Org mode 9.5)}, 
 pdflang={English}}
\begin{document}

\tableofcontents \clearpage\section{01.19.21}
\label{sec:org1080857}
\subsection{Tuples \& DFAs}
\label{sec:org4154304}
\begin{itemize}
\item Tuples are sequences which are always finite in length
\item The deterministic finite automaton shown is a 5-tuple:
\begin{enumerate}
\item Q: finite nonempty set of states
\begin{itemize}
\item state: configuration of logic of a machine
\end{itemize}
\item \(\Sigma\) (Sigma) - input alphabet
\begin{itemize}
\item alphabet: a finite, nonempty set of symbols where symbols are an object of length 1
\end{itemize}
\item \(\delta\) (Delta) - transition function
\item Q\textsubscript{0} \(\in\) Q - starting state
\item F \(\subset\) Q - set of final states
\end{enumerate}
\item For this deterministic finite automaton,
\begin{itemize}
\item \(\delta\): Q \texttimes{} \(\Sigma\) \(\to\) Q\textsubscript{2}
\end{itemize}
Represented as a table,
\begin{center}
\begin{tabular}{rlrl}
Step & State & Input & Transition\\
1 & Q\textsubscript{1} & 1 & Q\textsubscript{1} \(\to\) Q\textsubscript{2}\\
2 & Q\textsubscript{2} & 0 & Q\textsubscript{2} \(\to\) Q\textsubscript{1}\\
3 & Q\textsubscript{1} & 1 & Q\textsubscript{1} \(\to\) Q\textsubscript{2}\\
4 & Q\textsubscript{2} & 1 & Q\textsubscript{2} \(\to\) Q\textsubscript{2}\\
\end{tabular}
\end{center}
\end{itemize}
\subsection{Domains \& Codomains}
\label{sec:orgdfb464a}
\begin{itemize}
\item Domain: set of all possible function inputs
\item Codomain: set of all possible outputs
\end{itemize}
\subsection{Strings}
\label{sec:org9259da6}
\begin{itemize}
\item In computer science, strings are character arrays
\item In mathematics, strings are sequences of symbols
\item Specifically a string over an alphabet, \(\Sigma\), is a sequence of symbols belonging to \(\Sigma\)
\item \(\epsilon\) is the empty string
\item Concatenation: If w\textsubscript{1}, w\textsubscript{2} \(\in\) \(\Sigma\), w\textsubscript{1} \(\cdot\) w\textsubscript{2} = w\textsubscript{1}w\textsubscript{2}
\item If c \(\in\) \(\Sigma\), then \(\epsilon\) \(\cdot\) c = c \(\cdot\) \(\epsilon\) = c
\end{itemize}
\subsection{{\bfseries\sffamily TODO} Review Recursive Definitions}
\label{sec:org4787492}
\begin{itemize}
\item Base step: a step that can not be broken down any further, a fact that is always true regardless of the input
\item Recursive step:
\item Defining the length of a string over \(\Sigma\)
\begin{itemize}
\item Base: |\(\epsilon\)| = 0
\item Recursive:
\begin{itemize}
\item let w be a string over \(\Sigma\), and c \(\in\) \(\Sigma\)
\item then |w \(\cdot\) c| = |w| + 1
\end{itemize}
\end{itemize}
\item Using this to define |1011|,
\begin{enumerate}
\item |1011| = |101 \(\cdot\) 1| = |101| + 1 =
\item |10 \(\cdot\) 1| + 1 = |10| + 1 + 1 =
\item |1 \(\cdot\) 0| + 1 + 1 = |1| + 1 + 1 + 1 =
\item |\(\epsilon\) \(\cdot\) 1| + 1 + 1 + 1 =
\item |\(\epsilon\)| + 1 + 1 + 1 + 1 =
\item 0 + 1 + 1 + 1 + 1 = 4
\end{enumerate}
\end{itemize}
\subsection{Languages}
\label{sec:org15de515}
\begin{itemize}
\item Languages over \(\Sigma\) - a set of finite strings over \(\Sigma\)
\item Langauges recognized by an automaton, M, L(M) is the language accepted by M
\item \(\emptyset\) is the empty language
\item \(\epsilon\) \(\neq\) \(\emptyset\)
\item \(\epsilon\) \(\neq\) \{\(\epsilon\)\}
\item \(\epsilon\) is not a symbol in any alphabet
\end{itemize}
\section{01.14.21}
\label{sec:org8330122}
\subsection{Automaton (automata)}
\label{sec:org45ed3d7}
\begin{itemize}
\item Self running machine requiring a continuous power source
\begin{itemize}
\item Historically used power sources include water, steam, and electricity
\end{itemize}
\item Course revolves around defining the mathematics powering machines
\end{itemize}
\subsection{The Mathematics of Automata}
\label{sec:org5681dd1}
\subsubsection{Mathematicians \& History}
\label{sec:orge5b0207}
\begin{itemize}
\item Cantor defines sets as collections of objects
\item Cantor also argues that infinites can be of different magnitudes - there are infinitely more real numbers than natural numbers
\item Goedel eventually derives his incompleteness theorem
\begin{itemize}
\item No logical system that contains the natural numbers can prove its own soundness
\item Every sound logical system containing the natural numbers contains valid statements that cannot be proved or disproved
\end{itemize}
\item In 1936, Turing proves The Halting Problem is not decidable, it is impossible
\begin{itemize}
\item The Halting Problem is an algorithm that can analyze any other algorithm and determine whether or not it goes into an infinite loop
\end{itemize}
\item Turing creates the turing machine as an object consisting of sets and processes wherein the object can use any finite process to complete an action.
\item Turing machine sets the basis for a computer, which leads to a series of important questions:
\begin{itemize}
\item What can \& can't a machine do?
\item What does it mean for a problem ot be harder than another?
\item What does it mean for a machine to be more powerfule than another?
\end{itemize}
\end{itemize}
\subsubsection{Sequential Logic}
\label{sec:org9cc3259}
\begin{itemize}
\item Sentential Logic- based on boolean results
\begin{itemize}
\item Predicated on AND, OR, NOT
\item XOR, XAND, etc. can be derived using the above
\end{itemize}
\end{itemize}
\subsection{Necessary Review}
\label{sec:org3c8ed51}
\begin{itemize}
\item Textbook Ch. 0
\item Logic Statements
\item Set Theory
\item Functions
\end{itemize}
\subsection{Functions}
\label{sec:orgadcf5e5}
\begin{itemize}
\item Functions - something that maps objects from one set to another
\item Given f: a \(\to\) b;
\begin{itemize}
\item Everything in a is mapped to something in b
\begin{itemize}
\item For every x, such that x is an element of a, there exists a y, such that y is an element of b
\end{itemize}
\item No one point in the domain can be mapped to two different points in the codomain
\begin{itemize}
\item Logically, you can't have a function that takes in one input and returns two different outputs
\item If f maps x \(\to\) y1 and \(\to\) y2, y1 = y2
\end{itemize}
-\(\forall\) x \(\in\) A y\textsubscript{1},y\textsubscript{2} \(\in\) B [f(x)=y\textsubscript{1} \(\land\) f(x)=y\textsubscript{2} \(\to\) y\textsubscript{1} = y\textsubscript{2}]
\end{itemize}
\end{itemize}
\subsection{{\bfseries\sffamily TODO} Types of Functions - Definition \& Logical Statement}
\label{sec:orgcb6acd6}
\begin{itemize}
\item Injective Functions
\item Surjective Functions
\item Proof by Induction (\(\forall\))
\item Proof by Contradiction (\textlnot{}\(\exists\))
\end{itemize}
\subsection{Finite Automaton (Finite State Machine)}
\label{sec:org7361b71}
\begin{itemize}
\item States are logical confirgurations
\item States are generally based upon input
\item Purpose of a state machine is to make a yes/no decision
\end{itemize}
\end{document}
