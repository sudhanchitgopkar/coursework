% Created 2021-03-15 Mon 11:01
% Intended LaTeX compiler: pdflatex
\documentclass[11pt]{article}
\usepackage[utf8]{inputenc}
\usepackage[T1]{fontenc}
\usepackage{graphicx}
\usepackage{grffile}
\usepackage{longtable}
\usepackage{wrapfig}
\usepackage{rotating}
\usepackage[normalem]{ulem}
\usepackage{amsmath}
\usepackage{textcomp}
\usepackage{amssymb}
\usepackage{capt-of}
\usepackage{hyperref}
\usepackage[margin=1in]{geometry}
\author{Sudhan Chitgopkar}
\date{\today}
\title{Strategic Intelligence}
\hypersetup{
 pdfauthor={Sudhan Chitgopkar},
 pdftitle={Strategic Intelligence},
 pdfkeywords={},
 pdfsubject={},
 pdfcreator={Emacs 27.1 (Org mode 9.5)}, 
 pdflang={English}}
\begin{document}

\maketitle
\tableofcontents \clearpage\section{Intelligence Analysis}
\label{sec:org3b97a87}
\begin{itemize}
\item Analysts have the most direct contact with the consumer
\item Purpose of analysis:
\begin{itemize}
\item Information on difficult questions
\item Select most significant items
\item Tailor to consumers needs
\end{itemize}
\item Principles of analystic writing
\begin{enumerate}
\item Determine the context and put conclusions first
\item Know the customer's need
\item Organize logically
\item Understand formats
\item Use precise language
\item Economize your words
\item Strive for clarity of thought
\item Use active voice
\item Self-edit your writing
\end{enumerate}
\item Structure papers through AIM:
\begin{itemize}
\item A: Audience (who is the audience?)
\item I: Intelligence Question/Issues (what is the key question or concern? Is it actionable or answerable in more than one way?)
\item M: Message (what's the bottom line?)
\item S: Storyline (can the message be presented in a simple, cohesive manner?)
\end{itemize}
\item Tackling Analytical Arguments:
\begin{itemize}
\item Ground yourself
\item Assertion/Claim (Your main idea)
\item Evidence (Supporting the idea through warrants)
\item Relevance (link the claim and the evidence)
\item Acknowledge (the counterargument or other views)
\item Refine (restatement of the claim)
\end{itemize}
\item Inverted Pyramid Method
\begin{itemize}
\item Data-driven organization:
\begin{itemize}
\item Research question
\item Lit review
\item Data and hypotheses
\item Testing
\item Finding
\item Conclusion
\end{itemize}
\item Message-Driven Organization (used in intelligence)
\begin{itemize}
\item Title \& lead
\item Key judgements
\item Analysis
\item Appendices
\end{itemize}
\end{itemize}
\end{itemize}
\subsection{Going Wrong in Analysis}
\label{sec:org7e571b3}
\subsubsection{Heuristics}
\label{sec:orgca52fec}
\begin{itemize}
\item Cognitive shortcuts, mental simplifications
\item Anchoring effect (beginning from an incorrect starting position amd nuilding off it)
\item Associative Memory (predicting future events off rare events)
\item Availibility Heuristic (judging an event based on an analogy that first comes to mind)
\item Desire for Coherence and Uncertainty Reduction (Believing that coincidences or random events are actually patterns)
\item Groupthink (choosing the option that the majority of thr group agrees with to avoid in-group conflict)
\item Mental Shotgun (Lacking percision in the analsis or analytical thought)
\item Premature Closure (making an early decision without considering the entirety f the data or completing the analysis)
\item Satisficing (selecting the first answer that seems satisfying or appropriate)
\end{itemize}
\subsubsection{Cognitive Biases}
\label{sec:orgb2bf17b}
\begin{itemize}
\item mental errors caused by a simplification in thought
\item Confirmation Bias (considering only the data that seems to agree with your conclusion)
\item Evidence Acceptance Bias (Accepting data as true unless it was outright rejected)
\item Hindsight Bias (Believing that key items in the past were obvious to see at the time)
\item Mirror Imaging (incorrectly believing that others would act in the same way that we would, given the same information)
\item Vividness Bias (focusing on one vivid scenario instead of other potential scenarios or pieces of evidence)
\end{itemize}
\subsubsection{Intuitive Traps}
\label{sec:org6611c12}
\begin{itemize}
\item Assuming inevitability of an event
\item Assuming a single solution to an event
\item Confusing causality and correlation
\item Expecting marginal change (not expecting radical change)
\item Favoring firsthand information
\item Ignoring the effect of the absence of information
\item Ignoring initial base rate probabilities
\item Ignoring inconsistent evidence
\item Judging by emotion
\item Lacking sufficient bins or categories for the topic
\item Misstating probabilities, generally with subjective probabilities
\item Overestimating probabilities
\item Overinterpreting small samples
\item Overrating behavioral factors (fundamental attribution error)
\item Presuming patterns
\item Projecting past experiences
\item Rejecting evidence
\item Relying on first impressions
\end{itemize}
\subsubsection{Logical Fallacies}
\label{sec:org8a80778}
\begin{itemize}
\item Tautology
\item Inadequate sampling (Small sample size)
\item Hasty generalization
\item False analogy
\item False dichotomy  (reducing a situation down to two groups or choices)
\item Non-sequitur
\item Post Hoc, ergo Propter Hoc (After this, therefore because of this)
\item Slippery Slope
\item Red Herring (Distracting from the problem)
\item Ad Hominem
\item Ad Populum (focuses on public opinion, not facts)
\item Appeal to authority
\end{itemize}
\subsubsection{Analytical Traps}
\label{sec:org06dd68e}
\begin{itemize}
\item Familiarity (Reacting to sources and information you are already familiar with)
\item Out of date information or concepts
\item Satisficing
\item Oversimplification
\item Mismatched data and interpretations
\item Not consulting colleagues with broader perspectives
\item Vagueness
\end{itemize}
\section{Intelligence Processing}
\label{sec:orgf720deb}
\begin{itemize}
\item Consists of Evaluation and Collation
\end{itemize}
\subsection{Processing}
\label{sec:orgb4e6b60}
\begin{itemize}
\item Takes raw information and converts it into readable information for analysts to use
\item Accomplished through information management techniques
\end{itemize}
\subsection{Evaluation}
\label{sec:orgbcbe820}
\begin{itemize}
\item May occur during either the collection or processing phase
\item Focuses on source reliability + Data valididty
\end{itemize}
\subsubsection{Reliability}
\label{sec:orgaf0ac91}
\begin{itemize}
\item Generally can be evaluated on an A-F scale
\item Based on previous reporting from the source
\item F does not mean that it is bad information, it just comes from a source with no reporting history
\end{itemize}
\subsubsection{Validity}
\label{sec:org98a8a0f}
\begin{itemize}
\item Based on a 1-6 scale
\item Confirmed -> Probably true -> Possibly true -> Doubtfully true -> Improbable -> Cannot be judged
\end{itemize}
\subsection{Collation}
\label{sec:org7ac6b92}
\begin{itemize}
\item Grouping together of related items which facilitates further processing
\item Three types: (1) Automated, computer-driven data systems, (2) Manual visual formats, (3) Micrographic visual formats
\item Computer-Driven Data
\begin{itemize}
\item Recording extracted information in various formats
\item Includes digitized formats of hard data
\item Machine learning techniques
\begin{itemize}
\item Supervised machine learning
\begin{itemize}
\item Support Vector Systems
\item Naive Bayes
\item Random Forest
\end{itemize}
\item Unsupervised
\begin{itemize}
\item Neural Networks
\item Clustering
\item Latent Variable Models
\end{itemize}
\end{itemize}
\end{itemize}
\item Manual systems
\begin{itemize}
\item Cards
\item Files
\item Index Lists, etc
\end{itemize}
\item Micrographic formats
\begin{itemize}
\item Microfiche
\item Microfilm
\end{itemize}
\item Manual and micrographic formats are beneficial due to their accessibility, resource and time and intel dependent
\end{itemize}
\section{Intelligence Collection}
\label{sec:orga6c2368}
\begin{itemize}
\item HumInt = Human intelligence
\begin{itemize}
\item One of the oldest methods of collecting intelligence
\item Includes overt and clandestine activities
\item Can include diplomats, officials, hearings, etc.
\item Can provide key insights that technical collection can not
\item Can also provide documentary information
\item Most cost-effective
\item Three levels:
\begin{itemize}
\item Target-Specific: Closely managed, deep access, best ROI
\item Research-enabled: Lead-generated, environmental-monitoring with generalized searching with minimal investment and trying to focus collection
\item Opportunistic: Others come to you with specific information, low-hanging fruit
\end{itemize}
\end{itemize}
\item SigInt = Signals Intelligence
\begin{itemize}
\item Includes ComInt (communications), ElInt (Electronic Signals) FisInt (Foreign Instumentation Signals), TelInt (Telemetry)
\item ComInt is general broadcasting, ElInt is understanding non-explicit communications (radar), FisInt looks at different command and tool signals to estimate power/type of opponents tools
\end{itemize}
\item ImInt = Imagery Intelligence and analysis of imagery that comes from film, infrared, digital information
\begin{itemize}
\item ImInt provides geolocation, activity detection, facility analysis, area mapping
\item Disadvantages: Image quality generally degraded by darkness, weather, enemies that know ImInt is being used against them can fake information, requires tech-focused analysts
\end{itemize}
\item MasInt = Measurements and Signatures Intelligence
\begin{itemize}
\item Types of Intelligence
\begin{itemize}
\item Radar (RadInt)
\item Acoustic (AcoustInt)
\item Nuclear (NucInt)
\item Radio-Frequency/Electromagnetic Pulse (RF/EmpInt)
\item Electro-Optical (Electro-OpInt)
\item Laser (LasInt)
\item Materials (MatInt)
\item Chemical and Biological (CBInt)
\end{itemize}
\item Uses a cohesive picture of different measurements and signatures to gain intelligence
\item Done primarily by the DIA
\end{itemize}
\item GeoInt = Geospatial Imagery Information
\begin{itemize}
\item Focuses on depicting physical characteristics of geographical areas
\end{itemize}
\item TechInt = Technical Intelligence
\begin{itemize}
\item Focuses on weapons systems
\item Critical to covert operations
\end{itemize}
\item OSInt = Open Source Intelligence
\begin{itemize}
\item Use of materials available to the public
\item Generally use public databases
\item Problematic because journals often focus on theoreticals
\item Journalism may be used to decept adversaries
\end{itemize}
\end{itemize}
\subsection{Other Intelligence Tyoes}
\label{sec:orgbeeb9b2}
\begin{itemize}
\item Medical (MedInt) - looks at Medical status of a person or group
\item Financial (FinInt) - Looks at fnancial transactions of an individual or group
\item Cyber/Digital Network - Looks at exploitation potential for communication systems, computer threat intelligence
\item Protected Personal (ProtInt) - Exploitation of covert personal information and data
\item Social Media (SocmInt) - collective tools that allow for the analysis of social media at a more macro level and social media trends
\end{itemize}
\subsection{Collection Management}
\label{sec:org0aa4842}
\begin{itemize}
\item Intelligence Collection Plans (ICP)
\begin{itemize}
\item Requirement
\item Assets, Resources, Deterrents
\item Priorities
\item Taskings
\item Evaluation
\end{itemize}
\item NATO Collection Guidelines
\begin{itemize}
\item Discipline Selection
\item Alternative Disciplines
\item Support Resource Management
\end{itemize}
\end{itemize}
\section{Critiques of the Intelligence Cycle}
\label{sec:org31c34de}
\begin{itemize}
\item Hulnick argues that while the intelligence model is basic, it is inherently flawed because it isn't accurate
\item Regarding analysis, Hulnick finds that there is a disconnect between field officers and analysts, which is caused by lack of communication, cooperation, or mishandling of information
\end{itemize}
\section{Intelligence Cycle}
\label{sec:org52aa502}
\begin{itemize}
\item Decision-makers are expected to make the best-decision possible, and the intelligence cycle explains how intelligence is developed to give decision-makers the information they need
\item Components of the Intelligence Cycle
\begin{itemize}
\item Planning and Direction
\item Collection
\item Processing
\item Analysis
\item Dissemination
\end{itemize}
\end{itemize}
\subsection{Planning and Direction}
\label{sec:org6404c00}
\begin{itemize}
\item Policy-makers request intelligence on a particular subject or target
\item 3 subcategories
\begin{itemize}
\item Task Definiton - primary jumping off point
\item Analysis and formulation - consider all possible facets of the task, potential sources and challenges, and formulating the best plan. Very creative and challenging process. Critical to consider the specifics of the question and understand specifically what is requested from the customer/decision-maker
\item Core planning - allocation of finances, employees, and resources before the actual intelligence process can go underway. Specific steps and deadlines are also set up here.
\end{itemize}
\end{itemize}
\subsection{Collection}
\label{sec:org66225cd}
\begin{itemize}
\item Collection of raw information and intelligence
\item Draws on a variety of different types of information collection methods
\item Can come from a variety of different means (human, physical, technological, social media, etc.)
\end{itemize}
\subsection{Processing}
\label{sec:orgc12b942}
\begin{itemize}
\item Pre-analytical filtering
\item Collation refers to steps taken to turning raw data into something that may be analyzed, can also refer to proper organization. Transforming data into a readable state
\item Evaluation - combing through information to provide a credibility and validity scale to determine accuracy/reliability of information.
\begin{itemize}
\item Grading systems are used A-F, 1-6 to grade reliability of each sources
\item Grade determines the weight assigned to it
\item Evaluation also allows for security clearance can be derived for that information
\end{itemize}
\end{itemize}
\subsection{Analysis}
\label{sec:orgd025c2a}
\begin{itemize}
\item Analysts are generally subject matter experts and are tasked with creating a cohesive story with all of the information
\item Creation of written reports occurs here
\end{itemize}
\subsection{Dissemination}
\label{sec:orga3512e3}
\begin{itemize}
\item Distribution of final information and reports, leading to decisions and more intelligence tasks
\end{itemize}
\subsection{Critiques}
\label{sec:org1cb1f04}
\begin{itemize}
\item Cycle is over-simplified, doesn't account for specific types of intelligence collection or specifics of what policy-makers may want
\item Overly linear, process is much more complex and this is a simplistic representation. Furthermore, some stages can be started before others are done, states of the cycle are not discrete
\end{itemize}
\subsection{Additional Steps}
\label{sec:orgc27531e}
\begin{itemize}
\item Some argue that consumption should be included in the cycle. Disemmination is not the end of the cycle, the way in which information is consumed should be considered because of the effects that has on decision-making
\item Feedback may also be important to consider because of a continuous loop of feedback during and after the process is beind completed
\item Counter-intelligence and covert action not addressed by this cycle, which gives an incorrect understanding of contemporary intelligence operations
\end{itemize}
\subsection{Additional Questions}
\label{sec:orga625c02}
\begin{itemize}
\item Does the cycle's past affect its current use?
\item Should the cycle be refined?
\begin{itemize}
\item If so, how?
\item If not, what is the purpose of the cycle
\begin{itemize}
\item purely academic?
\item barebones basics?
\end{itemize}
\end{itemize}
\end{itemize}
\section{Johnson}
\label{sec:orgb56b5d3}
\subsection{Introduction}
\label{sec:org43a614f}
\begin{itemize}
\item Intelligence is defined as a set of activities carried out by government agencies that operate largely in secret including collection and interpretation of information from a mixture of open and clandestine sources to arrive at a product useful to illuminate foreign policy deliberations
\item These agencies also engage in covert action and manipulate events abroad
\end{itemize}
\subsection{The Intelligence Cycle}
\label{sec:org9691e16}
\begin{itemize}
\item describes the flow of activities for collection and analysis of info
\item not generally considered a defined cycle, rather considered a complex matrix of interactions
\item 5 stages
\begin{itemize}
\item planning and direction
\item collection
\item processing
\item production and analysis
\item dissemination
\end{itemize}
\end{itemize}
\subsection{Planning and Direction}
\label{sec:org8be2264}
\begin{itemize}
\item intel managers and policy officials must decide what data should be gathered
\item determine what the most critical information to policy-making is
\item scope = breadth of intelligence tasks
\item paradoxically, more wealthy nations are more likely to have information failures
\item The more affluent and globally oriented a nation, the larger its agenda of intelligence objectives and its institutional apparatus for espionage, and the more
likely its chances for a large number of successes as a result of this saturated
world coverage. For the same reason, they are more likely to experience international failures as they have very large global objectives.
\item As policymakers focus their informational needs and objectives, the chances of relevant intelligence successes increases
\end{itemize}
\subsection{Collection}
\label{sec:org4d0b75e}
\section{Turner}
\label{sec:orga7ff7b1}
\subsection{US Intelligence}
\label{sec:orgf20f984}
\begin{itemize}
\item While US intelligence is not funadmentally unique, it has a series of norms that make it very distinctive
\item Realism has been the dominant theory explaining intelligence gathering and behavior
\item Much of the US approach to intelligence takes from strategic culture, creating the US intelligence identity
\end{itemize}
\subsection{Constructivism}
\label{sec:orgb266cca}
\begin{itemize}
\item Constructivists see intelligence as highly malleable, made up of historical processes, accepted behavior, and contemporary beliefs and interests
\end{itemize}
\subsection{American Strategic Culture}
\label{sec:orgb5ffefd}
\begin{itemize}
\item While American culture and opinion is very varied, there are central themes defining strategic intelligence
\item 3 influences shape americans view of national security
\begin{enumerate}
\item lack of a sense of history
\begin{itemize}
\item leads to a positive, successful image of thesmselves
\end{itemize}
\item unique geography
\begin{itemize}
\item historical isolationism, general security, significant resources
\end{itemize}
\item Anglo-saxon heritage
\begin{itemize}
\item aversion to/suspicion of military and attachment to constitutionalism
\end{itemize}
\end{enumerate}
\end{itemize}
\subsection{Distinguishing Norms}
\label{sec:orge485da5}
\begin{itemize}
\item Institutional Survival - all intelligence agencies are bureaucracies trying to maximize resources and funding in the political marketplace
\item Secrecy - conflicts with American belief in the transparency of government, is foundational to the USIC, has lef to abuse and problems
\item Exceptionalism - occurs because of (1) secrecy, (2) breaking other country's laws, (3) subject to deception and disinformation, and (4) intelligence is fungible and can be used by politicians for a wide variety of purposes
\item Ambiguous Mandate - Mission has always been vague to fudge priorities and targets
\item Confederal Structure - While americans oppose a strong central intelligence authority, fragmentation
\item Competitive Intelligence - each bureau tries to compete with the other to increase innovation, eventually just ends up in redundancy and waste
\item Flexible Accountability - Many systems are rooted in accountability but intelligence seems to often get a free pass on many missions
\item Intelligence-Law Enforcement Separation - exists due to fear of combination and overpower (eg. Gestapo) and that intelligence is considered inherently different than law enforcement
\item Separation of Intelligence from Policy - Many argue that for intelligence to be truly objective, it must be separated from policy.  Some argue that intelligence works best when it is in tuen with a policy-makers objectives
\item Policy Support - due to the separation, intelligence is an area of the government. There is disagreement about how much intelligence should be used to support or advocate for a policy
\item ``Can Do'' Attitude - optimism and risk-taking inherent to intelligence efforts
\item Primacy of analysis - US has very significant amount of analysis capabilities, especially because of the role of intelligence on policy and decision-making in government
\item ``Accurate, timely, and relevant intelligence'' - phrase has become a mantra within USIC and shows main principles of intelligence work
\end{itemize}
\subsection{Road to Failure}
\label{sec:orgb54f5c2}
\begin{itemize}
\item Norms of USIC indicate uncertainty about the role of intelligence in government and society
\item US Intelligence is the product of political compromise and checks and balances, with certain positive qualities giving intelligence workers the tools necessary to do their job
\item Many good and bad aspects to intelligence
\item As a whole, intelligence identity of the US reflects the fact that people want intel to serve the national interest, but abide by the conutry's democratic princples - which eventually helps conduct important work but also sets up the intel community for failure in some cases.
\end{itemize}
\section{Warner}
\label{sec:org71b2247}
\begin{itemize}
\item No official definition for intelligence exists
\end{itemize}
\subsection{Intelligence Definitions}
\label{sec:orgf44dfd6}
\begin{itemize}
\item National Security Act of 1947 defines foreign intelligence as ``information relating to the capabilities, intentions, or activities of foreign governments or elements thereof.''
\item Hoover Commission 1955 define that ``intelligence deals with all the things which should be jnown in advance of initiating a course of action''
\item Brown-Aspin Commission defines that intelligence is ``simply and boradly information about things foreign - people, places, things, and events - needed by the government for the conduct of its functions''
\item Joint Chief if Staffs Dictionary of Military and Associated Terms defines it as ``the product resulting from the collection, processing, integration, analysis, evaluation, and interpretation of available information concerning foreign countries or areas'' or as ``information and knowledge about an adversary obtained through observation, investigation, analysis, or understanding''
\item CIA defines intelligence as ``the knowledge and foreknowledge of the world around us - the prelude to decision and action by US policymakers''
\end{itemize}
\subsection{Definition Analysis}
\label{sec:org46fafeb}
\begin{itemize}
\item Most definitions stress information over organization
\item Defining intelligence simply as information is generally to broad for intelligence professionals to carry out their jobs
\item Not every single peiece of information is intelligence
\item Intelligence can then be considered both an action and a product
\item Shulsky emphasizes the secret nature of this information as being a critical aspect of intelligence
\end{itemize}
\subsection{Final Steps}
\label{sec:org37573f9}
\begin{itemize}
\item Intelligence is then an activity and a product conducted through confidential circumstances on behalf of states so that policy-makers can understand foreign developments, and that it includes clandestine operations performed to cause certain foreign effects
\item Difference between law enforcement and intelligence is secrecy
\end{itemize}
\section{Intelligence Structure}
\label{sec:org2af7ef9}
\subsection{What is Intelligence}
\label{sec:org17effc4}
\begin{itemize}
\item Process
\item Activity - the actual job conducted by an individual or organization to obtain intelligence
\item Final Product - the final report or analysis derived through the process of gaining intelligence that is eventually disseminated
\item Elements of Intelligence
\begin{enumerate}
\item Dependent on confidential sources and methods for full effectiveness
\item Performed by officers of the state, for the state
\item Focused on foreigners - usually other states, but often foreign subjects, corporations, or groups
\item Linked to the production and dissemination of information
\item Involved in influencing foreign entities through means that can't trace back to the acting government
\end{enumerate}
\item Concise Definition: Intelligence is secret, state activity to understand or influence foreign entities
\end{itemize}
\subsection{Levels of Analysis}
\label{sec:org5cc73ba}
\begin{itemize}
\item Strategic Intelligence - broad, policy-oriented approach to intelligence. Understands the effects of intelligence and international factors on the world
\item Operational Intelligence - group-focused intelligence, understanding interplay between groups of people or institutions
\item Tactical Intelligence - low-level intelligence focused on field scenarios and day-to-day operations of intelligence
\end{itemize}
\subsection{US Intelligence Community}
\label{sec:org96475dd}
\subsubsection{Independent}
\label{sec:org4372c73}
\begin{itemize}
\item Office of the Director of National Intelligence (ODNI) - intermediary oversight agency consolidating all of the intelligence and pushing it to policy-makers
\item Central Intelligence Agency (CIA) - Leading expert in clandestine operations for the US, uses their own paramilitary. Only independent agency that runs operations
\end{itemize}
\subsubsection{Departments of Agencies}
\label{sec:org8aff9cf}
\begin{itemize}
\item Department of Energy Office of Intelligence and Counterintelligence (DOE-OIC) - leading experts in nuclear weapons, energy infrastructure, and security maintenance
\item Department of Homeland Security's Office of Intelligence and Analysis - Domestic security focus
\item FBI Intelligence Branch (FBIIB) - Focus on federal crimes and domestic security
\item DEA Office of National Security Agency (DEAONSI) - focus on drugs and drug trade
\item Department of Small Business Innovation Research (DOSBIR) - focused on diplomatic intelligence
\item Dept of Treasury Intelligence Agency (USDTOIA) - understanding how the US dollar could be used in criminal activities
\item US Coast Guard Intelligence (USCGI) - charged with keeping ports, waterways, cargo, and coasts safe
\end{itemize}
\subsubsection{Department of Defense}
\label{sec:org69bcea0}
\begin{itemize}
\item Defense Intelligence Agency (DIA) - DoD's version of the CIA, focused on troop movements, troop aquisitions
\item National Security Agency/Central Security Service (NSA/CSS) - leading experts in signal communications and telecommunications
\item National Geospatial Intelligence Agency (NGA) - focuses on GIS, geography
\item National Reconnaissance Office (NRO) - one of the most secret agencies, existence wasn't acknowledged until the '50's, control spy sattelite network
\item US Army Intelligence (USAI) - control field operations and movement of troops
\item Office of Naval Intelligence (ONI) - control water-based troop movements, cargo movement, and political intelligence
\item US Marine Corps Intelligence (USMCI) - provide tactical intelligence for troop movements through surge and occupancy operations, also provide counterintelligence consultation to the rest of the USIC
\item US Air Force Intelligence, Surveillance, and Reconnaissance (USAFISR/16AF) - focus on imagery intelligence, security countermeasures, telecommunications
\end{itemize}

\subsection{Questions to Consider}
\label{sec:orga415300}
\begin{itemize}
\item Do you agree with the concise definition of intelligence?
\item Which level of analysis would you most focus on?
\item Can you see the bureaucracy of the intelligence community helping or hurting its overall mission?
\end{itemize}
\end{document}
