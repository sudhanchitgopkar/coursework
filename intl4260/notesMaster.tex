% Created 2021-01-20 Wed 11:13
% Intended LaTeX compiler: pdflatex
\documentclass[11pt]{article}
\usepackage[utf8]{inputenc}
\usepackage[T1]{fontenc}
\usepackage{graphicx}
\usepackage{grffile}
\usepackage{longtable}
\usepackage{wrapfig}
\usepackage{rotating}
\usepackage[normalem]{ulem}
\usepackage{amsmath}
\usepackage{textcomp}
\usepackage{amssymb}
\usepackage{capt-of}
\usepackage{hyperref}
\author{Sudhan Chitgopkar}
\date{\today}
\title{}
\hypersetup{
 pdfauthor={Sudhan Chitgopkar},
 pdftitle={},
 pdfkeywords={},
 pdfsubject={},
 pdfcreator={Emacs 27.1 (Org mode 9.5)}, 
 pdflang={English}}
\begin{document}

\tableofcontents

\section{01.20.21}
\label{sec:org6771a37}
\subsection{Rational Decision-Making}
\label{sec:org3c5299d}
\begin{itemize}
\item Rational decision-making defines how we make decisions
\item A person's or institutions goal is not relevant, the process of pursuing that goal is the factor driving decision-making
\item This allows us to generalize decision-making significantly more
\item Critical to consider the probabilistic nature of benefits and harms when considering rational decision-making
\item Expected Value = ``weighted value'' for all costs and benefits
\begin{itemize}
\item Same thing as ``average payoff''
\end{itemize}
\end{itemize}
\subsection{Incentive Structures}
\label{sec:org5d3a98f}
\begin{itemize}
\item Incentive structures are the expected values for each of the strategies considered
\item Incentive structures impose a certain course of action upon us, given that we are rational actors
\end{itemize}
\section{Fearon}
\label{sec:org4201eea}
\begin{itemize}
\item Three reasons war may occur
\begin{itemize}
\item People are sometimes irrational and don't consider the costs of war due to this irrationality or their biases
\item Leaders may enjoy benefits of war but not pay the costs
\item People are rational and consider the risks but fight anyway (Rationalist explanation)
\end{itemize}
\item Flaws with contemporary rationalist arguments are that they don't address prewar bargains
\item Contemporary Rationalist reasons for war:
\begin{itemize}
\item Anarchy
\item Benefits o/w costs
\item Rational preventitive war
\item Rational miscalculation due to lack of info
\item Rational miscalculation due to diagreement about relative power
\end{itemize}
\item Fearon's reasons for war
\begin{itemize}
\item private or misrepresented info about relative capabilities
\item relationships are not possible because at least one party has an incentive to cheat
\item Despite being able to compromise, one or more party does not want to because of their beliefs on the issue
\end{itemize}
\end{itemize}
\subsection{The Puzzle}
\label{sec:orgfb41018}
\begin{itemize}
\item People often see war as something nobody wants though wars can often simply be costly but worthwhile gambles
\item Wars are always ex post inefficient because no matter how small, the costs of fighting still exist
\end{itemize}
\subsection{Anarchy}
\label{sec:org21e5785}
\begin{itemize}
\item War occurs because there is nothing to prevent it
\item Does not explain why wars still occur due to their inefficiency, therefore does not explain war completely
\item Anarchy may lead to arms races and insecurity, but little war outside or preemptive war
\end{itemize}
\subsection{Preventive War}
\label{sec:orgb3bbc5a}
\begin{itemize}
\item If a declining power suspects that it may be attacked in the future by a rising power, it will find a preventive war rational
\item Theory does not consider diplomacy and timeframe
\item Why should the declining power fear an attack if it's inefficient, even for the rising power
\end{itemize}
\subsection{Positive Expected Utility}
\label{sec:orgae45e7c}
\begin{itemize}
\item Argues that war is rational when both sides have a positive expected utility from it
\item While often presented, this argument doesn't explain specific condition in which both parties fighting a war have positive expected utility
\end{itemize}
\section{Expected Profit Khan Academy}
\label{sec:org0a74e8f}
\begin{itemize}
\item Expected value can be calculated as the sum of all the outcome probabilities multiplied by their corresponding profits.
\item Considering all outcome probabilities should yeild a total probability sum of 1 (100\%), with profits being positive (gains) or negative (losses)
\end{itemize}
\section{01.15.21}
\label{sec:org4e314b9}
\begin{itemize}
\item Brain has a complex set of structures that work together to do both really important, and fundamentally flawed actions
\end{itemize}
\subsection{Brain Stem}
\label{sec:org65dbfdd}
\begin{itemize}
\item The reptilian brain
\item Really just an extensino of the spinal cord
\item Controls automatic actions, no effect on decision-making
\end{itemize}
\subsection{Middle Brain (Limbic System)}
\label{sec:org37db9aa}
\begin{itemize}
\item Body's monitoring system to identify important elements of the environment
\item Discriminates things of importantance constantly and ambiently
\end{itemize}
\subsection{Brain Cortex}
\label{sec:org7a9fb19}
\begin{itemize}
\item Controls higher-level thinking
\item Moral decision-making, learning, conscious awareness
\end{itemize}
\subsection{Hierarchy of the brain}
\label{sec:org16f9b30}
\begin{itemize}
\item Information goes from the brain stem, to the limbic system, to the brain cortex
\item Critically, the limbic system was never designed to collect all the information around you - that incomplete information is used for decision-making
\item Understanding the interplay and potential biases of the limbic system can help us understand decision-making and prevent bad decision making
\item Fear and the triggering of fear prevents higher-level decision making and can prevent the intake of new information
\item Sources of information can also have a significant effect on the processing of that information - can be seen through in-group/out-group bias
\end{itemize}
\subsection{Rational Decision-Making}
\label{sec:org8677059}
\begin{enumerate}
\item Pick a goal
\item Evaluate all strategies
\begin{itemize}
\item Analyze costs
\item Analyze benefits
\end{itemize}
\item Select strategy with best cost/benefit ratio
\item Bias often occurs at stage 2 because of filtration of information through the limbic system
\end{enumerate}
\end{document}
