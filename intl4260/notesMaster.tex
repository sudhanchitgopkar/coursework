% Created 2021-02-15 Mon 23:04
% Intended LaTeX compiler: pdflatex
\documentclass[11pt]{article}
\usepackage[utf8]{inputenc}
\usepackage[T1]{fontenc}
\usepackage{graphicx}
\usepackage{grffile}
\usepackage{longtable}
\usepackage{wrapfig}
\usepackage{rotating}
\usepackage[normalem]{ulem}
\usepackage{amsmath}
\usepackage{textcomp}
\usepackage{amssymb}
\usepackage{capt-of}
\usepackage{hyperref}
\usepackage[margin=1in]{geometry}
\author{Sudhan Chitgopkar}
\date{\today}
\title{Foreign Pol. Decision Making}
\hypersetup{
 pdfauthor={Sudhan Chitgopkar},
 pdftitle={Foreign Pol. Decision Making},
 pdfkeywords={},
 pdfsubject={},
 pdfcreator={Emacs 27.1 (Org mode 9.5)}, 
 pdflang={English}}
\begin{document}

\maketitle
\section*{02.15.21}
\label{sec:org1d9a1d6}
\subsection*{Fairness \& Fairness Frames}
\label{sec:org0192469}
\begin{itemize}
\item Fairnesss and fairness frames, while stable, are rarely rational
\item Culture defines fairness
\item The violation of culturally-defined fairness causes people to reject offers that are objectively good for them
\item This occurs both in cases where the offers are unfair in favor of them (though the studies here are less robust) and in cases where the offers are unfair against them (this is more common)
\item It is not necessarily that we seek fairness so much as we are averse to unfairness
\item Because of this, using a fairness frame in foreign and domestic policy is extremely effective
\end{itemize}
\section*{02.10.21}
\label{sec:org9ce4508}
\subsection*{Prospect Theory (cont.)}
\label{sec:orgf623cce}
\begin{itemize}
\item It is critical for our leaders to be aware of:
\begin{itemize}
\item How they frame issues as they make decisions
\item How they frame issues when they talk to us
\item The frame our adversaries use when we are in conflict with them
\end{itemize}
\item Loss-framing is a critical part of negative campaigining
\item Democratic institutions create multiple frames, which is not necessarily the case in non-democratic regimes
\item In non-democratic regimes, leaders and the public may settle into a frame (generally loss framing), which leads them to make riskier decisions
\item One significant benefit of democratic systems is that they are less likely to settle into a particular frame, which may be bad
\end{itemize}
\section*{02.08.21}
\label{sec:org0bd0f8c}
\subsection*{Prospect Theory}
\label{sec:orgcb277fa}
\begin{itemize}
\item When we frame outcomes in terms of gains, we tend to be risk-averse, even when the gambles are objectively better
\item When we frame outcomes in terms of losses, we tend to be risk-acceptant in order to avoid larger losses, even when the gambles are objectively worse for us
\item This can often be seen in stock market crashes and stock behavior, wherein people sell during a stock downturn when the rational decision is to hold on to the stock
\begin{itemize}
\item This can further be seen with (1) Carter and the Iranian Hostage Crisis, (2) The 2016 election of Trump, the 2020 election of Biden
\end{itemize}
\item To remove yourself from a frame (regardless of whether it is risk-averse or risk-acceptant), it is critical to consider both benefits and losses
\item Risk aversion and prospect theory is applicable in multiple scenarios:
\begin{itemize}
\item Leaders, themselves, have risk-aversive tendencies and follow prospect theories in their decision-making
\item The public can constrain the win-set of the leaders through their risk-aversive tendencies and tendency to follow prospect theory
\item Leaders can change the public's perspective on a topic through framing it in a different way and playing on prospect theory and loss aversion
\end{itemize}
\end{itemize}
\section*{02.05.21}
\label{sec:orgc154c4d}
\subsection*{Loss Aversion \& Policy}
\label{sec:org7e804b3}
\begin{itemize}
\item Argues that policies are either loss averting or gain seeking
\begin{itemize}
\item Berejikian argues against this
\item Any policy can be reframed to be either loss avoiding or gain seeking
\end{itemize}
\item Public is more favorable towards policies that are loss avoiding
\end{itemize}
\section*{02.03.21}
\label{sec:org7bdbc33}
\subsection*{Concession Aversion}
\label{sec:orgcac1749}
\begin{itemize}
\item Because of loss aversion, anything given up in a negotiation has an inflated value.
\item Because this is true for both sides of a negotiation, there is a permanent hurdle to achieving a negotiated settlement
\item Also known as the endowment effect
\item concession aversion and loss aversion is stronger when bad behavior has started
\begin{itemize}
\item It is harder to stop behavior that has already been started than it is to pre-empt bad behavior
\end{itemize}
\item Due to loss aversion, states often double down on their failed policies and don't correct course
\item Empirically, great powers always decline and hasten that decline by trying to hold fast to their previous status
\begin{itemize}
\item Great powers can either try harder to hold on to their power and influence
\end{itemize}
\begin{itemize}
\item or great powers can accept their declining status and recalibrate their policies accordingly
\end{itemize}
\end{itemize}
\section*{02.01.21}
\label{sec:orgf844e73}
\subsection*{Loss Aversion}
\label{sec:org681a8d9}
\begin{itemize}
\item Consider a simple dilemma, wherein an actor chooses whether to play a game.
\begin{itemize}
\item A fair coin is flipped
\item If the coin is heads, the actor recieves \$125
\item If the coin is tails, the actor loses \$125
\end{itemize}
\item This dilemma has an EV of (0.5 * 125) + (0.5 * -125) = 0
\item Despite an EV of 0, the vast majority of people would not play this game
\item This is a result of loss aversion, people cognitively weigh losses and harms more than they would weigh wins and benefits
\item When compared to gains, equivalent losses hurt more
\item Pain and loss aversion is a more intense feeling than gain seeking
\item This pehnomenon is consistent across regions and cultures
\item To take advantage of this, framing each decision as loss aversion (loss framing) instead of gain seeking makes it more likely to be accepted
\item Under time pressure, individuals will be more resolved to avoid losses
\end{itemize}
\section*{01.29.21}
\label{sec:orga45fc1c}
\subsection*{Sagan Review}
\label{sec:orgda0f542}
\begin{itemize}
\item If nuclear weapons were a norm,
\begin{itemize}
\item conventional weapons would be preferred to nuclear weapons
\item this preference would not change even if the utility of nuclear weapons, when compared to conventional weapons, increased
\end{itemize}
\item Cricially, Sagan finds that although Americans prefer conventional weapons when presented a choice, a large proportion are willing to approve of a nuclear strike after the fact
\item The fraction of people that approved a nuclear strike grew with its effectiveness, indicating that perceptions towards nuclear use is based on nuclear utility
\item There is no significant domestic political constraint on nuclear weapon usage
\item Vast majority of people focused on utility to make their decision on nuclear weapons, with few focusing on moral factors
\end{itemize}
\section*{01.27.21}
\label{sec:org1c5b416}
\subsection*{Norms \& Nukes}
\label{sec:orgce187fb}
\begin{itemize}
\item Norms and taboos have been critical in nuclear policy
\item Norms and taboos have changed our definition of nuclear weapons as solely a means of deterrence to something that is more feasible in small-scale war (through more tactical nuclear weapons)
\end{itemize}
\subsection*{Norms \& Taboos}
\label{sec:orgf8a8bff}
\begin{itemize}
\item Cognitive constructs designed to guide our behavior
\item Generally exist in the context of societal interaction and behavior
\end{itemize}
\subsubsection*{Norms}
\label{sec:orga091ce6}
\begin{itemize}
\item Do's and Dont's
\begin{itemize}
\item prescribe some behavior and deter other behaviors
\end{itemize}
\item Context-specific
\begin{itemize}
\item e.g killing is generally considered something that is horrible to do, but is just
\end{itemize}
\item Large cultural variation in norms
\item Consequences for violation of norms can vary significantly
\end{itemize}
\subsubsection*{Taboos}
\label{sec:org9bba66f}
\begin{itemize}
\item Dont's - never explain things you should do, only address things you shouldn't
\item Tend to be universal, with limited exception
\item Significantly more limited variation, easier to translate across cultures
\item There tend to be very severe consequences to taboo violations
\end{itemize}
\subsection*{Norms, Taboos, and Decision-Making}
\label{sec:orgc06de3c}
\begin{itemize}
\item Generally, norms and taboos take certain strategies off the table and constrain the incentive structure
\item Taboos and Norms also change our win-sets because of norms and taboos that exist in their own societies
\item When the government needs to violate taboos or norms, they (1) argue that they aren't, through some loophole, and (2) argue that the benefits outweigh the harms
\item To erode a norm, infuse it with consequentialist logic
\end{itemize}
\section*{01.21.21}
\label{sec:org7194983}
\subsection*{Public Opinion \& Decision-Making}
\label{sec:orga699eb4}
\begin{itemize}
\item domestic decision-making has significant foreign policy effects
\item domestic public opinion has an effect on foreign policy decisions
\begin{itemize}
\item public opinions can help shape the incentive structure that a decision-maker has when faced with a decision
\item public opinion can either be an opportunity or a cosntraint
\end{itemize}
\item who we listen to and what they say can intrinsically change our incentive structure
\end{itemize}
\subsection*{Putnam Review}
\label{sec:org2ce9233}
\begin{itemize}
\item For any foreign policy issue, there is a chief of government (CoG)
\begin{itemize}
\item CoG has ultimate decision-making authority on the issue
\item e.g, president on war, secretary for commerce on tariffs, etc.
\end{itemize}
\item CoG's job is to find a way to align their international goals with what's possible domestically
\begin{itemize}
\item This is the two-level game, domestic and international balances
\end{itemize}
\item ``Win-Set'' defines the set of acceptable outcomes that is affected by the policy
\begin{itemize}
\item ``Win-Set'' defined by war and peace is the entire US population
\item ``Win-Set'' defined by shoe-lace imports is extremely small
\end{itemize}
\item Veto power must be considered
\item Assumes a rational actor analysis
\item While this applies largely to democratic states, the same general principles can also appply to authoritarian regimes
\end{itemize}
\section*{Putnam}
\label{sec:org0e73eea}
\subsection*{Domestic-International Entanglements}
\label{sec:org0ae4f42}
\begin{itemize}
\item Current literature lists domestic influences on foreign policy and theorizes about links between the two
\item Deutsch and Haas theorize about the impact of parties and interest groups on spillover from domestic policy to international objectives
\item Recent work has focised on structural factors such as state strength causing an effect on foreign economic policy
\begin{itemize}
\item central decision-makers must be concerned with domestic and international factors simultaneously
\item theory does not properly explain differences in state foreign policy occurring despite static state structures
\end{itemize}
\end{itemize}
\subsection*{Two-level Games}
\label{sec:org0fc50be}
\begin{itemize}
\item Politics of international negotiations can often be considerd a two-level game
\begin{itemize}
\item At the national level, domestic groups pressure governemnt, politicians seek power through the the favor of those organizations
\item At the international leve, governments seek to minimize harms, maximize ability to solve domestic pressures
\end{itemize}
\item Creates a very complex, sometimes contradictory situation for actors at both boards (decision-makers)
\end{itemize}
\subsection*{Win-Sets}
\label{sec:org39e7cc6}
\begin{itemize}
\item Negotiation occurs at a 2-stage process:
\begin{itemize}
\item Level 1: bargaining between negotiators leading to tentative agreement
\item Level 2: Separate discussions within each group about ratification
\end{itemize}
\item In reality, process is not always linear - generally happens multiple times in multiple stages at multiple levels
\item Larger win-sets make Level 1 agreement more likely
\item The relative size of the respective level 2 win-sets will affect the distribution of the joint gains from the international bargain (the larger the win-set of actor 1, the more he can be pushed around by other actors)
\end{itemize}
\subsection*{Win-Set Determinants}
\label{sec:orgf60064a}
\begin{itemize}
\item Three factors critical to win-set size
\begin{itemize}
\item Level 2 preferences and coalitions
\item Level 2 institutions
\item Level 1 negotiation strategies
\end{itemize}
\end{itemize}
\subsection*{Uncertainty and Bargaining}
\label{sec:org15900c6}
\begin{itemize}
\item Level 1 negotiators are often badly misinformed about elvel 2 politics, especially on the opposing side
\item Uncertainty about win set size can be both good and bad in 2 level negotiations
\item Each bargainer has an incentive to understate his own win-sets
\item Uncertainty about opponent's win set increases concern about risk of involuntary defection by the other side
\end{itemize}
\subsection*{Role of the Chief Negotiator}
\label{sec:org2429c82}
\begin{itemize}
\item Chief negotiator is the only formal link between level 1 and 2 of negotiation
\item Assumed that chief negotiator has no independent policy views, acts merely as an honest broker on behalf of his constitutents
\item Motives of the chief negotiator:
\begin{itemize}
\item enhancing level 2 game by having benefits outweigh harms as much as possible
\item shifting balance of power at level 2 in favor of his own person domestic policies
\item pursuing his own conception of national interest in the international sense
\end{itemize}
\begin{itemize}
\item Also assumed that the chief negotiator has some sort of veto power to outright reject anything that wholly contradicts his personal beliefs
\end{itemize}
\end{itemize}

\section*{Fearon}
\label{sec:org3244e2d}
\subsection*{Introduction}
\label{sec:orgbf7c07e}
\begin{itemize}
\item Three reasons war may occur
\begin{itemize}
\item People are sometimes irrational and don't consider the costs of war due to this irrationality or their biases
\item Leaders may enjoy benefits of war but not pay the costs
\item People are rational and consider the risks but fight anyway (Rationalist explanation)
\end{itemize}
\item Flaws with contemporary rationalist arguments are that they don't address prewar bargains
\item Contemporary Rationalist reasons for war:
\begin{itemize}
\item Anarchy
\item Benefits o/w costs
\item Rational preventitive war
\item Rational miscalculation due to lack of info
\item Rational miscalculation due to diagreement about relative power
\end{itemize}
\item Fearon's reasons for war
\begin{itemize}
\item private or misrepresented info about relative capabilities
\item relationships are not possible because at least one party has an incentive to cheat
\item Despite being able to compromise, one or more party does not want to because of their beliefs on the issue
\end{itemize}
\end{itemize}
\subsection*{The Puzzle}
\label{sec:org97b8696}
\begin{itemize}
\item People often see war as something nobody wants though wars can often simply be costly but worthwhile gambles
\item Wars are always ex post inefficient because no matter how small, the costs of fighting still exist
\end{itemize}
\subsection*{Anarchy}
\label{sec:org30bb343}
\begin{itemize}
\item War occurs because there is nothing to prevent it
\item Does not explain why wars still occur due to their inefficiency, therefore does not explain war completely
\item Anarchy may lead to arms races and insecurity, but little war outside or preemptive war
\end{itemize}
\subsection*{Preventive War}
\label{sec:orgecc186c}
\begin{itemize}
\item If a declining power suspects that it may be attacked in the future by a rising power, it will find a preventive war rational
\item Theory does not consider diplomacy and timeframe
\item Why should the declining power fear an attack if it's inefficient, even for the rising power
\end{itemize}
\subsection*{Positive Expected Utility}
\label{sec:org83db5d3}
\begin{itemize}
\item Argues that war is rational when both sides have a positive expected utility from it
\item While often presented, this argument doesn't explain specific condition in which both parties fighting a war have positive expected utility
\end{itemize}
\subsection*{Utility and Rationality}
\label{sec:org2ee1490}
\begin{itemize}
\item Positive expected utility alone is not enough to provide a rationalist explanatino for war
\item Indivisibility of factors of war can also be a rational explanation of war
\end{itemize}
\subsection*{War and Private information}
\label{sec:org1bac424}
\begin{itemize}
\item War is often the product of rational miscalculation
\item Leaders overestimate their chance of military victory
\item State lack information about other side's willingness to fight
\item Truly rational agents will make the same prediction about the outcome of an uncertain event when given the same set of facts
\begin{itemize}
\item This does not happen when miscalculation occurs, which leads to war
\end{itemize}
\item There also exist incentives to misrepresent in bargaining
\item Combination of private info about relative power or will to fight and strategic incentive to misrepresent positions in baragaining constitute a rational explanation of war
\end{itemize}
\subsection*{War \& Commitment Problems}
\label{sec:org00caef6}
\begin{itemize}
\item With anarchy, states become suspicious of one another and build weapons and engage in attacks
\item Anarchy matters when it seems as if a states preferences and opportunities for action imply that one or both sides in a dispute have incentives to renege on peaceful bargains which would be mutually preferable to war
\item Preemptive war is one such case where if one wants to go to war, doing so stealthily would be the most save. While both parties would prefer to live in peace, they are constantly afraid of doing so because of the anarchic state of internaitonal affairs
\begin{itemize}
\item Seems to work similar to the prisoner's dilemm
\end{itemize}
\item The same principle can be applied to preventive war, lack of trust is not the driving factor behind war in these instances. Rather, circumstances that give one party an incentive to renege are
\end{itemize}
\subsection*{Conclusion}
\label{sec:orga547c8a}
\begin{itemize}
\item Because fighting is costly and risky, rational actors should prefer negotiations to war
\item Rational actors may be unable to agree on these negotiations because
\begin{itemize}
\item private information about resolve and capability, and the incentives that exist to misrepresent these
\item inability to commit to hold ip a deal
\end{itemize}
\item Not arguing irrelevance for empirical studies concluding that war is based on irrationality
\end{itemize}
\section*{01.22.21}
\label{sec:orgbfb7198}
\subsection*{Calculating Costs of War}
\label{sec:org22c2b53}
\subsubsection*{Constants}
\label{sec:org3cc4c05}
\begin{itemize}
\item Fight occurs over \$100
\item Cost of war: \$20
\item P(Winning): 50\%
\end{itemize}
\subsubsection*{Expected Value}
\label{sec:orgd7b1f2b}
\begin{itemize}
\item (Gains Winning) + (Gains Fighting) - (Cost of War)
\item (0.5 \texttimes{} 100) + (0) - (1 \texttimes{} 20) = 50 - 20 = \$30
\item Because each side could negotiate in order to get an expected value of 31 <, it is not a rational deciison to go to war
\end{itemize}
\subsubsection*{Miscalculation}
\label{sec:org55cf9ec}
\begin{itemize}
\item When both sides overestimate the probability of winning, their expected value goes up, thereby making their minimum threshold for negotiation too high for the other side.
\item Consider miscalculation wherein both sides believe they have an 80\% probability of winning:
\begin{itemize}
\item EV\textsubscript{war} = (0.8 \texttimes{} 100) + (0) - (1 \texttimes{} 20) = 80 - 20 = 60.
\item Both sides therefore want an expected value of > 60, which is impossible given the limited value of the thing being fought over
\end{itemize}
\end{itemize}
\section*{01.20.21}
\label{sec:org5875a6f}
\subsection*{Rational Decision-Making}
\label{sec:orgf2bc108}
\begin{itemize}
\item Rational decision-making defines how we make decisions
\item A person's or institutions goal is not relevant, the process of pursuing that goal is the factor driving decision-making
\item This allows us to generalize decision-making significantly more
\item Critical to consider the probabilistic nature of benefits and harms when considering rational decision-making
\item Expected Value = ``weighted value'' for all costs and benefits
\begin{itemize}
\item Same thing as ``average payoff''
\end{itemize}
\end{itemize}
\subsection*{Incentive Structures}
\label{sec:org6eb4d47}
\begin{itemize}
\item Incentive structures are the expected values for each of the strategies considered
\item Incentive structures impose a certain course of action upon us, given that we are rational actors
\end{itemize}
\section*{Expected Profit Khan}
\label{sec:org8afd0bd}
\begin{itemize}
\item Expected value can be calculated as the sum of all the outcome probabilities multiplied by their corresponding profits.
\item Considering all outcome probabilities should yeild a total probability sum of 1 (100\%), with profits being positive (gains) or negative (losses)
\end{itemize}
\section*{01.15.21}
\label{sec:org7ac888e}
\begin{itemize}
\item Brain has a complex set of structures that work together to do both really important, and fundamentally flawed actions
\end{itemize}
\subsection*{Brain Stem}
\label{sec:orgf9de9d0}
\begin{itemize}
\item The reptilian brain
\item Really just an extensino of the spinal cord
\item Controls automatic actions, no effect on decision-making
\end{itemize}
\subsection*{Middle Brain (Limbic System)}
\label{sec:org9e9dfed}
\begin{itemize}
\item Body's monitoring system to identify important elements of the environment
\item Discriminates things of importantance constantly and ambiently
\end{itemize}
\subsection*{Brain Cortex}
\label{sec:org11d85ac}
\begin{itemize}
\item Controls higher-level thinking
\item Moral decision-making, learning, conscious awareness
\end{itemize}
\subsection*{Hierarchy of the brain}
\label{sec:org4ad0083}
\begin{itemize}
\item Information goes from the brain stem, to the limbic system, to the brain cortex
\item Critically, the limbic system was never designed to collect all the information around you - that incomplete information is used for decision-making
\item Understanding the interplay and potential biases of the limbic system can help us understand decision-making and prevent bad decision making
\item Fear and the triggering of fear prevents higher-level decision making and can prevent the intake of new information
\item Sources of information can also have a significant effect on the processing of that information - can be seen through in-group/out-group bias
\end{itemize}
\subsection*{Rational Decision-Making}
\label{sec:orga31999d}
\begin{enumerate}
\item Pick a goal
\item Evaluate all strategies
\begin{itemize}
\item Analyze costs
\item Analyze benefits
\end{itemize}
\item Select strategy with best cost/benefit ratio
\item Bias often occurs at stage 2 because of filtration of information through the limbic system
\end{enumerate}

\section*{Notes Config}
\label{sec:org857aa53}
\end{document}
