% Created 2021-01-24 Sun 17:44
% Intended LaTeX compiler: pdflatex
\documentclass[11pt]{article}
\usepackage[utf8]{inputenc}
\usepackage[T1]{fontenc}
\usepackage{graphicx}
\usepackage{grffile}
\usepackage{longtable}
\usepackage{wrapfig}
\usepackage{rotating}
\usepackage[normalem]{ulem}
\usepackage{amsmath}
\usepackage{textcomp}
\usepackage{amssymb}
\usepackage{capt-of}
\usepackage{hyperref}
\author{Sudhan Chitgopkar}
\date{\today}
\title{}
\hypersetup{
 pdfauthor={Sudhan Chitgopkar},
 pdftitle={},
 pdfkeywords={},
 pdfsubject={},
 pdfcreator={Emacs 27.1 (Org mode 9.5)}, 
 pdflang={English}}
\begin{document}

\tableofcontents \clearpage\section{Mod 1 (Reader)}
\label{sec:orge3fae22}
\subsection{Overview}
\label{sec:org70fa73e}
\begin{itemize}
\item Since 1951, UN Refugee Convention defines refugees as ``persons fleeing racial, ethnonational, religious, or political persecution in their state of origin.''
\item Existing international law does not acknowledge climate refugees
\end{itemize}
\subsection{UN Context}
\label{sec:orgf1a6659}
\subsubsection{Concert of Europe}
\label{sec:orgbaa1d3f}
\begin{itemize}
\item After the fall of France in 1815, European powers form a multilateral forum, known as the Concert of Europe
\item Was effective because of Britain and its insurance of peace by balancing alliances using its strong navy
\item ended with the imperial drive towards Africa
\end{itemize}
\subsubsection{Functionalism \& Technocrats}
\label{sec:org0ecf25b}
\begin{itemize}
\item Industrial revolution led to increased international cooperation among technical expers (Technocrats)
\item Cooperation led to the standardization of many important things including railroad track width, shipping protocols, and signs
\item This is generally referred to as functionalism
\end{itemize}
\subsubsection{League of Nations}
\label{sec:org333b3ca}
\begin{itemize}
\item League of Nations was drafted during WWI and created after the war as a way of maintaining some sort of international forum.
\item Created significant tension with Germany because of the war guilt clause and the effect it had on Germany's international standing
\item Americans didn't like the LoN and ultimately didn't join it, leading to the ineffectiveness of the body
\end{itemize}
\subsubsection{The UN}
\label{sec:orgddc2b35}
\begin{itemize}
\item UN created at the end of WWII to promote peace and international security
\item Like the LoN, was founded by the victors of war and excluded Axis powers
\end{itemize}
\subsection{Situation Background}
\label{sec:org825b2b9}
\begin{itemize}
\item Religious Persecution - Persecution is widespread and perhaps the oldest form of persecution.
\item Political eprsecution - increased in the modern era due to conflict of political ideologies
\begin{itemize}
\item Eg. opponents to communism after the Russian Revolution, Spanish loyalists, Cuban and Hatian refugees in the US post-WWII
\end{itemize}
\item LoN achieves first steps to granting protection to political refugees and asylum
\item Refugee convention
\begin{itemize}
\item Standardized policies on refugees critical after WWII and the holocaust, which caused the displacement of a significant amount of jews. Eventually led to the Refugee Convention of 1951
\end{itemize}
\end{itemize}
\section{Mod 1 (Survey)}
\label{sec:org3eddb61}
\begin{itemize}
\item I would define a refugee as any person with a demonstrated need to relocate from their original place of residence due to hazardous, life-threatening conditions
\item Someone may need to flee from an area impacted by the effects of climate change for a variety of reasons. Firstly, climate change has significant, documented effects on crop productivity and cultivation potential. Accordingly, severe effects of climate change may prevent food and water from being accessible to a location where it once was. Furthermore, extreme weather events occur disproportionately more in areas that are significantly affected by climate change, leading to hazardous living conditions for people in the area.
\item I believe that a country is no more obligated to allow climate refugees to obtain asylum than refugees fleeing political, religious, or racial persecution. That being said, I do think that countries have a moral obligation to provide any assistance possible to refugees, including asylum, when doing so does not significantly hurt their citizens.
\end{itemize}
\end{document}
